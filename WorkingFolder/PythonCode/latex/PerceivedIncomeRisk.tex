\documentclass[12pt,notitlepage,onecolumn,aps,pra]{article}


    
\usepackage[T1]{fontenc}
\usepackage{graphicx}
% We will generate all images so they have a width \maxwidth. This means
% that they will get their normal width if they fit onto the page, but
% are scaled down if they would overflow the margins.
\makeatletter
\def\maxwidth{\ifdim\Gin@nat@width>\linewidth\linewidth
\else\Gin@nat@width\fi}
\makeatother
\let\Oldincludegraphics\includegraphics
% Set max figure width to be 80% of text width, for now hardcoded.
\renewcommand{\includegraphics}[1]{\Oldincludegraphics[width=.8\maxwidth]{#1}}
% Ensure that by default, figures have no caption (until we provide a
% proper Figure object with a Caption API and a way to capture that
% in the conversion process - todo).
\usepackage{caption}
\usepackage{adjustbox} % Used to constrain images to a maximum size
\usepackage{xcolor} % Allow colors to be defined
\usepackage{enumerate} % Needed for markdown enumerations to work
\usepackage{geometry} % Used to adjust the document margins
\usepackage{amsmath} % Equations
\usepackage{amssymb} % Equations
\usepackage{textcomp} % defines textquotesingle
% Hack from http://tex.stackexchange.com/a/47451/13684:
\AtBeginDocument{%
    \def\PYZsq{\textquotesingle}% Upright quotes in Pygmentized code
}
\usepackage{upquote} % Upright quotes for verbatim code
\usepackage{eurosym} % defines \euro
\usepackage[mathletters]{ucs} % Extended unicode (utf-8) support
\usepackage[utf8x]{inputenc} % Allow utf-8 characters in the tex document
\usepackage{fancyvrb} % verbatim replacement that allows latex
\usepackage{grffile} % extends the file name processing of package graphics
                     % to support a larger range
% The hyperref package gives us a pdf with properly built
% internal navigation ('pdf bookmarks' for the table of contents,
% internal cross-reference links, web links for URLs, etc.)
\usepackage{hyperref}
\usepackage{natbib}
\usepackage{booktabs}  % table support for pandoc > 1.12.2
\usepackage[inline]{enumitem} % IRkernel/repr support (it uses the enumerate* environment)
\usepackage[normalem]{ulem} % ulem is needed to support strikethroughs (\sout)
                            % normalem makes italics be italics, not underlines
\usepackage{braket}

\usepackage{rotating}
\usepackage{threeparttable}
\usepackage{subcaption}
\usepackage[capposition=top]{floatrow}


    
    % Colors for the hyperref package
    \definecolor{urlcolor}{rgb}{0,.145,.698}
    \definecolor{linkcolor}{rgb}{.71,0.21,0.01}
    \definecolor{citecolor}{rgb}{.12,.54,.11}

    % ANSI colors
    \definecolor{ansi-black}{HTML}{3E424D}
    \definecolor{ansi-black-intense}{HTML}{282C36}
    \definecolor{ansi-red}{HTML}{E75C58}
    \definecolor{ansi-red-intense}{HTML}{B22B31}
    \definecolor{ansi-green}{HTML}{00A250}
    \definecolor{ansi-green-intense}{HTML}{007427}
    \definecolor{ansi-yellow}{HTML}{DDB62B}
    \definecolor{ansi-yellow-intense}{HTML}{B27D12}
    \definecolor{ansi-blue}{HTML}{208FFB}
    \definecolor{ansi-blue-intense}{HTML}{0065CA}
    \definecolor{ansi-magenta}{HTML}{D160C4}
    \definecolor{ansi-magenta-intense}{HTML}{A03196}
    \definecolor{ansi-cyan}{HTML}{60C6C8}
    \definecolor{ansi-cyan-intense}{HTML}{258F8F}
    \definecolor{ansi-white}{HTML}{C5C1B4}
    \definecolor{ansi-white-intense}{HTML}{A1A6B2}
    \definecolor{ansi-default-inverse-fg}{HTML}{FFFFFF}
    \definecolor{ansi-default-inverse-bg}{HTML}{000000}

    % commands and environments needed by pandoc snippets
    % extracted from the output of `pandoc -s`
    \providecommand{\tightlist}{%
      \setlength{\itemsep}{0pt}\setlength{\parskip}{0pt}}
    \DefineVerbatimEnvironment{Highlighting}{Verbatim}{commandchars=\\\{\}}
    % Add ',fontsize=\small' for more characters per line
    \newenvironment{Shaded}{}{}
    \newcommand{\KeywordTok}[1]{\textcolor[rgb]{0.00,0.44,0.13}{\textbf{{#1}}}}
    \newcommand{\DataTypeTok}[1]{\textcolor[rgb]{0.56,0.13,0.00}{{#1}}}
    \newcommand{\DecValTok}[1]{\textcolor[rgb]{0.25,0.63,0.44}{{#1}}}
    \newcommand{\BaseNTok}[1]{\textcolor[rgb]{0.25,0.63,0.44}{{#1}}}
    \newcommand{\FloatTok}[1]{\textcolor[rgb]{0.25,0.63,0.44}{{#1}}}
    \newcommand{\CharTok}[1]{\textcolor[rgb]{0.25,0.44,0.63}{{#1}}}
    \newcommand{\StringTok}[1]{\textcolor[rgb]{0.25,0.44,0.63}{{#1}}}
    \newcommand{\CommentTok}[1]{\textcolor[rgb]{0.38,0.63,0.69}{\textit{{#1}}}}
    \newcommand{\OtherTok}[1]{\textcolor[rgb]{0.00,0.44,0.13}{{#1}}}
    \newcommand{\AlertTok}[1]{\textcolor[rgb]{1.00,0.00,0.00}{\textbf{{#1}}}}
    \newcommand{\FunctionTok}[1]{\textcolor[rgb]{0.02,0.16,0.49}{{#1}}}
    \newcommand{\RegionMarkerTok}[1]{{#1}}
    \newcommand{\ErrorTok}[1]{\textcolor[rgb]{1.00,0.00,0.00}{\textbf{{#1}}}}
    \newcommand{\NormalTok}[1]{{#1}}
    
    % Additional commands for more recent versions of Pandoc
    \newcommand{\ConstantTok}[1]{\textcolor[rgb]{0.53,0.00,0.00}{{#1}}}
    \newcommand{\SpecialCharTok}[1]{\textcolor[rgb]{0.25,0.44,0.63}{{#1}}}
    \newcommand{\VerbatimStringTok}[1]{\textcolor[rgb]{0.25,0.44,0.63}{{#1}}}
    \newcommand{\SpecialStringTok}[1]{\textcolor[rgb]{0.73,0.40,0.53}{{#1}}}
    \newcommand{\ImportTok}[1]{{#1}}
    \newcommand{\DocumentationTok}[1]{\textcolor[rgb]{0.73,0.13,0.13}{\textit{{#1}}}}
    \newcommand{\AnnotationTok}[1]{\textcolor[rgb]{0.38,0.63,0.69}{\textbf{\textit{{#1}}}}}
    \newcommand{\CommentVarTok}[1]{\textcolor[rgb]{0.38,0.63,0.69}{\textbf{\textit{{#1}}}}}
    \newcommand{\VariableTok}[1]{\textcolor[rgb]{0.10,0.09,0.49}{{#1}}}
    \newcommand{\ControlFlowTok}[1]{\textcolor[rgb]{0.00,0.44,0.13}{\textbf{{#1}}}}
    \newcommand{\OperatorTok}[1]{\textcolor[rgb]{0.40,0.40,0.40}{{#1}}}
    \newcommand{\BuiltInTok}[1]{{#1}}
    \newcommand{\ExtensionTok}[1]{{#1}}
    \newcommand{\PreprocessorTok}[1]{\textcolor[rgb]{0.74,0.48,0.00}{{#1}}}
    \newcommand{\AttributeTok}[1]{\textcolor[rgb]{0.49,0.56,0.16}{{#1}}}
    \newcommand{\InformationTok}[1]{\textcolor[rgb]{0.38,0.63,0.69}{\textbf{\textit{{#1}}}}}
    \newcommand{\WarningTok}[1]{\textcolor[rgb]{0.38,0.63,0.69}{\textbf{\textit{{#1}}}}}
    
    
    % Define a nice break command that doesn't care if a line doesn't already
    % exist.
    \def\br{\hspace*{\fill} \\* }
    % Math Jax compatibility definitions
    \def\gt{>}
    \def\lt{<}
    \let\Oldtex\TeX
    \let\Oldlatex\LaTeX
    \renewcommand{\TeX}{\textrm{\Oldtex}}
    \renewcommand{\LaTeX}{\textrm{\Oldlatex}}
    % Document parameters
    % Document title
    
    
    
    
% Pygments definitions
\makeatletter
\def\PY@reset{\let\PY@it=\relax \let\PY@bf=\relax%
    \let\PY@ul=\relax \let\PY@tc=\relax%
    \let\PY@bc=\relax \let\PY@ff=\relax}
\def\PY@tok#1{\csname PY@tok@#1\endcsname}
\def\PY@toks#1+{\ifx\relax#1\empty\else%
    \PY@tok{#1}\expandafter\PY@toks\fi}
\def\PY@do#1{\PY@bc{\PY@tc{\PY@ul{%
    \PY@it{\PY@bf{\PY@ff{#1}}}}}}}
\def\PY#1#2{\PY@reset\PY@toks#1+\relax+\PY@do{#2}}

\expandafter\def\csname PY@tok@w\endcsname{\def\PY@tc##1{\textcolor[rgb]{0.73,0.73,0.73}{##1}}}
\expandafter\def\csname PY@tok@c\endcsname{\let\PY@it=\textit\def\PY@tc##1{\textcolor[rgb]{0.25,0.50,0.50}{##1}}}
\expandafter\def\csname PY@tok@cp\endcsname{\def\PY@tc##1{\textcolor[rgb]{0.74,0.48,0.00}{##1}}}
\expandafter\def\csname PY@tok@k\endcsname{\let\PY@bf=\textbf\def\PY@tc##1{\textcolor[rgb]{0.00,0.50,0.00}{##1}}}
\expandafter\def\csname PY@tok@kp\endcsname{\def\PY@tc##1{\textcolor[rgb]{0.00,0.50,0.00}{##1}}}
\expandafter\def\csname PY@tok@kt\endcsname{\def\PY@tc##1{\textcolor[rgb]{0.69,0.00,0.25}{##1}}}
\expandafter\def\csname PY@tok@o\endcsname{\def\PY@tc##1{\textcolor[rgb]{0.40,0.40,0.40}{##1}}}
\expandafter\def\csname PY@tok@ow\endcsname{\let\PY@bf=\textbf\def\PY@tc##1{\textcolor[rgb]{0.67,0.13,1.00}{##1}}}
\expandafter\def\csname PY@tok@nb\endcsname{\def\PY@tc##1{\textcolor[rgb]{0.00,0.50,0.00}{##1}}}
\expandafter\def\csname PY@tok@nf\endcsname{\def\PY@tc##1{\textcolor[rgb]{0.00,0.00,1.00}{##1}}}
\expandafter\def\csname PY@tok@nc\endcsname{\let\PY@bf=\textbf\def\PY@tc##1{\textcolor[rgb]{0.00,0.00,1.00}{##1}}}
\expandafter\def\csname PY@tok@nn\endcsname{\let\PY@bf=\textbf\def\PY@tc##1{\textcolor[rgb]{0.00,0.00,1.00}{##1}}}
\expandafter\def\csname PY@tok@ne\endcsname{\let\PY@bf=\textbf\def\PY@tc##1{\textcolor[rgb]{0.82,0.25,0.23}{##1}}}
\expandafter\def\csname PY@tok@nv\endcsname{\def\PY@tc##1{\textcolor[rgb]{0.10,0.09,0.49}{##1}}}
\expandafter\def\csname PY@tok@no\endcsname{\def\PY@tc##1{\textcolor[rgb]{0.53,0.00,0.00}{##1}}}
\expandafter\def\csname PY@tok@nl\endcsname{\def\PY@tc##1{\textcolor[rgb]{0.63,0.63,0.00}{##1}}}
\expandafter\def\csname PY@tok@ni\endcsname{\let\PY@bf=\textbf\def\PY@tc##1{\textcolor[rgb]{0.60,0.60,0.60}{##1}}}
\expandafter\def\csname PY@tok@na\endcsname{\def\PY@tc##1{\textcolor[rgb]{0.49,0.56,0.16}{##1}}}
\expandafter\def\csname PY@tok@nt\endcsname{\let\PY@bf=\textbf\def\PY@tc##1{\textcolor[rgb]{0.00,0.50,0.00}{##1}}}
\expandafter\def\csname PY@tok@nd\endcsname{\def\PY@tc##1{\textcolor[rgb]{0.67,0.13,1.00}{##1}}}
\expandafter\def\csname PY@tok@s\endcsname{\def\PY@tc##1{\textcolor[rgb]{0.73,0.13,0.13}{##1}}}
\expandafter\def\csname PY@tok@sd\endcsname{\let\PY@it=\textit\def\PY@tc##1{\textcolor[rgb]{0.73,0.13,0.13}{##1}}}
\expandafter\def\csname PY@tok@si\endcsname{\let\PY@bf=\textbf\def\PY@tc##1{\textcolor[rgb]{0.73,0.40,0.53}{##1}}}
\expandafter\def\csname PY@tok@se\endcsname{\let\PY@bf=\textbf\def\PY@tc##1{\textcolor[rgb]{0.73,0.40,0.13}{##1}}}
\expandafter\def\csname PY@tok@sr\endcsname{\def\PY@tc##1{\textcolor[rgb]{0.73,0.40,0.53}{##1}}}
\expandafter\def\csname PY@tok@ss\endcsname{\def\PY@tc##1{\textcolor[rgb]{0.10,0.09,0.49}{##1}}}
\expandafter\def\csname PY@tok@sx\endcsname{\def\PY@tc##1{\textcolor[rgb]{0.00,0.50,0.00}{##1}}}
\expandafter\def\csname PY@tok@m\endcsname{\def\PY@tc##1{\textcolor[rgb]{0.40,0.40,0.40}{##1}}}
\expandafter\def\csname PY@tok@gh\endcsname{\let\PY@bf=\textbf\def\PY@tc##1{\textcolor[rgb]{0.00,0.00,0.50}{##1}}}
\expandafter\def\csname PY@tok@gu\endcsname{\let\PY@bf=\textbf\def\PY@tc##1{\textcolor[rgb]{0.50,0.00,0.50}{##1}}}
\expandafter\def\csname PY@tok@gd\endcsname{\def\PY@tc##1{\textcolor[rgb]{0.63,0.00,0.00}{##1}}}
\expandafter\def\csname PY@tok@gi\endcsname{\def\PY@tc##1{\textcolor[rgb]{0.00,0.63,0.00}{##1}}}
\expandafter\def\csname PY@tok@gr\endcsname{\def\PY@tc##1{\textcolor[rgb]{1.00,0.00,0.00}{##1}}}
\expandafter\def\csname PY@tok@ge\endcsname{\let\PY@it=\textit}
\expandafter\def\csname PY@tok@gs\endcsname{\let\PY@bf=\textbf}
\expandafter\def\csname PY@tok@gp\endcsname{\let\PY@bf=\textbf\def\PY@tc##1{\textcolor[rgb]{0.00,0.00,0.50}{##1}}}
\expandafter\def\csname PY@tok@go\endcsname{\def\PY@tc##1{\textcolor[rgb]{0.53,0.53,0.53}{##1}}}
\expandafter\def\csname PY@tok@gt\endcsname{\def\PY@tc##1{\textcolor[rgb]{0.00,0.27,0.87}{##1}}}
\expandafter\def\csname PY@tok@err\endcsname{\def\PY@bc##1{\setlength{\fboxsep}{0pt}\fcolorbox[rgb]{1.00,0.00,0.00}{1,1,1}{\strut ##1}}}
\expandafter\def\csname PY@tok@kc\endcsname{\let\PY@bf=\textbf\def\PY@tc##1{\textcolor[rgb]{0.00,0.50,0.00}{##1}}}
\expandafter\def\csname PY@tok@kd\endcsname{\let\PY@bf=\textbf\def\PY@tc##1{\textcolor[rgb]{0.00,0.50,0.00}{##1}}}
\expandafter\def\csname PY@tok@kn\endcsname{\let\PY@bf=\textbf\def\PY@tc##1{\textcolor[rgb]{0.00,0.50,0.00}{##1}}}
\expandafter\def\csname PY@tok@kr\endcsname{\let\PY@bf=\textbf\def\PY@tc##1{\textcolor[rgb]{0.00,0.50,0.00}{##1}}}
\expandafter\def\csname PY@tok@bp\endcsname{\def\PY@tc##1{\textcolor[rgb]{0.00,0.50,0.00}{##1}}}
\expandafter\def\csname PY@tok@fm\endcsname{\def\PY@tc##1{\textcolor[rgb]{0.00,0.00,1.00}{##1}}}
\expandafter\def\csname PY@tok@vc\endcsname{\def\PY@tc##1{\textcolor[rgb]{0.10,0.09,0.49}{##1}}}
\expandafter\def\csname PY@tok@vg\endcsname{\def\PY@tc##1{\textcolor[rgb]{0.10,0.09,0.49}{##1}}}
\expandafter\def\csname PY@tok@vi\endcsname{\def\PY@tc##1{\textcolor[rgb]{0.10,0.09,0.49}{##1}}}
\expandafter\def\csname PY@tok@vm\endcsname{\def\PY@tc##1{\textcolor[rgb]{0.10,0.09,0.49}{##1}}}
\expandafter\def\csname PY@tok@sa\endcsname{\def\PY@tc##1{\textcolor[rgb]{0.73,0.13,0.13}{##1}}}
\expandafter\def\csname PY@tok@sb\endcsname{\def\PY@tc##1{\textcolor[rgb]{0.73,0.13,0.13}{##1}}}
\expandafter\def\csname PY@tok@sc\endcsname{\def\PY@tc##1{\textcolor[rgb]{0.73,0.13,0.13}{##1}}}
\expandafter\def\csname PY@tok@dl\endcsname{\def\PY@tc##1{\textcolor[rgb]{0.73,0.13,0.13}{##1}}}
\expandafter\def\csname PY@tok@s2\endcsname{\def\PY@tc##1{\textcolor[rgb]{0.73,0.13,0.13}{##1}}}
\expandafter\def\csname PY@tok@sh\endcsname{\def\PY@tc##1{\textcolor[rgb]{0.73,0.13,0.13}{##1}}}
\expandafter\def\csname PY@tok@s1\endcsname{\def\PY@tc##1{\textcolor[rgb]{0.73,0.13,0.13}{##1}}}
\expandafter\def\csname PY@tok@mb\endcsname{\def\PY@tc##1{\textcolor[rgb]{0.40,0.40,0.40}{##1}}}
\expandafter\def\csname PY@tok@mf\endcsname{\def\PY@tc##1{\textcolor[rgb]{0.40,0.40,0.40}{##1}}}
\expandafter\def\csname PY@tok@mh\endcsname{\def\PY@tc##1{\textcolor[rgb]{0.40,0.40,0.40}{##1}}}
\expandafter\def\csname PY@tok@mi\endcsname{\def\PY@tc##1{\textcolor[rgb]{0.40,0.40,0.40}{##1}}}
\expandafter\def\csname PY@tok@il\endcsname{\def\PY@tc##1{\textcolor[rgb]{0.40,0.40,0.40}{##1}}}
\expandafter\def\csname PY@tok@mo\endcsname{\def\PY@tc##1{\textcolor[rgb]{0.40,0.40,0.40}{##1}}}
\expandafter\def\csname PY@tok@ch\endcsname{\let\PY@it=\textit\def\PY@tc##1{\textcolor[rgb]{0.25,0.50,0.50}{##1}}}
\expandafter\def\csname PY@tok@cm\endcsname{\let\PY@it=\textit\def\PY@tc##1{\textcolor[rgb]{0.25,0.50,0.50}{##1}}}
\expandafter\def\csname PY@tok@cpf\endcsname{\let\PY@it=\textit\def\PY@tc##1{\textcolor[rgb]{0.25,0.50,0.50}{##1}}}
\expandafter\def\csname PY@tok@c1\endcsname{\let\PY@it=\textit\def\PY@tc##1{\textcolor[rgb]{0.25,0.50,0.50}{##1}}}
\expandafter\def\csname PY@tok@cs\endcsname{\let\PY@it=\textit\def\PY@tc##1{\textcolor[rgb]{0.25,0.50,0.50}{##1}}}

\def\PYZbs{\char`\\}
\def\PYZus{\char`\_}
\def\PYZob{\char`\{}
\def\PYZcb{\char`\}}
\def\PYZca{\char`\^}
\def\PYZam{\char`\&}
\def\PYZlt{\char`\<}
\def\PYZgt{\char`\>}
\def\PYZsh{\char`\#}
\def\PYZpc{\char`\%}
\def\PYZdl{\char`\$}
\def\PYZhy{\char`\-}
\def\PYZsq{\char`\'}
\def\PYZdq{\char`\"}
\def\PYZti{\char`\~}
% for compatibility with earlier versions
\def\PYZat{@}
\def\PYZlb{[}
\def\PYZrb{]}
\makeatother


    % For linebreaks inside Verbatim environment from package fancyvrb. 
    \makeatletter
        \newbox\Wrappedcontinuationbox 
        \newbox\Wrappedvisiblespacebox 
        \newcommand*\Wrappedvisiblespace {\textcolor{red}{\textvisiblespace}} 
        \newcommand*\Wrappedcontinuationsymbol {\textcolor{red}{\llap{\tiny$\m@th\hookrightarrow$}}} 
        \newcommand*\Wrappedcontinuationindent {3ex } 
        \newcommand*\Wrappedafterbreak {\kern\Wrappedcontinuationindent\copy\Wrappedcontinuationbox} 
        % Take advantage of the already applied Pygments mark-up to insert 
        % potential linebreaks for TeX processing. 
        %        {, <, #, %, $, ' and ": go to next line. 
        %        _, }, ^, &, >, - and ~: stay at end of broken line. 
        % Use of \textquotesingle for straight quote. 
        \newcommand*\Wrappedbreaksatspecials {% 
            \def\PYGZus{\discretionary{\char`\_}{\Wrappedafterbreak}{\char`\_}}% 
            \def\PYGZob{\discretionary{}{\Wrappedafterbreak\char`\{}{\char`\{}}% 
            \def\PYGZcb{\discretionary{\char`\}}{\Wrappedafterbreak}{\char`\}}}% 
            \def\PYGZca{\discretionary{\char`\^}{\Wrappedafterbreak}{\char`\^}}% 
            \def\PYGZam{\discretionary{\char`\&}{\Wrappedafterbreak}{\char`\&}}% 
            \def\PYGZlt{\discretionary{}{\Wrappedafterbreak\char`\<}{\char`\<}}% 
            \def\PYGZgt{\discretionary{\char`\>}{\Wrappedafterbreak}{\char`\>}}% 
            \def\PYGZsh{\discretionary{}{\Wrappedafterbreak\char`\#}{\char`\#}}% 
            \def\PYGZpc{\discretionary{}{\Wrappedafterbreak\char`\%}{\char`\%}}% 
            \def\PYGZdl{\discretionary{}{\Wrappedafterbreak\char`\$}{\char`\$}}% 
            \def\PYGZhy{\discretionary{\char`\-}{\Wrappedafterbreak}{\char`\-}}% 
            \def\PYGZsq{\discretionary{}{\Wrappedafterbreak\textquotesingle}{\textquotesingle}}% 
            \def\PYGZdq{\discretionary{}{\Wrappedafterbreak\char`\"}{\char`\"}}% 
            \def\PYGZti{\discretionary{\char`\~}{\Wrappedafterbreak}{\char`\~}}% 
        } 
        % Some characters . , ; ? ! / are not pygmentized. 
        % This macro makes them "active" and they will insert potential linebreaks 
        \newcommand*\Wrappedbreaksatpunct {% 
            \lccode`\~`\.\lowercase{\def~}{\discretionary{\hbox{\char`\.}}{\Wrappedafterbreak}{\hbox{\char`\.}}}% 
            \lccode`\~`\,\lowercase{\def~}{\discretionary{\hbox{\char`\,}}{\Wrappedafterbreak}{\hbox{\char`\,}}}% 
            \lccode`\~`\;\lowercase{\def~}{\discretionary{\hbox{\char`\;}}{\Wrappedafterbreak}{\hbox{\char`\;}}}% 
            \lccode`\~`\:\lowercase{\def~}{\discretionary{\hbox{\char`\:}}{\Wrappedafterbreak}{\hbox{\char`\:}}}% 
            \lccode`\~`\?\lowercase{\def~}{\discretionary{\hbox{\char`\?}}{\Wrappedafterbreak}{\hbox{\char`\?}}}% 
            \lccode`\~`\!\lowercase{\def~}{\discretionary{\hbox{\char`\!}}{\Wrappedafterbreak}{\hbox{\char`\!}}}% 
            \lccode`\~`\/\lowercase{\def~}{\discretionary{\hbox{\char`\/}}{\Wrappedafterbreak}{\hbox{\char`\/}}}% 
            \catcode`\.\active
            \catcode`\,\active 
            \catcode`\;\active
            \catcode`\:\active
            \catcode`\?\active
            \catcode`\!\active
            \catcode`\/\active 
            \lccode`\~`\~ 	
        }
    \makeatother

    \let\OriginalVerbatim=\Verbatim
    \makeatletter
    \renewcommand{\Verbatim}[1][1]{%
        %\parskip\z@skip
        \sbox\Wrappedcontinuationbox {\Wrappedcontinuationsymbol}%
        \sbox\Wrappedvisiblespacebox {\FV@SetupFont\Wrappedvisiblespace}%
        \def\FancyVerbFormatLine ##1{\hsize\linewidth
            \vtop{\raggedright\hyphenpenalty\z@\exhyphenpenalty\z@
                \doublehyphendemerits\z@\finalhyphendemerits\z@
                \strut ##1\strut}%
        }%
        % If the linebreak is at a space, the latter will be displayed as visible
        % space at end of first line, and a continuation symbol starts next line.
        % Stretch/shrink are however usually zero for typewriter font.
        \def\FV@Space {%
            \nobreak\hskip\z@ plus\fontdimen3\font minus\fontdimen4\font
            \discretionary{\copy\Wrappedvisiblespacebox}{\Wrappedafterbreak}
            {\kern\fontdimen2\font}%
        }%
        
        % Allow breaks at special characters using \PYG... macros.
        \Wrappedbreaksatspecials
        % Breaks at punctuation characters . , ; ? ! and / need catcode=\active 	
        \OriginalVerbatim[#1,codes*=\Wrappedbreaksatpunct]%
    }
    \makeatother

    % Exact colors from NB
    \definecolor{incolor}{HTML}{303F9F}
    \definecolor{outcolor}{HTML}{D84315}
    \definecolor{cellborder}{HTML}{CFCFCF}
    \definecolor{cellbackground}{HTML}{F7F7F7}
    
    % prompt
    \makeatletter
    \newcommand{\boxspacing}{\kern\kvtcb@left@rule\kern\kvtcb@boxsep}
    \makeatother
    \newcommand{\prompt}[4]{
        \ttfamily\llap{{\color{#2}[#3]:\hspace{3pt}#4}}\vspace{-\baselineskip}
    }
    

    
    % Prevent overflowing lines due to hard-to-break entities
    \sloppy 
    % Setup hyperref package
    \hypersetup{
      breaklinks=true,  % so long urls are correctly broken across lines
      colorlinks=true,
      urlcolor=urlcolor,
      linkcolor=linkcolor,
      citecolor=citecolor,
      }
    % Slightly bigger margins than the latex defaults
    
    \geometry{verbose,tmargin=1in,bmargin=1in,lmargin=1in,rmargin=1in}
    
    

\begin{document}
    
    \title{Perceived Income Risks and Subjective Attribution}\author{Tao Wang \thanks{Johns Hopkins University, twang80@jhu.edu. I thank Chris Carroll, Jonathan Wright, Robert Moffitt, Edmund Crawley, Corina Boar, Yueran Ma, and participants of the behavioral economics conference at Yale SOM for useful comments.}}

\date{\today \\(Incomplete Draft)}
\maketitle\begin{abstract}Heterogenous-agent models under uninsured risks typically assume that agents have a perfect understanding of the size and nature of income risks. This paper explores the implications of the imperfect understanding. I first document the following empirical patterns of subjective risk profiles utilizing a density income survey. First, cohorts that have experienced higher income volatility both at individual and aggregate levels perceive higher-income risks. Second, perceptions of risks countercyclically react to recent realizations and past experiences of macro labor markets. Third, earners who are younger, from low-income households, and low numeracy have higher perceived risks. Fourth, higher perceived risks decrease the readiness to spend via precautionary saving motives. Theoretically, these empirical patterns can be reconciled by a model of learning from past experiences of income realizations based on subjective attribution of the nature of income shocks. By introducing attribution errors due to the psychological tendency to external (internal) attribution for bad (good) outcomes, the model provides a consistent explanation for the non-monotonic income-profile of risk perceptions and its countercyclical dynamics. It may also explain some emerging evidence on income-dependent patterns of inequality/fairness perceptions.\end{abstract}


    
    

    
    \hypertarget{introduction}{%
\section{Introduction}\label{introduction}}

Income risks matter for both individual behaviors and aggregate
outcomes. With identical expected income and homogeneous risk
preferences, different degrees of risks lead to different
saving/consumption and portfolio choices. This is well understood in
models in which agents are inter-temporally risk-averse, or prudent
(\cite{kimball1990precautionary}, \cite{carroll2001liquidity}), and the
risks associated with future marginal utility motivate precautionary
motives. Since it is widely known from the empirical research that
idiosyncratic income risks are at most partially insured
(\cite{blundell_consumption_2008}) or because of borrowing constraints,
such behavioral regularities equipped with market incompleteness leads
to ex-post unequal wealth distribution and different degrees of marginal
propensity to consume (MPC) (\cite{huggett1993risk},
\cite{aiyagari1994uninsured}). This has important implications for the
transmission of macroeconomic
policies\footnote{\cite{krueger2016macroeconomics}, \cite{kaplan2018monetary}, \cite{auclert2019monetary}, \cite{bayer2019precautionary}.}

One important assumption prevailing in macroeconomic models with
uninsured risks is that agents have a perfect understanding of the
income risks. Under the assumption, economists typically estimate the
income process based on micro income data and then treat the estimates
as the true model parameters known by the agents making decisions in the
model\footnote{For example, \cite{krueger2016macroeconomics}, \cite{bayer2019precautionary}.}.
But given the mounting evidence that people form expectations in ways
deviating from full-information rationality, leading to perennial
heterogeneity in economic expectations held by micro agents, this
assumption seems to be too stringent. To the extent that agents make
decisions based on their \emph{respective} perceptions, understanding
the \emph{perceived} income risk profile and its correlation structure
with other macro variables are the keys to explaining their behavior
patterns.

The contributions of this paper are both theoretical and empirical. In
the theory, it constructs a subjective model of learning that features
agents' imperfect understanding of the size as well as the nature of
income risks. In terms of the size, the imperfect understanding is
modeled as a lack of knowledge of the true parameter of an assumed
income process. Agents form their best guess about the parameters by
learning from the past experienced income of their own as well as
others. Such experience-based learning mechanisms engenders the
perceived risks to be dependent on age, generation, and macroeconomic
history. At the same time, the imperfect understanding of the nature of
risks is captured by assuming that individuals do not understand if the
income risks are commonly shared aggregate shock or idiosyncratic ones.
They learn about the model based on a subjective determination of the
nature of the past shocks, which is called \emph{attribution} in this
paper. The role of attribution in explaining and predicting has been
long developed in social pyschology
literature\footnote{See \cite{heider1958psychology}, \cite{kelley1967attribution}, and \cite{fiske1991social}.}.
This paper adapts the idea into a learning framework used by economists.
With different subjective attributions, agents arrive at different
degrees of model parameter uncertainty, thus different perceptions about
income risks.

This general framework allows me to explore the implications of
different attributions. For instance, I show that a higher degree of
external attribution, i.e.~perceiving the shocks to be common ones
instead of idiosyncratic ones, leads to higher risk perceptions. This
introduces a well-specified channel through which some imperfect
understanding of the nature of shocks leads to differences in risk
perceptions. The intuition for this is straightforward. As an
econometrician would perfectly understand, learning comes from
variations in the sample either cross-sectionally or over time. As
agents subjectively perceive the correlation of her own income shocks
and others' to be higher, the cross-sectional variation from the sample
useful to the learning is reduced, which leads to higehr uncertainty
associated with the parameter estimate. I show such a mechanism is
generalizable to different assumed income processes such as AR(1) or one
with permanent/transitory components of time-varying risks.

Incorporating attribution in learning allows it possible to explore the
implications of possible mischaracterization of experienced shocks.
Among various possible deviations, I explore a particular kind of
attribution error which is reminiscent of the ``self-serving bias'' in
the social psychology. In particular, it assumes that people have a
tendency to external (internal) attribution in the face of negative
(positive) experiences. By allowing the subjective correlation to be a
function of the recent experience such as income change, the model
neatly captures this psychological tendency. What is interesting is that
such a state-dependence of attribution in learning may help explain why
the average perceived risk is lower for the high-income groups than the
low-income ones. In the presence of aggregate risks, it also generates
counter-cyclical patterns of the average perceived risks, i.e.~bad times
are associated with high subjective risks. In addition, such a mechanism
of asymmetric attribution may also help explain perception patterns by
people at different income distributions about their relative position
in the society. For instance, based on matched administrative records of
income and perception surveys in Danmark, \cite{hvidberg2020social}
found that the high-income group tends to underestimate its rank in the
distribution while the low-income group is prone to overestimation. In
my model, this can be explained by the fact that high-income(low-income)
groups on average are prone to internal-attribution(external
attribution), thus perceiving higher (lower) ex-post income inequality,
leading to underperceived(overperceived) ranks. Drawing on another line
of findings in their paper, it suggests that income perceptions not only
affect micro/macro decisions, but also affect people' views about
fairness and redistribution policies. In that regard, my model could be
potentially useful in explaining differences in fairness attitudes.

Empirically, the paper sheds light on the perceptions of income risks by
utilizing the recently available density forecasts of labor income
surveyed by New York Fed's Survey of Consumer Expectation (SCE). What is
special about this survey is that agents are asked to provide
histogram-type forecasts of their earning growth over the next 12 months
together with a set of expectational questions about the macroeconomy.
When the individual density forecast is available, a parametric density
estimation can be made to obtain the individual-specific subjective
distribution. And higher moments reflecting the perceived income risks
such as variance, as well as the asymmetry of the distribution such as
skewness allow me to directly characterize the perceived risk profile
without relying on external estimates from cross-sectional microdata.
This provides the first-hand measured perceptions on income risks that
are truly relevant to individual decisions. Perceived income risks
exhibits a number of important patterns that are consistent with the
predictions of my model of experience-based learning with subjective
attribution.

\begin{itemize}
\item
  Higher experienced volatility is associated with higher perceived
  income risks. This helps explain why perceived risks differ
  systematically across different generations, who have experienced
  different histories of the income shocks. Besides, perceived risks
  declines with one's age.
\item
  Perceived income risks have a non-monotonic correlation with the
  current income, which can be best described as a skewed U shape.
  Perceived risk decreases with current income over the most range of
  income values follwed by an uppick in perceived risks for high-income
  group.
\item
  Perceived income risks are counter-cyclical with the labor market
  conditions or broadly business cycles. I found that average perceived
  income risks by U.S. earners are negatively correlated with the
  current labor market tightness measured by wage growth and
  unemployment rate. Besides, earners in states with higher unemployment
  rates and low wage growth also perceive income risks to be higher.
  This bears similarities to but important difference with a few
  previous studies that document the counter-cyclicality of income risks
  estimated by cross-sectional microdata (\cite{guvenen2014nature},
  \cite{catherine_countercyclical_2019}).
\item
  Perceived income risks translate into economic decisions in a way
  consistent with precautionary saving motives. In particular,
  households with higher income risk perceptions expect a higher growth
  in expenditure, i.e.~lower consumption today versus tomorrow.
\end{itemize}

These patterns suggest that individuals have a roughly good yet
imperfect understanding of their income risks. Good, in the sense that
subjective perceptions are broadly consistent with the realization of
cross-sectional income patterns. This is attained in my model because
agents learn from past experiences, roughly as econometricians do. In
contrast, subjective perceptions are imperfect in that bounded
rationality prevents people from knowing about the true income process
perfectly, which even hardworking economists equipped with different
advanced econometrical techniques and a larger sample of income data
cannot easily claim to have.

As illustrated by much empirical work of testing the rationality in
expectation formation, it is admittedly challenging to separately
account for the differences in perceptions driven by the ``truth'' and
the part driven by the pure subjective heterogeneity. The most
straightforward way seems to be to treat econometrician's external
estimates of the income process as the proxy to the truth, for which the
subjective surveys are compared. But this approach implicitly assumes
that econometricians correctly specify the model of the income process
and ignores the possible superior information that is available only to
the people in the sample but not to econometricians. The model built in
this paper reconciles both possibilities by modeling agents as boundedly
rational econometricians subject to model misspecification.

Finally, the subjective learning model will be incorporated into an
otherwise standard life-cycle consumption/saving model with uninsured
idiosyncratic and aggregate risks. Experience-based learning makes
income expectations and risks state-dependent when agents make
dynamically optimal decisions at each point of the time. In particular,
higher perceived risks will induce more precautionary saving behaviors.
If this perceived risk is state-dependent on recent income changes, it
will potentially shift the distribution of MPCs along income deciles,
therefore, amplify the channels aggregate demand responses to shocks.

\hypertarget{related-literature}{%
\subsection{Related literature}\label{related-literature}}

From both theoretical and empirical points of view, this paper is an
extension of experience-based learning that is developed to account for
how experiences shape people's economic expectations and subsequent
behaviors. This paper extends the framework in three directions. First,
building on the work that shows that experiences affect
expectations\footnote{A very close line of literature focuses on the impacts of experience on preferences, such as risk-taking \cite{malmendier2011depression}.},
such as inflation (\cite{malmendier2015learning}) risky asset return
(\cite{malmendier2019investor}), and housing prices
(\cite{kuchler2019personal}). I show in the same framework that
perceptions of second moments such as income risks can be influenced by
the experienced outcome as well as its volatility. This is confirmed by
the empirical evidence using the recently available density survey. This
contributes to the understanding of how experiences of first and second
moments shape perceptions in higher
moments\footnote{For similar evidence for the house price, see \cite{kuchler2019personal}.}.Second,
I introduce the subjective attribution into the framework to allow for
the possibility of an imperfect understanding of the nature of the
shocks. Finally, this paper tries to build such a belief-formation
mechanism into workhorse models of consumption/saving to explore its
macroeconomic implications.

Besides, this paper is relevant to four lines of literature. First, it
is related to an old but recently reviving interest in studying
consumption/saving behaviors in models incorporating imperfect
expectations and perceptions. For instance, the closest to the current
paper, \cite{pischke1995individual} explores the implications of the
incomplete information about aggregate/individual income innovations by
modeling agent's learning about inome component as a signal extraction
problem. \cite{wang2004precautionary} extends the framework to
incorporate precautionary saving motives. In a similar spirit,
\cite{carroll_sticky_2018} reconciles the low micro-MPC and high
macro-MPCs by introducing to the model an information rigidity of
households in learning about macro news while being updated about micro
news. \cite{rozsypal_overpersistence_2017} found that households'
expectation of income exhibits an over-persistent bias using both
expected and realized household income from Michigan household survey.
The paper also shows that incorporating such bias affects the aggregate
consumption function by distorting the cross-sectional distributions of
marginal propensity to consume(MPCs) across the population.
\cite{lian2019imperfect} shows that an imperfect perception of wealth
accounts for such phenomenon as excess sensitivity to current income and
higher MPCs out of wealth than current income and so forth. My paper has
a similar flavor to all of these works in that I also explore the
behavioral implications of households' perceptual imperfection. But it
has important two distinctions. First, this paper focuses on higher
moments such as income risks. Second, most of these existing work either
considers inattention of shocks or bias introduced by the model
parameter, none of these explores the possible misperception of the
nature of income shocks.
\footnote{For instance, \cite{pischke1995individual} assumes that agents know perfectly about the variance of permanent and transitory income so that they could filter the two components from observable income changes. This paper instead assumes that that the agents do not observe the two perfectly.}

Second, empirically, this paper also contributes to the literature
studying expectation formation using subjective surveys. There has been
a long list of ``irrational expectation'' theories developed in recent
decades on how agents deviate from full-information rationality
benchmark, such as sticky expectation, noisy signal extraction,
least-square learning, etc. Also, empirical work has been devoted to
testing these theories in a comparable manner (\cite{coibion2012can},
\cite{fuhrer2018intrinsic}). But it is fair to say that thus far,
relatively little work has been done on individual variables such as
labor income, which may well be more relevant to individual economic
decisions. Therefore, understanding expectation formation of the
individual variables, in particular, concerning both mean and higher
moments, will provide fruitful insights for macroeconomic modeling
assumptions.

Third, the paper is indirectly related to the research that advocated
for eliciting probabilistic questions measuring subjective uncertainty
in economic surveys (\cite{manski_measuring_2004},
\cite{delavande2011measuring}, \cite{manski_survey_2018}). Although the
initial suspicion concerning to people's ability in understanding, using
and answering probabilistic questions is understandable,
\cite{bertrand_people_2001} and other works have shown respondents have
the consistent ability and willingness to assign a probability (or
``percent chance'') to future events. \cite{armantier_overview_2017}
have a thorough discussion on designing, experimenting and implementing
the consumer expectation surveys to ensure the quality of the responses.
Broadly speaking, the advocates have argued that going beyond the
revealed preference approach, availability to survey data provides
economists with direct information on agents' expectations and helps
avoids imposing arbitrary assumptions. This insight holds for not only
point forecast but also and even more importantly, for uncertainty,
because for any economic decision made by a risk-averse agent, not only
the expectation but also the perceived risks matter a great deal.

Lastly, the idea of this paper echoes with an old problem in the
consumption insurance literature: `insurance or information'
(\cite{pistaferri_superior_2001},
\cite{kaufmann_disentangling_2009},\cite{meghir2011earnings}). In any
empirical tests of consumption insurance or consumption response to
income, there is always a worry that what is interpreted as the shock
has actually already entered the agents' information set or exactly the
opposite. For instance, the notion of excessive sensitivity, namely
households consumption highly responsive to anticipated income shock,
maybe simply because agents have not incorporated the recently realized
shocks that econometricians assume so (\cite{flavin_excess_1988}). Also,
recently, in the New York Fed
\href{https://libertystreeteconomics.newyorkfed.org/2017/11/understanding-permanent-and-temporary-income-shocks.html}{blog},
the authors followed a similar approach to decompose the permanent and
transitory shocks. My paper shares a similar spirit with these studies
in the sense that I try to tackle the identification problem in the same
approach: directly using the expectation data and explicitly controlling
what are truly conditional expectations of the agents making the
decision. This helps economists avoid making assumptions on what is
exactly in the agents' information set. What differentiates my work from
other authors is that I focus on higher moments, i.e.~income risks and
skewness by utilizing the recently available density forecasts of labor
income. Previous work only focuses on the sizes of the realized shocks
and estimates the variance of the shocks using cross-sectional
distribution, while my paper directly studies the individual specific
variance of these shocks perceived by different individuals.


    \hypertarget{theoretical-framework}{%
\section{Theoretical framework}\label{theoretical-framework}}

\hypertarget{income-process-and-risk-perceptions}{%
\subsection{Income process and risk
perceptions}\label{income-process-and-risk-perceptions}}

Log income of individual \(i\) from cohort \(c\) at time \(t\) follows
the following process (\cite{meghir2004income}). Cohort \(c\) represents
the year of entering the job market. It contains a predictable component
\(z\), and a stochastical component \(e\). The latter consists of
aggregate component \(g\), idiosyncratic permanent \(p\), MA(1)
component \(\eta\) and a transitory component \(\psi\). Here I do not
consider unemployment risk since the perceived risk measured in the
survey conditions on staying employed.

\begin{equation}
\begin{split}
& y_{i,c,t} = z_{i,c,t}+e_{i,c,t} \\
& e_{i,c,t}=g_t+p_{i,c,t}+\eta_{i,c,t}+\psi_{i,c,t}  \\
& g_t = g_{t-1} + \xi_{t} \\
& p_{i,c,t} = p_{i,c,t-1}+\theta_{i,c,t} \\
& \eta_{i,c,t} = \phi\epsilon_{i,c,t-1}+\epsilon_{i,c,t}
\end{split}
\end{equation}

All shocks including the aggregate \(\xi_t\), and idiosyncratic ones
follow normal distributions, with zero means and time-invariant
variances denoted as \(\sigma^2_{\xi}\),
\(\sigma^2_{\theta}\),\(\sigma^2_{\epsilon}\),\(\sigma^2_{\psi}\).
Hypothetically, these variances could differ both across cohorts and
time. I focus on the most simple case here and results with
cohort-specific income risks are reported in the appendix.

Income growth from \(t\) to \(t+1\) consists predictable changes in
\(z_{i,c,t+1}\), and those from realized income shocks.

\begin{equation}
\begin{split}
\Delta y_{i,c,t+1} & =  \Delta z_{i,c,t+1}+\Delta e_{i,c,t} \\
&= \Delta z_{i,c,t+1}+\xi_{t+1}+\theta_{i,c,t+1}+\epsilon_{i,c,t+1}+(\phi-1)\epsilon_{i,c,t}+\psi_{i,c,t+1}
\end{split}
\end{equation}

All shocks that have realized till \(t\) are observed by the agent at
time \(t\). Therefore, under full-information rational
expectation(FIRE), namely when the agent perfectly knows the income
process and parameters, the perceived income risks or the perceived
variance of income growth from \(t\) to \(t+1\) is equal to the
following.

\begin{equation}
Var_{t}^*(\Delta y_{i,c,t+1}) =Var_{t}^*(\Delta e_{i,c,t+1}) =   \sigma^2_\xi+\sigma^2_\theta + \sigma^2_\epsilon+\sigma^2_\psi 
\end{equation}

FIRE has a number of testable predictions about the behaviors of
perceived risks.

\begin{itemize}
\item
  First, agents who share the same income process have no disagreements
  on perceived risks. This can be checked by comparing
  within-cohort/group dispersion in perceived risks.
\item
  Second, the perceived risks under such an assumed income process are
  not dependent on past/recent income realizations. This can be tested
  by estimating the correlation between perceived risks and past income
  realizations or their proxies if the latter is not directly observed.
\item
  Third, under the assumed progress, the variances of different-natured
  shocks sum up to exactly the perceived income risks and the loadings
  of all components are all positive. I report detailed derivations and
  proofs in the Appendix that these predictions are robust to the
  alternative income process and the time-aggregation problem discussed
  in the literature. The latter arises when the underlying income
  process is set at a higher frequency than the observed frequency of
  reported income or income expectations. This will cause different
  sizes of loadings of all future shocks to perceived annual risk but
  does not change the positivity of the loadings from different
  components onto perceived risk.
\end{itemize}

The challenge of testing the third prediction is that the risk
parameters are not directly observable. Econometricians and modelers
usually estimate them relying upon cross-sectional moment information
from some panel data and take them as the model parameters understood
perfectly by the agents. I can therefore use econometricians' best
estimates using past income data as the FIRE benchmark (I will discuss
the concerns of this approach later). Assuming the unexplained income
residuals from this estimation regression is
\(\hat e_{i,t}= y_{i,c,t}-\hat z_{i,c,t}\)(\(\hat z_{i,c,t}\) is the
observable counterpart of \(z_{i,c,t}\) from data). The unconditional
cross-sectional variance of the change in residuals(equivalent to the
\texttt{income\ volatility\textquotesingle{}\textquotesingle{}\ or}instability'\,'
in the literature) is the following. Let's call this growth volatility.
It can be further decomposed into different components in order to get
the component-specific risk.

\begin{equation}
Var(\Delta \hat e_{i,c,t}) = \hat\sigma^2_\xi+\hat\sigma^2_\theta + ((1-\phi)^2+1)\hat\sigma^2_\epsilon+\hat\sigma^2_\psi 
\end{equation}

Notice the unconditional growth volatility overlaps with the FIRE
perceived risk in every component of the risks. But it is unambiguously
greater than the perceived risk under FIRE because econometricians' do
not directly observe the MA(1) shock from \(t-1\). But this suffices to
suggest that growth volatility is positively correlated with perceived
risk.

Corresponding to the growth volatility, let's also define the level
volatility as the cross-sectional variance of the levels of the
residuals, which is denoted by \(Var(\hat e_{i,t})\). Different from
growth volatility, it includes the cumulative volatility from all the
past permanent shocks as well as the MA shock from \(t-1\), all of which
are not correlated with the perceived risk under FIRE. Therefore, it any
it will have only a weak correlation with perceived risks under FIRE.

This suggests the fourth testable prediction stated as below. Since we
can obtain \(Var(\Delta \hat e_{i,c,t})\) and \(Var(\hat e_{i,c,t})\)
using past income data, we can test this prediction.

\begin{itemize}
\tightlist
\item
  Growth volatility and risk perceptions are positively correlated. In
  contrast, level volatility is only weakly correlated with perceived
  risks.
\end{itemize}

It is worth asking how sensitive this prediction is to possible model
misspecification of the income process in the first place. We consider
three most common issues in the literature regarding the income risks.

\begin{itemize}
\item
  \textbf{Permanent versus persistent shock}. Replacing the permanent
  shock to \(p\) with a persistent one in the above process essentially
  adds another component from the previous period to the growth in
  residuals, hence increases overall unconditional volatility. In the
  meantime, it does not change the perceived risk under FIRE. Therefore,
  the effect of making the permanent shock persistent will lead to a
  smaller correlation between FIRE risk perception and growth
  volatility. But the correlation will remain positive.
\item
  \textbf{Moving average shock or purely transitory shock}. Our model
  actually nests both cases. Setting \(\phi=0\) corresponds to purely
  transitory shocks. Any positive \(\phi\) allows the coexistence of MA
  shock and transitory shocks. Our test is also robust to this
  alternative specification.
\item
  \textbf{Time-invariant versus time-varying volatility}. Under the
  former assumption, the income volaility estimated from past income
  data can be directly comparable with the perceived risks reported for
  a different period. But doing so under the later assumption is
  inconsistent with the model. It requires the perceived risks and the
  realized income data are for the same time horizon. This is hard to
  satisfy based on the current data avaiability. But what's assuring is
  that if the stochastical volatilities are persistent over the time, we
  should still expect to see positive correlation between past
  volatility and FIRE risk perception even if the former is estimated
  from an earlier period.
\end{itemize}

There is another complication regarding the FIRE test: the superior
information problem. It states that what econometrician's treat as
income shocks are actually in the information set of the FIRE agents.
Think this as when the known characteristics \(\hat z\) used in the
regression only partially captures the true predictable components
\(z\). Hence the sample residuals \(\hat e\) are bigger than its true
counterparts and this results in higher estimated growth and level
volatility from data than the level relevant to FIRE agents in the
model. It is true that this leads to a lower correlation between
volatility and perceived risks, but it does not alter the prediction
about the positive correlation between the two.

    \hypertarget{data-variables-and-density-estimation}{%
\section{Data, variables and density
estimation}\label{data-variables-and-density-estimation}}

\hypertarget{data}{%
\subsection{Data}\label{data}}

The data used for this paper is from the core module of Survey of
Consumer Expectation(SCE) conducted by the New York Fed, a monthly
online survey for a rotating panel of around 1,300 household heads over
the period from June 2013 to January 2020, over a total of 80 months.
This makes about 95113 household-year observations, among which around
68361 observations provide non-empty answers to the density question on
earning growth.

Particularly relevant for my purpose, the questionnaire asks each
respondent to fill perceived probabilities of their same-job-hour
earning growth to pre-defined non-overlapping bins. The question is
framed as ``suppose that 12 months from now, you are working in the
exact same {[}``main'' if Q11\textgreater1{]} job at the same place you
currently work and working the exact same number of hours. In your view,
what would you say is the percentage chance that 12 months from now:
increased by x\% or more?''.

As a special feature of the online questionnaire, the survey only moves
on to the next question if the probabilities filled in all bins add up
to one. This ensures the basic probabilistic consistency of the answers
crucial for any further analysis. Besides, the earning growth
expectation is formed for exactly the same position, same hours, and the
same location. This has two important implications for my analysis.
First, these conditions help make sure the comparability of the answers
across time and also excludes the potential changes in earnings driven
by endogenous labor supply decisions, i.e.~working for longer hours.
Empirical work estimating income risks are often based on data from
received income in which voluntary labor supply changes are inevitably
included. Our subjective measure is not subject to this problem and this
is a great advantage. Second, the earning expectations and risks
measured here are only conditional on non-separation from the current
job. It excludes either unemployment, i.e.~likely a zero earning, or an
upward movement in the job ladder, i.e.~a different earning growth rate.
Therefore, this does not fully reflect the entire income risk profile
relevant to each individual.

Unemployment and other involuntary job separations are undoubtedly
important sources of income risks, but I choose to focus on the
same-job/hour earning with the recognition that individuals' income
expectations, if any, may be easier to be formed for the current
job/hour than when taking into account unemployment risks. Given the
focus of this paper being subjective perceptions, this serves as a
useful benchmark. What is more assuring is that the bias due to omission
of unemployment risk is unambiguous. We could interpret the moments of
same-job-hour earning growth as an upper bound for the level of growth
rate and a lower bound for the income risk. To put it in another way,
the expected earning growth conditional on current employment is higher
than the unconditional one, and the conditional earning risk is lower
than the unconditional one. At the same time, since SCE separately
elicits the perceived probability of losing the current job for each
respondent, I could adjust the measured labor income moments taking into
account the unemployment risk.

\hypertarget{density-estimation-and-variables}{%
\subsection{Density estimation and
variables}\label{density-estimation-and-variables}}

With the histogram answers for each individual in hand, I follow
\cite{engelberg_comparing_2009} to fit each of them with a parametric
distribution accordingly for three following cases. In the first case
when there are three or more intervals filled with positive
probabilities, it was fitted with a generalized beta distribution. In
particular, if there is no open-ended bin on the left or right, then
two-parameter beta distribution is sufficient. If there is either
open-ended bin with positive probability, since the lower bound or upper
bound of the support needs to be determined, a four-parameter beta
distribution is estimated. In the second case, in which there are
exactly two adjacent intervals with positive probabilities, it is fitted
with an isosceles triangular distribution. In the third case, if there
is only one positive-probability of interval only, i.e.~equal to one, it
is fitted with a uniform distribution.

Since subjective moments such as variance is calculated based on the
estimated distribution, it is important to make sure the estimation
assumptions of the density distribution do not mechanically distort my
cross-sectional patterns of the estimated moments. This is the most
obviously seen in the tail risk measure, skewness. The assumption of log
normality of income process, common in the literature (See again
\cite{blundell_consumption_2008}), implicitly assume zero skewness,
i.e.~that the income increase and decrease from its mean are equally
likely. This may not be the case in our surveyed density for many
individuals. In order to account for this possibility, the assumed
density distribution should be flexible enough to allow for different
shapes of subjective distribution. Beta distribution fits this purpose
well. Of course, in the case of uniform and isosceles triangular
distribution, the skewness is zero by default.

Since the microdata provided in the SCE website already includes the
estimated mean, variance and IQR by the staff economists following the
exact same approach, I directly use their estimates for these moments.
At the same time, for the measure of tail-risk, i.e.~skewness, as not
provided, I use my own estimates. I also confirm that my estimates and
theirs for the first two moments are correlated with a coefficient of
0.9.

For all the moment's estimates, there are inevitably extreme values.
This could be due to the idiosyncratic answers provided by the original
respondent, or some non-convergence of the numerical estimation program.
Therefore, for each moment of the analysis, I exclude top and bottom
\(3\%\) observations, leading to a sample size of around 48,000.

I also recognize what is really relevant to many economic decisions such
as consumption is real income instead of nominal income. I, therefore,
use the inflation expectation and inflation uncertainty (also estimated
from density question) to convert nominal earning growth moments to real
terms for some robustness checks in this paper. In particular, the real
earning growth rate is expected nominal growth minus inflation
expectation.

\begin{eqnarray}
\overline {\Delta y^{r}}_{i,t} = \overline\Delta y_{i,t} - \overline \pi_{i,t}
\end{eqnarray}

The variance associated with real earning growth, if we treat inflation
and nominal earning growth as two independent stochastic variables, is
equal to the summed variance of the two. The independence assumption is
admittedly an imperfect assumption because of the correlation of wage
growth and inflation at the macro level. So it is should be interpreted
with caution.

\begin{eqnarray}
\overline{var}_{i,t}(\Delta y^r_{i,t+1}) = \overline{var}_{i,t}(\Delta y_{i,t+1}) + \overline{var}_{i,t}(\pi_{t+1})
\end{eqnarray}

Not enough information is available for the same kind of transformation
of IQR and skewness from nominal to real, so I only use nominal
variables. Besides, as there are extreme values on inflation
expectations and uncertainty, I also exclude top and bottom \(5\%\) of
the observations. This further shrinks the sample, when using real
moments, to around 40,000.

    \hypertarget{perceived-income-risks-basic-facts}{%
\section{Perceived income risks: basic
facts}\label{perceived-income-risks-basic-facts}}

\hypertarget{cross-sectional-heterogeneity}{%
\subsection{Cross-sectional
heterogeneity}\label{cross-sectional-heterogeneity}}

This section inspects some basic cross-sectional patterns of the subject
moments of labor income. In the Figure \ref{fig:histmoms}, I plot the
distribution of perceived income risks in nominal and real terms,
\(\overline{var}_{i,t}\) and \(\overline{var^{r}}_{i,t}\), respectively.

There is a sizable dispersion in perceived income risks. In both nominal
and real terms, the distribution is right-skewed with a long tail.
Specifically, most of the workers have perceived a variance of nominal
earning growth ranging from zero to \(20\) (a standard-deviation
equivalence of \(4-4.5\%\) income growth a year). But in the tail, some
of the workers perceive risks to be as high as \(7-8\%\) standard
deviation a year. To have a better sense of how large the risk is,
consider a median individual in our sample, who has an expected earnings
growth of \(2.4\%\), and a perceived risk of \(1\%\) standard deviation.
This implies by no means negligible earning risk.
\footnote{In the appendix, I also include histograms of expected income growth and subjective skewness, which show intuitive patterns such as nominal rigidity. Besides, about half of the sample exhibits non-zero skewness in their subjective distribution, indicating asymmetric upper/lower tail risks.}

\begin{center}
[FIGURE \ref{fig:histmoms} HERE]
\end{center}

How are perceived income risks different across a variety of demographic
factors? Empirical estimates of income risks of different demographic
groups from microdata have been
rare\footnote{For instance, \cite{meghir2004income} estimated that high-education group is faced with higher income risks than the low-education group.  \cite{bloom2018great} documented that income risks decreases with age and varies with current income level in a U-shaped.},
not mentioning in subjective risk perceptions. Figure
\ref{fig:ts_incvar_age} plots the average perceived risks of young,
middle-aged, and old workers over the sample period. It is clear that
for most of the months, perceived risks decrease with age.
Hypothetically, this may be either because of more stable earning
dynamics as one is older in the market in reality, or a better grasp of
the true income process and higher subjective certainty. The model I
will build allows both to play a role.

\begin{center}
 [FIGURE \ref{fig:ts_incvar_age} HERE]
 \end{center}

Another important question is how income risk perceptions depend on the
realized income. This is unclear ex-ante because it depends on the true
income process as well as the perception formation. SCE does not
directly report the current earning by the individual who reports
earning forecasts. Instead, I use what's available in the survey, the
total pretax household income in the past year as a proxy to the past
realizations of labor income. As Figure \ref{fig:barplot_byinc} shows,
perceived risks gradually declines as one's household income increases
for most range of income. But the pattern reverses for the top income
group. Such a non-monotonic relationship between risk perceptions and
past realizations, as I will show later in the theoretical section, will
be reconciled by people's state-dependent attribution and learning.

\begin{center}
 [FIGURE \ref{fig:barplot_byinc} HERE]
\end{center}




    \hypertarget{counter-cyclicality-of-perceived-risk}{%
\subsection{Counter-cyclicality of perceived
risk}\label{counter-cyclicality-of-perceived-risk}}

Some studies have documented that income risks are counter-cyclical
based on cross-sectional income data.
\footnote{But they differ in exactly which moments of the income are counter-cyclical. For instance, \cite{storesletten2004cyclical} found that variances of income shocks are counter-cyclical, while \cite{guvenen2014nature} and \cite{catherine_countercyclical_2019}, in contrast, found it to be the left skewness.}
It is worth inspecting if the subjective income risk profile has a
similar pattern. Figure \ref{fig:tshe} plots the average perceived
income risks from SCE against the YoY growth of the average hourly wage
across the United States, which shows a clear negative correlation.
Table \ref{macro_corr_he} further confirms such a counter-cyclicality by
reporting the regression coefficients of different measures of average
risks on the wage rate of different lags. All coefficients are
significantly negative.

\begin{center}
[FIGURE \ref{fig:tshe} HERE]
\end{center}

\begin{center}
[TABLE \ref{macro_corr_he} HERE]
\end{center}

The pattern can be also seen at the state level. Table
\ref{macro_corr_he_state} reports the regression coefficients of the
monthly average perceived risk within each state on the state labor
market conditions, measured by either wage growth or the state-level
unemployment rate, respectively. It shows that a tighter labor market
(higher wage growth or a lower unemployment rate) is associated with
lower perceived income risks. Note that our sample stops in June 2019
thus not covering the outbreak of the pandemic in early 2020. The
counter-cyclicality will be very likely more salient if it includes the
current period, which was marked by catastrophic labor market
deterioration and increase market risks.

\begin{center}
[TABLE \ref{macro_corr_he_state} HERE]
\end{center}

The counter-cyclicality in subjective risk perceptions seen in the
survey may suggest the standard assumption of state-independent symmetry
in income shocks is questionable. But it may well be, alternatively,
because people's subjective reaction to the positive and negative shocks
are asymmetric even if the underlying process being symmetric. The model
to be constructed in the theoretical section explores the possible role
of both.




    \hypertarget{experiences-and-perceived-risk}{%
\subsection{Experiences and perceived
risk}\label{experiences-and-perceived-risk}}

Different generations also have different perceived income risks. Let us
explore to what extent the cohort-specific risk perceptions are
influenced by the income volatility experienced by that particular
cohort. Different cohorts usually have experienced distinct
macroeconomic and individual histories. On one hand, these non-identical
experiences could lead to long-lasting differences in realized life-long
outcomes. An example is that college graduates graduating during
recessions have lower life-long income than others.
(\cite{kahn2010long}, \cite{oreopoulos2012short},
\cite{schwandt2019unlucky}). On the other hand, experiences may have
also shaped people's expectations directly, leading to behavioral
heterogeneity across cohorts (\cite{malmendier2015learning}). Benefiting
from having direct access to the subjective income risk perceptions, I
could directly examine the relationship between experiences and
perceptions.

Individuals from each cohort are borned in the same year and obtained
the same level of their respective highest education. The experienced
volatility specific to a certain cohort \(c\) at a given time \(t\) can
be approximated as the average squared residuals from an income
regression based on the historical sample only available to the cohort's
life time. This is approximately the unexpected income changes of each
person in the sample. I use the labor income panel data from PSID to
estimate the income shocks.
\footnote{I obtain the labor income records of all household heads between 1970-2017. Farm workers, youth and olds and observations with empty entries of major demographic variables are dropped. }
In particular, I first undertake a Mincer-style regression using major
demographic variables as regressors, including age, age polynomials,
education, gender and time-fixed effect. Then, for each cohort-time
sample, the regression mean-squared error (RMSE) is used as the
approximate to the cohort/time-specific income volatility.

There are two issues associated with such an approximation of
experienced volatility. First, I, as an economist with PSID data in my
hand, am obviously equipped with a much larger sample than the sample
size facing an individual that may have entered her experience. Since
larger sample also results in a smaller RMSE, my approximation might be
smaller than the real experienced volatility. Second, however, the
counteracting effect comes from the superior information problem,
i.e.~the information set held by earners in the sample contains what is
not available to econometricians. Therefore, not all known factors
predictable by the individual are used as a regressor. This will bias
upward the estimated experienced volatility. Despite these concerns, my
method serves as a feasible approximation sufficient for my purpose
here.

The right figure in Figure \ref{fig:var_experience_data} plots the
(logged) average perceived risk from each cohort \(c\) at year \(t\)
against the (logged) experienced volatility estimated from above. It
shows a clear positive correlation between the two, which suggests that
cohorts who have experienced higher income volatility also perceived
future income to be riskier. The results are reconfirmed in Table
\ref{micro_reg}, for which I run a regression of logged perceived risks
of each individual in SCE on the logged experienced volatility specific
to her cohort while controlling individuals age, income, educations,
etc. What is interesting is that the coefficient of \(expvol\) declines
from 0.73 to 0.41 when controlling the age effect because that
variations in experienced volatility are indeed partly from age
differences. While controlling more individual factors, the effect of
the experienced volatility becomes even stronger. This implies potential
heterogeneity as to how experience was translated into perceived risks.

How does experienced income shock per se affect risk perceptions? We can
also explore the question by approximating experienced income growth as
the growth in unexplained residuals. As shown in the left figure of
Figure \ref{fig:var_experience_data}, it turns out that that a better
past labor market outcome experienced by the cohort is associated with
lower risk perceptions. This indicates that it is not not just the
volatility, but also the change in level of the income, that is
assymmetrically extrapolated into their perceiptions of risk.

\begin{center}
[FIGURE \ref{fig:var_experience_data} HERE]
\end{center}

In theory, individual income change is driven by both aggregate and
indiosyncratic risks. It is thus worth examining how experienced outcome
from the two respective source translate into risk perceptions
differently. In order to do so, we need to approximate idiosyncratic and
aggregate experiences, separately. The former is basically the
unexplained income residual from a regression controlling time fixed
effect and also time-education effect. Since the two effects pick up the
samplewide or groupwide common factors of each calender year, it
excludes aggregate income shocks. The difference between such a residual
and one from a regression dropping the two effects can be used to
approximate aggregate shocks. As an alternative measure of aggregate
economy, I use the official unemployment rate. For all aggregate
measures, the volatility is correspondingly computed as the variance
across time periods specific to each cohort.

Figure \ref{fig:experience_id_ag_data} plot income risk perceptions
against both aggregate and idiosyncratic experiences measured by the
level and the volatility of shocks. It suggests different patterns
between the aggregate and idiosyncratic experiences. In particular, a
positive aggregate shock (both indicated by a higher aggregate income
growth, or a lower unexmployment rate) is associated with lower risk
perceptions. Such a negative relationship seems to be non-existent at
the individual level. What's common between aggregate and idiosyncratic
risks is that the volatility of both kinds of experiences are positively
correlated with risk perceptions. Such correlations are confirmed in a
regression of controlling other individual characteristics, as shown in
Table \ref{micro_reg}. Individual volatility, aggregate volatility and
experience in unemployment rates are all significantly positively
correlated with income risk perceptions.

\begin{center}
[FIGURE \ref{fig:experience_id_ag_data} HERE]
\end{center}

As another dimension of the inquiry, one may wonder the effects from
experiences of income changes of different degree of persistance. In
particular, do experiences of volatility from permanent and transitory
shocks affect risk perceptions differently? In order to examine this
question, what we can do is to decompose the experienced income
volatility of different cohorts into components of different degree of
persistences and see how the they are loaded into the future
perceptions, separately. In particular, I follow the tradition of a long
list of labor/macro literature by assuming the unexplained earning to
consist of a permanent and a transitory component which have
time-varying volatilities
\footnote{\cite{gottschalk1994growth}, \cite{carroll1997nature}, \cite{meghir2004income}, etc.}.
Then relying upon the cross-sectional moment restrictions of income
changes, one could estimate the size of the permanent and transitory
income risks based on realized income data. Experienced permanent and
transitory volatility is approximated as the the average of estimated
risks of respective component from the year of birth of the cohort till
the year for which the income risk perceptions is formed.

In theory, both permanent and transitory risks increase the volatility
in income changes in the same way. But the results here suggest the
pattern only holds for transitory income risks, as shown in the Figure
\ref{fig:experience_var_per_tran_var_data}. In contrast, higher
experienced permanent volatility is associated with lower perceived
risk. I also confirm that the pattern is not sensitive to the particular
definition of the cohort here by alterantively letting people's income
risks be specific to education level. To understand which compoennt
overweights the other in determining overal risk perception given their
opposite signs, I also examine how the relative size of permanent and
transitory volatility affect income risk perceptions. The ratio
essentially reflect the degree of persistency of income changes,
i.e.~higher permanent/transitory risk ratio leads to a higher serial
correlation of unexpected income changes. The right graph in Figure
\ref{fig:experience_var_per_tran_var_data} suggest that higher
permanent/transitory ratio is actually associated with lower perceived
income risks. The fact that income risk perceptions differ depending
upon the nature of experienced volatility suggests that income risk
perceptions are formed in a way that is not consistent with the
underlying income process. This will be a crutial input to the modeling
of perception formation in the next section.

\begin{center}
[FIGURE \ref{fig:experience_var_per_tran_var_data} HERE]
\end{center}






    \hypertarget{other-individual-characteristics}{%
\subsection{Other individual
characteristics}\label{other-individual-characteristics}}

What other factors are associated with risk perceptions? This section
inspects the question by regressing the perceived income risks at the
individual level on four major blocks of variables: experiences,
demographics, unemployment expectations by the respondent, as well as
job-specific characteristics. The regression is specified as followed.

\begin{eqnarray}
\overline{risk}_{i,c,t} = \alpha + \beta_0 \textrm{HH}_{i,c,t} + \beta_1 \textrm{Exp}_{c,t} + \beta_2 \textrm{Prob}_{i,c,t} + \beta_3 \textrm{JobType}_{i,c,t} + \epsilon_{i,t}
\end{eqnarray}

The dependent variable is the individual \(i\) from cohort \(c\)'s
perceived risk. The experience block \(\textit{Exp}_{c,t}\) includes
individual experienced volatility \(\textit{IdExpVol}_{c,t}\), the
aggregate experience of volatility \(\textit{AgExpVol}_{c,t}\) and
experience of unemployment rate \(\textit{AgExpUE}_{c,t}\). They are all
cohort/time-specific since different birth cohort at different points of
time have had difference experience of both micro and macro histories.
The second type of factors denoted \(\textit{HH}_{i,t}\) represents
household-specific demographics such as the age, household income level,
education, gender as well as the numeracy of the respondent. In
particular, the numeracy score is generated based on the individual's
answers to seven questions that are designed to measure the individual's
basic knowlege of probability, intrest rate compounding, the difference
between nominal and real return and risk diversification. Third,
\(\textit{Prob}_{i,t}\) represents other subjective probabilities
regarding unemployment held by the same individual. As far as this paper
is concerned, I include the perceived probability of unemployment
herself and the probability of a higher nationwide unemployment rate.
The fourth block of factors, as called \(\textit{Jobtype}_{i,t}\)
includes dummy variables indicating if the job is part-time or if the
work is for others or self-employed.

Besides, since many of the regressors are time-invariant household
characteristics, I choose not to control household fixed effects in
these regressions (\(\omega_i\)). Throughout all specifications, I
cluster standard errors at the household level because of the concern of
unobservable household heterogeneity.

The regression results reported in Table \ref{micro_reg} are rather
intuitive. From the first to the sixth column, I gradually control more
factors. All specifications confirm that higher experienced volatility
at both idiosyncratic level and aggregate level, as well as high
unemployment rate experience in the past all lead to higher risk
perceptions. Besides, workers from low-income households, females, and
lower education and self-employed jobs have higher perceived income
risks.

In our sample, there are around \(15\%\) (6000) of the individuals who
report themselves to be self-employed instead of working for others. The
effects are statistically and economically significant. Whether a
part-time job is associated with higher perceived risk is ambiguous
depending on if we control household demographics. At first sight,
part-time jobs may be thought of as more unstable. But the exact nature
of part-time job varies across different types and populations. It is
possible, for instance, that the part-time jobs available to high-income
and educated workers bear lower risks than those by the low-income and
low-education groups.

Another interesting finding is that individual risk perception decreases
as the individual's numeracy test scores higher. This is not
particularly driven by the difference in education as the pattern
remains even if we jointly control for the education. This coroborates
the findings that individual's perception and decisions are affected by
the financial literacy (\cite{van2011financial},
\cite{lusardi2014economic}).

In addition, higher perceived the probability of losing the current job,
which I call individual unemployment risk, \(\textit{UEprobInd}\) is
associated with higher earning risks of the current job. The perceived
chance that the nationwide unemployment rate going up next year, which I
call aggregate unemployment risk, \(\textit{UEprobAgg}\) has a similar
correlation with perceived earning risks. Such a positive correlation is
important because this implies that a more comprehensively measured
income risk facing the individual that incorporates not only the current
job's earning risks but also the risk of unemployment is actually
higher. Moreover, the perceived risk is higher for those whose
perceptions of the earning risk and unemployment risk are more
correlated than those less correlated.

\begin{center}
 [TABLE \ref{micro_reg} HERE]
 \end{center}



    \hypertarget{perceived-income-risk-and-decisions}{%
\subsection{Perceived income risk and
decisions}\label{perceived-income-risk-and-decisions}}

Finally, how individual-specific perceived risks affect household
economic decisions such as consumption? The testable prediction is
higher perceived risks shall increase precautionary saving motive
therefore lower current consumption (higher consumption growth.)
Although we cannot directly observe the respondent's spending decisions,
we can alternatively rely on the self-reported spending plan in the SCE
to shed some light on this
\footnote{Other work that directly examines the impacts of expectations on readiness to spend includes \cite{bachmann2015inflation},\cite{coibion2020forward}.}.

Table \ref{spending_reg} reports the regression results of planned
spending growth over the next year on the expected earning's growth (the
first column) as well as a number of perceived income risk measures.
\footnote{There is one important econometric concern when I run regressions of the decision variable on perceived risks due to the measurement error in the regressor used here. In a typical OLS regression in which the regressor has i.i.d. measurement errors, the coefficient estimate for the imperfectly measured regressor will have a bias toward zero. For this reason, if I find that willingness to consume is indeed negatively correlated with perceived risks, taking into account the bias, it implies that the correlation of the two is greater in the magnitude.}
Each percentage point increase in expected income growth is associated
with a 0.39 percentage point increase in spending growth. At the same
time, one percentage point higher in the perceived risk increases the
planned spending growth by 0.58 percentage. This effect is even stronger
for real income risks. As a double-check, the individual's perceived
probability of a higher unemployment rate next year also has a similar
effect. These results suggest that individuals do exhibit precautionary
saving motives according to their own perceived risks.

\begin{center}
[TABLE \ref{spending_reg} HERE]
\end{center}



    \hypertarget{a-model-of-risk-perception-formation}{%
\section{A model of risk perception
formation}\label{a-model-of-risk-perception-formation}}

This section will show that the empirical patterns discussed above can
be reconciled by a model of learning featuring an imperfect
understanding of the income process. In particular, the model will
specify how experienced volatility is translated into future risk
perceptions and how the process also depends on people's perceived
nature of the income risks.

\hypertarget{under-a-simple-income-process}{%
\subsection{Under a simple income
process}\label{under-a-simple-income-process}}

We start by defining an AR(1) process of the individual income. In the
next section for a life-cycle consumption problem, we will extend the
model for a more realistic income process with risks of different
persistence, i.e.~permanent and transitory components. In particular,
the income of individual \(i\) from the cohort \(c\) at time \(t\)
depends on her previous-period income with a persistence parameter of
\(\rho\) and an individual and time-specific shock
\(\epsilon_{i,c,t}\)\footnote{There are usually predictable components of income by known individual characteristics. It is more accurate to think $y$ here as the unexplained income component.}.
I define cohort \(c\) to be measured by the year of entry in the job
market.

\begin{eqnarray}
y_{i,c,t} = \rho y_{i,c,t-1} + \epsilon_{i,c,t}
\end{eqnarray}

It is assumed that the \(\rho\) is the same across all inviduals. Also,
I assume the income shock \(\epsilon_{i,c,t}\) to be i.i.d., namely
independent across individuals and the time and with an identical
variance, as defined in the equation below. Later sections will relax
this assumption by allowing for cross-sectional correlation, namely some
aggregate risks. Further extensions are also allowed for cohort or
time-specific volatility. The i.i.d. assumption implies at any time
\(t\), the variance-covariance matrix of income shocks across
individuals is a diagonal matrix defined as below.

\begin{eqnarray}
E(\epsilon_{t}'\epsilon_{t}|Y_{t-1}) = \sigma^2 I_n \quad \forall t 
\end{eqnarray}

where \(\sigma^2\) is the volatility of income shock and \(I_n\) is an
identity matrix whose length is the number of agents in the economy,
\(n\). Although income volatility is not cohort-specific, any past shock
still leaves different impacts on the young and old generations because
of different experiences of histories. This is reminiscent of
\cite{bansal2004risks}. Since both \(\rho\) and \(\sigma^2\) are not
cohort-specific, I drop the subscript \(c\) from now on to avoid
clustering.

Both \(\rho\) and \(\sigma\) are the ``true'' parameters only known by
the modeler, but unknown by agents in the economy. Individual \(i\)
learns about the income process by ``running a regression'' of the form
laid out above using a small sample of her past experience starting from
the year of entering the job market \(c\) till \(t\). Critically, for
this paper's purpose, I allow the experience used for learning to
include both her own and others' past income over the same period. It is
admittedly bizarre to assume individual agents have full access to the
entire population's income. A more realistic assumption could be that
only a small cross-sectional sample is available to the agent. Any scope
of cross-sectional learning suffices for the point to be made in this
paper.

    If each agent knows \emph{perfectly} the model parameters \(\rho\) and
\(\sigma\), the uncertainty about future income growth is

\begin{eqnarray}
\begin{split}
Var^*_{i,t}(\Delta y_{i,t+1}) & =  Var^*_{i,t}(y_{i,t+1}- y_{i,t}) \\ 
& =  Var^*_{i,t}((\rho-1)y_{i,t} + \epsilon_{i,t+1}) \\
& = Var^*_{i,t}(\epsilon_{i,t+1}) \\
& = \sigma^2
\end{split}
\end{eqnarray}

The superscript \(*\) is the notation for perfect understanding. The
first equality follows because both \(y_{i,t}\) and the persistent
parameter \(\rho\) is known by the agent. The second follows because
\(\sigma^2\) is also known.

Under \emph{imperfect} understanding and learning, both \(\rho\) and
\(\sigma^2\) are unknown to agents. Therefore, the agent needs to learn
about the parameters from the small panel sample experienced up to that
point of the time. We represent the sample estimates of \(\rho\) and
\(\sigma^2\) using \(\widehat \rho\) and \(\hat{\sigma}^2\).

\begin{eqnarray}
\begin{split}
\widehat Var_{i,t}(\Delta y_{i,t+1}) & = y_{i,t}^2 \underbrace{\widehat{Var}^{\rho}_{i,t}}_{\text{Persistence uncertainty}} + \underbrace{\hat{\sigma}^2_{i,t}}_{\text{Shock uncertainty}}
\end{split}
\end{eqnarray}

The perceived risks of future income growth have two components. The
first one comes from the uncertainty about the persistence parameter. It
reflects how uncertain the agent feels about the degree to which
realized income shocks will affect her future income, which is
non-existent under perfect understanding. I will refer to this as the
parameter uncertainty or persistence uncertainty hereafter. Notice the
persistence uncertainty is scaled by the squared size of the
contemporary income. It implies that the income risks are size-dependent
under imperfect understanding. It introduces one of the possible
channels via which current income affects perceived income risk.

The second component of perceived risk has to do with the unrealized
shock itself. Hence, it can be called shock uncertainty. Because the
agent does not know perfectly the underlying volatility of the income
shock, she can only infer that from the past volatility.

We assume agents learn about the parameters using a least-square rule
widely used in the learning literature (For instance,
\cite{marcet1989convergence}, \cite{evans2012learning},
\cite{malmendier2015learning}) The bounded rationality prevents her from
adopting any more sophisticated rule that econometricians may consider
to be superior to the OLS in this context. (For instance, OLS applied in
autocorrelated models induce bias in estimate.) Under OLS learning, the
parameter estimate is the following.

\begin{eqnarray}
\hat \rho_{i,t} = (\sum^{t-c}_{k=0}\sum^{n}_{j=1}y^2_{j,t-k-1})^{-1}(\sum^{t-c}_{k=0}\sum^{n}_{j=1}y_{j,t-k-1}y_{j,t-k})
\end{eqnarray}

The sample variance of regression residuals \(\widehat e\), or the mean
squared errors (MSE), in econometrician's word, are the agents' best
guess of the income volatility \(\sigma^2\). It can be seen as the
experienced volatility over the past history.

\begin{eqnarray}
\widehat{\sigma}^2_{i,t} = s^2_{i,t} = \frac{1}{N_{i,t}-1} \sum^{n}_{j=1}\sum^{t-c}_{k=0} \hat e_{j,t-k}^2
\end{eqnarray}

where \(N_{i,t}\) is the size of the panel sample available to the agent
\(i\) at time t. It is equal to \(n_{i}(t-c_{i})\), the number of people
in the sample times the duration of agent \(i\)'s career.

We first consider the case when the agent understands that the income
shocks are i.i.d. To put it differently, this is when the agent
correctly specify the income model when learning. Under i.i.d.
assumption, the estimated uncertainty about the estimate is

\begin{eqnarray}
\widehat {Var}^{\rho}_{i,t} = (\sum^{t-c}_{k=0}\sum^{n}_{j=1}y^2_{j,t-k-1})^{-1}\widehat{\sigma}^2_{i,t}
\end{eqnarray}

Experience-based learning naturally introduces a mechanism for the
perceived income risks to be cohort-specific and age-specific. Different
generations who have experienced different realizations of the income
shocks have different estimates of \(Var^{\rho}\) and \(\sigma^2\), thus
differ in their uncertainty about future income. In the meantime, people
at an older age are faced with a larger sample size than younger ones,
this will drive the age profile of perceived risks in line with the
observation that the perceived risk is lower as one grows older. Also,
note that the learning literature has explored a wide variety of
assumptions on the gains from learning to decline over time or age.
These features can be easily incorporated into my framework. For now,
equal weighting of the past experience suffices for the exposition here.

We can rewrite the perceived risk under correct model specification as
the following.

\begin{eqnarray}
\widehat{Var}_{i,t}(\Delta y_{i,t+1}) = [(\sum^{t-c}_{k=0}\sum^{n}_{j=1}y^2_{j,t-k-1})^{-1}y^2_{i,t} + 1] \hat{\sigma}^2_{i,t}
\end{eqnarray}

    \hypertarget{attribution}{%
\subsection{Attribution}\label{attribution}}

Attribution means that agents subjectively determine the nature of the
income shocks, or equivalently, form perceptions about the correlation
between their own outcome and others. This opens a room for possible
model-misspecification about the nature of income shock due to bounded
rationality. Although people specify the form of the regression model
correctly, they do not necessarily perceive the nature of the income
shocks correctly.

Without taking stances on how exactly people might mis-attribute, it is
worth discussing the general implications of subjective attribution for
risk perceptions. For instance, would it imply a higher perceived income
risk if the subjective perception of the cross-sectional correlation is
higher?

Under the least-square estimation, the estimated parameter uncertainty
takes a general sandwich form as below. It is similar to accounting for
within-time clustering in computing standard errors by econometricians.
The main distinction is that the following is subjective.

\begin{eqnarray}
\begin{split}
\tilde {Var}^{\rho}_{i,t} & =   (\sum^{t-c}_{k=0}\sum^{n}_{j=1}y^2_{j,t-k-1})^{-1}(\sum^{t-c}_{k=0}\tilde \Omega_{i,t-k})(\sum^{t-c}_{k=0}\sum^{n}_{j=1}y^2_{j,t-k-1})^{-1}
\end{split}
\end{eqnarray}

where \(\tilde \Omega_{t-k}\) is the perceived variance-covariance of
income and income shocks within each point of time. It reflects how
individual \(i\) thinks about the correlation between her own income and
others'.

\begin{eqnarray}
\begin{split}
\tilde \Omega_{i,t} = \tilde E_{i,t}(Y_{t-1}e_{t}'e_{t}Y_{t-1})
\end{split}
\end{eqnarray}

In order to get more intuition, we can simplify the matrix in the
following way. If we assume constant group size \(n\) over time and the
homoscedasticity, i.e.~income risks \(\sigma\) do not change over time,
given the individual ascribes a subjective correlation coefficient of
\(\tilde \delta_{\epsilon, i,t}\) across income shocks and a correlation
\(\tilde \delta_{y, i,t}\) across the levels of income,
\(\tilde \Omega_{i,t}\) can be approximated as the following. (See the
appendix for derivation)
\footnote{This is analogous to the cluster-robust standard error by \cite{cameron2011robust}. But again, they are different because here it is subjective.}

\begin{eqnarray}
\begin{split}
\tilde \Omega_{i,t} & \approx \sum^{n}_{j=1}y^2_{j,t} (1+\tilde \delta_{y,i,t}\tilde \delta_{\epsilon,i,t}(n-1))\tilde \sigma^2_{t}
\end{split}
\end{eqnarray}

It follows that the parameter uncertainty under the subjective
attribution writes as below.

\begin{eqnarray}
\begin{split}
\tilde {Var}^{\rho}_{i,t} & = (\sum^{t-c}_{k=0}\sum^{n}_{j=1}y^2_{j,t-k-1})^{-1}(1+ \tilde\delta_{i,t}(n-1))\tilde{\sigma}^2_{t}
\end{split}
\end{eqnarray}

Notice we bundle the two correlation coefficients parameters together as
a single parameter of the attribution \(\tilde\delta_{i,t}\).

\begin{eqnarray}
\tilde \delta_{y,i,t}\tilde \delta_{\epsilon,i,t}\equiv \tilde \delta_{i,t}  
\end{eqnarray}

The subjective attribution is jointly determined by two perceived
correlation parameters, \(\tilde \delta_{\epsilon}\) and
\(\tilde \delta_y\). They can be more intuively thought as long-run
attribution and short-run attribution, respectively, because the former
is the perceived correlation in the level of the income and later in
shocks. The multiplication of two jointly governs the degree to which
the agents inflate experienced volatility in forming perceptions about
future income risks.

We can define internal and external attribution as the following.

\begin{eqnarray}
\begin{split}
\textrm{Internal attribution: }\quad \tilde\delta_{i,t} = 0 \\
\textrm{External attribution: }\quad \tilde\delta_{i,t} \in (0,1]
\end{split}
\end{eqnarray}

The internal attribution represents the scenario when the agent \(i\)
thinks that her income shock or the long-run income is uncorrelated with
others' (\(\tilde \delta_{\epsilon} = 0\) or \(\tilde \delta_y = 0\)).
In contrast, the external attribution stands for the case when the agent
perceive her own income to be positively correlated with others'. The
external attribution attains its maximum value of \(1\) if the agent
thinks both her income shock and income are perfectly correlated with
others. In general, \(\tilde \delta_{\epsilon,i,t}\) and
\(\tilde \delta_{y,i,t}\) are not necessarily consistent with the true
income process. Since long-run correlation increases with the short-run
correlation, bundling them together as a single parameter is feasible.

Now, it is clear that the subjective attribution affects perceived risks
through its effect on parameter uncertainty. One can show that a higher
degree of external attribution (a higher \(\tilde \delta_{i,t}\)) is
associated with higher parameter uncertainty as well as higher perceived
risks.

\begin{eqnarray}
\begin{split}
\tilde {Var}_{i,t}(\Delta y_{i,t+1}|\tilde \delta_{i,t}) >  \tilde {Var}_{i,t}(\Delta y_{i,t+1}|\tilde \delta'_{i,t}) \quad \forall \tilde \delta_{i,t} > \tilde \delta'_{i,t}
\end{split}
\end{eqnarray}

Taking stock, we have the following predictions about how the perceived
income risks depend on experienced volatility and subjective
attribution.

\begin{itemize}
\item
  Higher experienced volatility, measured by
  \(s^2 \equiv \tilde{\sigma}^2_{i,t}\) leads to higher perceived income
  risks.
\item
  In the same time, future perceptions of the risks inflate the past
  volatility proportionally depending on their subjective attribution. A
  higher degree of external attribution reflected by a higher
  \(\tilde \delta_{i,t}\) implies a higher parameter uncertainty (Figure
  \ref{fig:corr_var}) and higher inflation of past volatility into the
  future. (Figure \ref{fig:var_experience_var})
\end{itemize}



    \hypertarget{attribution-errors}{%
\subsection{Attribution errors}\label{attribution-errors}}

Now, let us explore some possible attribution errors. Being subjective
attribution, one can possibly think of many forms in which agents'
attribution is not in line with the true nature of income shocks. This
section explores one plausible possibility. In particular, we
incorporate the psychological tendency of ascribing bad experiences to
external causes and good experience to internal ones, which is known as
self-serving bias.
\footnote{See \cite{al1993attributional}, \cite{campbell1999self}, \cite{seidel2010blame}.}
The manifesto of the attribution error is that people asymmetrically
assign the subjective correlation \(\tilde \delta_{i,t}\) depending on
the recent income change (or the realized shocks) being positive or
negative.

More formally, define an attribution function that maps the recent
income change \(\Delta y_{i,t}\) into the subjective attribution
\(\tilde \delta_{i,t}\), according to the following form.
\begin{eqnarray}
\begin{split}
\tilde \delta(\Delta y_{i,t}) = 1- \frac{1}{(1+e^{\alpha-\theta \Delta y_{i,t}})}
\end{split}
\end{eqnarray}

Figure \ref{fig:theta_corr} plots the attribution function under
different parameter value. Basically, the attribution function is a
variant of a logistic function with its function value bounded between
\([0,1]\). It takes an s-shape and the parameter \(\theta\) governs the
steepness of the s-shape around its input value. The higher the value of
\(\theta\) is, the more prone to such an error. It takes any
non-negative value. \(\alpha\) is an adjustable parameter chosen such
that the attribution free of attribution errors happens to be equal to
the true correlation dictated by the underlying income process,
\(\delta\). Or it can be used to capture the degree of bias in
attribution, which is a separate mechanism from the attribution error
examined here.

Such a state-dependence of the attribution leads to systematically
higher perceived risks by the ``unlucky group'', i.e.~agents who have
experienced negative income shocks than the ``lucky group'' even if the
underlying income process does not have such a feature. It is important
to note that this difference still exists even if the underlying shocks
are indeed non-independent. Although different types of income shocks
have different implications as to which group correctly or mis-specifies
the model, it does not alter the distinction between the lucky and
unlucky group. To put it differently, the underlying process only
determines who is over-confident or under-confident. But the group with
a positive experience (thus internal distribution) is always more
certain about their future income than the negative-experienced group.

So far, we have maintained the assumption of i.i.d. shock. What if there
is indeed some aggregate risk? It turns out the presence of aggregate
risk will induce the counter-cyclical pattern of the average perceived
risk under such attribution errors. To see this clearly, I define at
each point of the time \(t\), there is \(\lambda_t\) fraction of the
agents that have experienced a positive income change, the ``lucky
group''. Then the average perceived risk across all agents is a weighted
average of the perceived risks between two groups.

\begin{eqnarray}
\begin{split}
\tilde {Var}_{t}(\Delta y_{i,t+1}) & = \underbrace{\lambda_t}_{\text{lucky fraction}} \tilde{Var_t}^{internal} + (1-\lambda_t) \tilde{Var_t}^{external} 
\end{split}
\end{eqnarray}

The counter-cyclicality of perceived risk is very straightforward in
such an environment. Imagine a positive aggregate shock at a given time,
it leads to a larger fraction of the people who have experienced
positive shocks, thus internal attribution and lowers perceived risks,
while a negative aggregate shock induces more people to externally
attribute and higher perceive risks.

Both the presence of attribution errors and aggregate risks are
necessary conditions for generating the counter-cyclicality. Under
aggregate risk, the fraction of lucky group \(\lambda_t\) is
procyclical, while under idiosyncratic risks, it is a constant \(0.5\).
With attribution error, the perceived risk under external attribution is
always higher than that of internal attribution,
i.e.~\(\tilde{Var_t}^{external} >\tilde{Var_t}^{internal}\), while the
two are equal without attribution errors.

To summarize, introducing a particular form of attribution errors leads
to testable predictions.

\begin{itemize}
\item
  Perceived income risks are state-dependent, i.e.~the recent past
  income, even if the underlying income process assumes income risks are
  independent of the past. Because of the attribution errors, we will
  also see the perceived income risks will be systematically lower for
  the high-income group than the low-income group.
\item
  In the presence of aggregate risks and attribution error, the average
  perceived risks are counter-cyclical.
\end{itemize}



    \hypertarget{simulation}{%
\subsection{Simulation}\label{simulation}}

\hypertarget{current-income-and-perceived-risks}{%
\subsubsection{Current income and perceived
risks}\label{current-income-and-perceived-risks}}

How do perceived risks depend on the current income level of
\(y_{i,t}\)? Since the recent income changes \(\Delta y_{i,t}\) triggers
asymmetric attribution, the perceived risks depend on the current level
of income beyond the past-dependence of future income on current income
that is embodied in the AR(1) process. In particular,
\(\widehat{Var}^\rho_{i,t}\) does not depend on \(\Delta y_{i,t}\) while
\(\tilde{Var}^\rho_{i,t}\) does and is always greater than the former as
a positive, it will amplify the loading of the current level of income
into perceived risks about future income. This generates a U-shaped
perceived income profile depending on current level income.

Figure \ref{fig:var_recent} and \ref{fig:var_recent_sim} plots both the
theory-predicted and simulated correlation between \(y_{i,t}\) and
perceived income risks with/without attribution errors. In the former
scenario, perceived risks only mildly change with current income and the
entire income profile of perceived risk is approximately flat. In the
latter scenario, in contrast, perceived risks exhibit a clear U-shape
across the income distribution. People sitting at both ends of the
income distribution have high perceived risks than ones in the middle.
The non-monotonic of the income profile arise due to the combined
effects directly from \(y_{i,t}\) and indirectly via its impact on
\(\tilde Var^{\rho}\). The former effect is symmetric around the
long-run average of income (zero here). Deviations from the long-run
mean on both sides lead to higher perceived risk. The latter
monotonically decreases with current income because higher income level
is associated with a more positive income change recently. The two
effects combined create a U-shaped pattern.

A subtle but interesting point is that the U-shape is skewed toward
left, meaning perceived risks decrease with the income over the most
part of the income distribution before the pattern reverses. More
intuitively, it means that although low and high income perceived risks
to be higher because of its deviation from the its long-run mean. This
force is muted for the high income group because they have a lower
peceived risks due to the attribution errors.


    \hypertarget{age-and-experience-and-perceived-risks}{%
\subsubsection{Age and experience and perceived
risks}\label{age-and-experience-and-perceived-risks}}

Figure \ref{fig:var_age_sim} plots the simulated age profile of
perceived income risks with and without attribution errors. Due to
experience-based learning, older agents have a larger sample size when
learning about the model parameter, which induces the parameter
uncertainty to be lower. What is interesting is that with attribution
errors, since the parameter uncertainty is inflated proportionally with
the degree of attribution, it makes the negative correlation between the
perceived income risks and age more salient. This is consistent with our
empirical findings shown in Figure \ref{fig:ts_incvar_age}.



    \hypertarget{aggregate-risk-and-counter-cyclicality}{%
\subsubsection{Aggregate risk and
counter-cyclicality}\label{aggregate-risk-and-counter-cyclicality}}

Previously, I assume the underlying shock is i.i.d. This section
considers the implication of the attribution errors in the presence of
both aggregate and idiosyncratic risks. This can be modeled by assuming
that the shocks to individuals' income are positively correlated with
each other at each point of the time. Denoting \(\delta>0\) as the true
cross-sectional correlation of income shocks, the conditional
variance-covariance of income shocks within each period is the
following.

\begin{eqnarray}
\begin{split}
E(\epsilon_{t}'\epsilon_{t}|Y_{t-1}) = \Sigma^2 = \sigma^2\Omega \quad \forall t  
\end{split}
\end{eqnarray}

where \(\Omega\) takes one in its diagonal and \(\delta\) in
off-diagonal.

The learning process and the attribution errors all stay the same as
before. Individuals specify their subjective structure of the shocks
depending on the sign and size of their own experienced income changes.
By the same mechanism elaborated above, a lucky person has lower
perceived risks than her unlucky peer at any point of the time. This
distinction between the two group stays the same even if the underlying
income shocks are indeed correlated.

What's new in the presence of aggregate risks lies in the behaviors of
average perceived risks, because there is an aggregate shock that drives
the comovement of the income shocks affecting individuals. Compared to
the environment with pure idiosyncratic risks, there is no longer an
approximately equal fraction of lucky and unlucky agents at a given
time. Instead, the relative fraction of each group depends on the
recently realized aggregate shock. If the aggregate shock is positive,
more people have experienced good luck and may, therefore, underestimate
the correlation (a smaller \(\tilde \delta\)). This drives down the
average perceived income risks among the population. If the aggregate
shock is negative, more people have just experienced income decrease
thus arriving at a higher perceived income uncertainty.

This naturally leads to a counter-cyclical pattern of the average
perceived risks in the economy. The interplay of aggregate risks and
attribution errors adds cyclical movements of the average perceived
risks. The two conditions are both necessary to generate this pattern.
Without the aggregate risk, both income shocks and perceived income
shocks are purely idiosyncratic and they are averaged out in the
aggregate level. Without attribution errors, agents symmetrically
process experiences when forming future risk perceptions.

The upper panel in Figure \ref{fig:recent_change_var_sim} illustrates
the first point. The scatter plots showcase the correlation between
average income changes across population and average perceive risks
under purely idiosyncratic risks and aggregate risks. The negative
correlation with aggregate risks illustrate the counter-cylical
perceived risks. There is no such a correlation under purely
idiosyncratic risks. The bottom panel in Figure
\ref{fig:recent_change_var_sim} testifies the second point. It plots the
same correlation with and without attribution errors when the aggregate
risk exists. Attribution errors brings about the asymmetry not seen when
the bias is absent.



    \hypertarget{experience-based-life-cycle-consumptionsaving}{%
\section{Experience-based life-cycle
consumption/saving}\label{experience-based-life-cycle-consumptionsaving}}

Each consumer solves a life-cycle consumption/saving problem formulated
by \cite{gourinchas2002consumption}. There is only one deviation from
the original model: each agent imperfectly knows the parameters of
income process over the life cycle and forms his/her best guess at each
point of the time based on past experience. I first set up the model
under the assumption of perfect understanding and then extend it to the
imperfect understanding scenarior in the next section.

\hypertarget{the-standard-life-cycle-problem}{%
\subsection{The standard life-cycle
problem}\label{the-standard-life-cycle-problem}}

Each agent works for \(T\) periods since entering the labor market,
during which he/she earns stochastic labor income \(y_\tau\) at the
work-age of \(\tau\). After retiring at age of \(T+1\), the agent lives
for for another \(L-T\) periods of life. Since a cannonical life-cycle
problem is the same in nature regardless of the cohort and calender
time, we set up the problem generally along the age of work \(\tau\).
The consumer chooses the whole future consumption path to maximize
expected life-long utility.

\begin{equation}
\begin{split}
E\left[\sum^{\tau=L}_{\tau=1}\beta^\tau u(c_{\tau})\right] \\
u(c) = \frac{c^{1-\rho}}{1-\rho}
\end{split}
\end{equation}

where \(c_\tau\) represents consumption at the work-age of \(\tau\). The
felicity function \(u\) takes the standard CRRA form with relative risk
aversion of \(\rho\). We assume away the bequest motive and
preference-shifter along life cycle that are present in the original
model without loss of the key insights regarding income risks.

Denote the financial wealth at age of \(\tau\) as \(b_{\tau}\). Given
initial wealth \(b_1\), the consumer's problem is subject to the
borrowing constraint

\begin{equation}
b_{\tau}\geq 0
\end{equation}

and inter-temporal budget constraint.

\begin{equation}
\begin{split}
b_{\tau}+y_{\tau} = m_\tau   \\
b_{\tau+1} = (m_\tau-c_{\tau})R
\end{split}
\end{equation}

where \(m_\tau\) is the total cash in hand at the begining of period
\(\tau\). \(R\) is the risk-free interest rate. Note that after
retirement labor income is zero through the end of life.

The stochastic labor income during the agent's career consists of a
mulplicative predictable component by known factors \(Z_\tau\) and a
stochastic component \(\epsilon_{\tau}\) which embodies shocks of
different nature.

\begin{equation}
\begin{split}
y_{\tau} = \phi Z_{\tau}\epsilon_{\tau} 
\end{split}
\end{equation}

Notice here I explicitly include the predictable component, deviating
from the common practice in the literature. Although under perfect
understanding, the predictable component does not enter consumption
decision effectively since it is anticipated ex ante, this is no longer
so once we introduce imperfect understanding regarding the parameters of
the income process \(\phi\). The prediction uncertainty enters the
perception of income risks. We will return to this point in the next
section.

The stochastic shock to income \(\epsilon\) is composed of two
components: a permanent one \(p_t\) and a transitory one \(u\). The
former grows by a age-specific growth rate \(G\) along the life cycle
and is subject to a shock \(n\) at each period.

\begin{equation}
\begin{split}
\epsilon_{\tau} = p_{\tau}u_{\tau} \\
p_{\tau} = G_{\tau}p_{\tau-1} n_{\tau}
\end{split}
\end{equation}

The permanent shock \(n\) follows a log normal distribution,
\(ln(n_\tau) \sim N(0,\sigma^2_\tau)\). The transitory shock \(u\)
either takes value of zero with probability of \(0\leq p<1\),
i.e.~unemployment, or otherwise follows a log normal with
\(ln(u_\tau) \sim N(0,\sigma^2_u)\). Following
\cite{gourinchas2002consumption}, I assume the size of the volatility of
the two shocks are time-invariant. The results of this paper are not
sensitive to this assumption.

At this stage, we do not seek to differentiate the
aggregate/idiosyncratic components of either one the two enter the
individual consumption problem indifferently under perfect
understanding. With imperfect understanding and subjective attribution,
however, the differences of the two matters since it affects the
prediction uncertainty and perceived income risks.

The following value function characterizes the problem.

\begin{equation}
\begin{split}
V_{\tau}(m_\tau, p_\tau) = \textrm{max} \quad u(c_\tau) + \beta E_{\tau}\left[V_{\tau+1}(m_{\tau+1}, p_{\tau+1})\right] 
\end{split}
\end{equation}

where the agents treat total cash in hand and permanent income as the
two state variables. On the backgroud, the income process parameters
\(\Gamma = [\phi,\sigma_n,\sigma_u]\) affect the consumption decisions.
But to the extent that the agents have perfect knowledge of them, they
are simply taken as given.

\hypertarget{under-imperfect-understandinglearning-from-experience}{%
\subsection{Under imperfect understanding/learning from
experience}\label{under-imperfect-understandinglearning-from-experience}}

The crucial deviation of this model from the standard framework
reproduced above is that the agents do not know about the income
parameters \(\Gamma\), and the decisions are only based on their best
guess obtained through learning from experience in a manner we formulate
in the previous section. This changes the problem in at least two ways.
First, given agents potentially differ in their experiences, perceived
income processes differ. Second, even if under same experiences,
different subjective determinations of the nature of income shocks
result in different risk perceptions. To allow for the cross-sectional
heterogeneity across individuals and cohorts in income risk perceptions,
now explicitly define the problem using agent-time-cohort-specific value
function. For agent \(i\) from cohort \(c=t-\tau\) at time \(t\), the
value function is the following.

\begin{equation}
\begin{split}
V_{i,\tau,t}(m_{i,\tau,t}, p_{i,\tau,t}) = \textrm{max} \quad u(c_{i,\tau+1,t+1}) + \beta E_{i,\tau,t}\left[V_{i,\tau+1,t+1}(m_{i,\tau+1,t+1}, p_{i,\tau+1,t+1})\right] 
\end{split}
\end{equation}

Notice that the key difference of the new value function from the one
under a perfect understanding is that expectational operator of
next-period value function becomes subjective and potentially
agent-time-specific. Another way to put it is that \(E_{i,\tau,t}\) is
conditional on the most recent parameter estimate of the income process
\(\tilde \Gamma_{i,\tau,t} = \left[\tilde \phi_{i,\tau,t},\tilde \sigma_{n,i,\tau,t}, \tilde \sigma_{u,i,\tau,t}\right]\)
and the uncertainty about the estimate \(Var_{i,\tau,t}(\tilde \phi)\).

The perceived income risk affects the value of the expected value. It
implicitly contains two components. The first is the shock uncertainty
that can be predicted at best by the past income volatility estimation
of different components. The second is the parameter uncertainty
regarding the agents' estimation of a parameter associated with the
deterministic components \(\phi\). Since both components imply a further
dispersion from the perfect understanding case, it will unambiguously
induce a stronger precautionary saving motive than the latter case.

\hypertarget{consumption-functions}{%
\subsection{Consumption functions}\label{consumption-functions}}

I compare the life cycle consumption functions between

\begin{itemize}
\tightlist
\item
  perfectly understanding vs imperfect understanding
\item
  same age different experiences
\item
  under different digree of attribution
\end{itemize}

\hypertarget{implications-for-consumption-inequality}{%
\subsection{Implications for consumption
inequality}\label{implications-for-consumption-inequality}}

\begin{itemize}
\tightlist
\item
  consumption inequality(thus wealth inequality) and heterogeneity in
  MPCs now is further amplified by belief differences in income risks.
\end{itemize}

    \hypertarget{conclusion}{%
\section{Conclusion}\label{conclusion}}

How do people form perceptions about their income risks? Theoretically,
this paper builds an experience-based learning model that features an
imperfect understanding of the size of the risks as well as its nature.
By extending the learning-from-experience into a cross-sectional
setting, I introduce a mechanism in which future risk perceptions are
dependent upon past income volatility or cross-sectional distributions
of the income shocks. I also introduce a novel channel - subjective
attribution, into the learning to capture how income risk perceptions
are also affected by the subjective determination of the nature of
income risks. It is shown that the model generates a few testable
predictions about the relationship between experience/age/income and
perceived income risks.

Empirically, I utilize a recently available panel of income density
surveys of U.S. earners to shed light directly on subjective income risk
profiles. I explore the cross-sectional heterogeneity in income risk
perceptions across ages, generations, and income group as well as its
cyclicality with the current labor market outcome. I found that risk
perceptions are positively correlated with experienced income
volatility, therefore differing across age and cohorts. I also found
perceived income risks of earners counter-cyclically react to the recent
labor market conditions.

Finally, the paper builds the experience-based-learning and subjective
attribution into an otherwise standard life cycle model of consumption.
I show an imperfect understanding of the income process unambiguously
motivates additional precautionary saving than in a model of perfect
understanding. I also show that the consumption decisions of agents at
the same age may still differ as long as they have experienced different
histories at both individual and aggregate levels. Such belief
heterogeneity further amplifies the inequality in consumption and wealth
accumulation of different generations.

Many interesting questions are worth exploring although they are beyond
the scope of the paper. First, to what extent the model in this paper
could help account for the well-documented differences between
millennials and earlier generations in their saving behaviors,
homeownership, and stock market investment? Within the very short span
of their early career, millennials have had experienced two aggregate
economic catastrophes, namely the global financial crisis and the
pandemic. The evidence and model in this paper both suggest that this
may have persistent impacts on the new generations' risk perceptions,
thus economic decisions.

Second, what general equilibrium consequences would the life-cycle and
intergenerational differences in risk perceptions generate? It is true
that demographic compositions are slow-moving variables. But the gradual
change in the demographic structure of the economy may interact with the
different experienced macroeconomic histories, generating non-stationary
belief distributions across time. This will undoubtedly lead to a
different macroeconomic equilibrium.

Third, although this paper focuses on the size and nature of income
risks, the imperfect understanding could also and may very likely take
the form of misperceiving correlation between different random variables
relevant to economic decisions. A perfect example of this is the
correlation between income risks and stock market returns. The
subjective correlation between the two may shed light on participation
puzzles and equity premium in the macroeconomic finance literature.


                                                                                                                                                                                                                                                                                                                                                                                                                                                                                                       \newpage 
   
  \section*{Tables and Figures} 
    
    
    % figures 
    

        \begin{figure}[!ht]
        		\caption{Distribution of Individual Moments}
        	\label{fig:histmoms}
    	\begin{center}
    		\adjustimage{max size={0.4\linewidth}{0.3\paperheight}}{../../Graphs/ind/hist_incvar.jpg}
    		\adjustimage{max size={0.4\linewidth}{0.3\paperheight}}{../../Graphs/ind/hist_rincvar.jpg}    
\end{center}
    	\floatfoot{Note: this figure plots the perceived risks of nominal and real income, measured by the subjective variance of log income changes one year from the time of the survey. Real risk is the sum of the perceived risk of nominal income and inflation uncertainty.}
    \end{figure}
    
    \clearpage
    
    \begin{figure}[!ht]
    	\caption{Perceived Income by Age}
    	\label{fig:ts_incvar_age}
    	\begin{center}\adjustimage{max size={0.7\linewidth}}{../../Graphs/ind/ts/ts_incvar_age_g_mean.png}\end{center}
    	\floatfoot{Note: this figure plots average perceived income risks of different age groups over time.}
    \end{figure}

\clearpage
\begin{figure}[!ht]
	\caption{Realized and Perceived Age Profile of Income Risks}
	\label{fig:log_wage_shk_gr_by_age_compare}
	\begin{center}\adjustimage{max size={0.7\linewidth}}{../../Graphs/psid/log_wage_shk_gr_by_age_compare.png}\end{center}
	\floatfoot{Note: this figure plots average realized and perceived income risks of different ages. Realized income risks is defined as the age-specific income volatility estimated as the standard-deviation of change in unexplained income residuals from PSID income panel. Perceived income risk is obtained from SCE.}
\end{figure}
    
    \clearpage
    \begin{figure}[!ht]
    	\caption{Perceived Income by Income}
    	\label{fig:barplot_byinc}
    	\begin{center}\adjustimage{max size={0.7\linewidth}}{../../Graphs/ind/boxplot_var_HHinc_stata.png}\end{center}
    	\floatfoot{Note: this figure plots average perceived income risks by the range of household income.}
    \end{figure}
    
    \clearpage
    \begin{figure}[!ht]
      \caption{Recent Labor Market Outcome and Perceived Risks}
    \label{fig:tshe}
    	\begin{center}\adjustimage{max size={\linewidth}}{../../Graphs/pop/tsMean3mvvar_he.jpg}\end{center}
    	\floatfoot{Note: recent labor market outcome is measured by hourly wage growth (YoY). The 3-month moving average is plotted for both series.}
    \end{figure}

 \clearpage
\begin{figure}[!ht]
	\caption{Experience and Perceived Income Risk}
	\label{fig:var_experience_data}
	\begin{center}
		\adjustimage{max size={0.7\linewidth}{0.4\paperheight}}{../../Graphs/ind/experience_gr_var_data.png}
	\adjustimage{max size={0.7\linewidth}{0.4\paperheight}}{../../Graphs/ind/experience_var_var_data.png}
\end{center}
	\floatfoot{Note: experienced growth is the average growth of unexplained income residual and the experienced volatility is its cross-sectional variance within each corresponding cohort. The latter is essentially computed as the mean squred error(MSE) from an income regression on observable individual characteristics including age, age-squred, time, education and gender. The perceived income risk is the average across all individuals from the cohort in that year. Cohorts are time/year-of-birth/education-specific and all cohort sized 30 or smaller are excluded.}
\end{figure}

\clearpage
\begin{figure}[!ht]
	\caption{Experience and Perceived Income Risk: Permanent and Transitory}
	\label{fig:experience_var_per_tran_var_data}
	\begin{center}
		\adjustimage{max size={0.6\linewidth}{0.3\paperheight}}{../../Graphs/ind/experience_var_permanent_var_data.png}
		\adjustimage{max size={0.6\linewidth}{0.3\paperheight}}{../../Graphs/ind/experience_var_transitory_var_data.png}
		\adjustimage{max size={0.6\linewidth}{0.3\paperheight}}{../../Graphs/ind/experience_var_ratio_var_data.png}
\end{center}
	\floatfoot{Note: experienced permanent (transitory) volatility is average of the estimated risks of the permanent (transitory) component of a particular year-cohort sample. The perceived income risk is the average across all individuals from the cohort in that year. }
\end{figure}

\clearpage

\begin{figure}[!ht]
	\caption{Experience and Perceived Income Risk: Aggregate and Idiosyncratic}
	\label{fig:experience_id_ag_data}
	\begin{center}\adjustimage{max size={0.4\linewidth}{0.3\paperheight}}{../../Graphs/ind/experience_id_gr_var_data.png} 
		\adjustimage{max size={0.4\linewidth}{0.3\paperheight}}{../../Graphs/ind/experience_var_id_var_data.png}
	\adjustimage{max size={0.4\linewidth}{0.3\paperheight}}{../../Graphs/ind/experience_ag_gr_var_data.png}
	\adjustimage{max size={0.4\linewidth}{0.3\paperheight}}{../../Graphs/ind/experience_var_ag_var_data.png}
\adjustimage{max size={0.4\linewidth}{0.3\paperheight}}{../../Graphs/ind/experience_ue_var_data.png}
\adjustimage{max size={0.4\linewidth}{0.3\paperheight}}{../../Graphs/ind/experience_ue_var_var_data.png}
\end{center}
	\floatfoot{Note: experienced idiosyncratic income shocks are approximated as the cohort-specific average unexplained income residuals from a regression controlling time-fixed and education/time fixed effect.  Aggregate shock is approixmated as the average income change explained by the two effects.  The perceived income risk is the average across all individuals from the cohort in that year.}
\end{figure}

\clearpage

\begin{figure}[!ht]
	\caption{Attribution and Parameter Uncertainty}
	\label{fig:corr_var}
	\begin{center}\adjustimage{max size={0.9\linewidth}{0.4\paperheight}}{../../Graphs/theory/corr_var.jpg}\end{center}
	\floatfoot{Note: this figure illustrates how parameter uncertainty changes with the subjective correlation of one's own income and others'.}
\end{figure}

\clearpage
\begin{figure}[!ht]
	\caption{Experienced Volatility and Perceived Risk}
	\label{fig:var_experience_var}
	\begin{center}\adjustimage{max size={0.9\linewidth}{0.4\paperheight}}{../../Graphs/theory/var_experience_var.jpg}\end{center}
	\floatfoot{Note: this figure illustrates the relationship between experienced volatility and perceived income income risk under different attributions.}
\end{figure}

\clearpage
\begin{figure}[!ht]
	\caption{Attribution Function}
	\label{fig:theta_corr}
	\begin{center}\adjustimage{max size={0.9\linewidth}{0.4\paperheight}}{../../Graphs/theory/theta_corr.jpg}\end{center}
	\floatfoot{Note: this figure illustrates the parameterized attribution function under different degree of attribution error governed by $\theta$.}
\end{figure}

\clearpage
\begin{figure}[!ht]
	\caption{Current Income and Perceived Risk}
	\label{fig:var_recent}
	\begin{center}\adjustimage{max size={0.9\linewidth}{0.4\paperheight}}{../../Graphs/theory/var_recent.jpg}\end{center}
	\floatfoot{Note: this figure plots the theoretical prediction of the relationship between current income and perceived income risks.}
\end{figure}


\clearpage
\begin{figure}[!ht]
	\caption{Simulated Income Profile of Perceived Risk}
	\label{fig:var_recent_sim}
	\begin{center}\adjustimage{max size={0.9\linewidth}{0.4\paperheight}}{../../Graphs/theory/var_recent_sim.jpg}\end{center}
	\floatfoot{Note: this figure plots the simulated relationship between current income and perceived income risks under the theory.}
\end{figure}

    
    \clearpage
    \begin{figure}[!ht]
    	\caption{Simulated Age Profile of Perceived Risk}
    	\label{fig:var_age_sim}
    	\begin{center}\adjustimage{max size={0.9\linewidth}{0.4\paperheight}}{../../Graphs/theory/var_age_sim.jpg}\end{center}
    	\floatfoot{Note: this figure plots the simulated relationship between age and perceived income risks.}
    \end{figure}
    
    \clearpage
    \begin{figure}[!ht]
    	\caption{Simulatd Average Labor Market and Perceived Risk}
    	\label{fig:recent_change_var_sim}
    	\begin{center}   		
    		\adjustimage{max size={0.9\linewidth}{0.4\paperheight}}{../../Graphs/theory/var_recent_change_sim.jpg} \\
    	\adjustimage{max size={0.9\linewidth}{0.4\paperheight}}{../../Graphs/theory/var_recent_change_sim2.jpg}
\end{center}
    	\floatfoot{Note: this figure plots the simulated relationship between average perceived risks and average income changes with/without attribution errors (the upper panel) and under aggregate/idiosyncratic risks (the bottom panel). }
    \end{figure}
    
    
\clearpage
   
   % tables   
\clearpage

\begin{table}[ht]
\centering
\begin{adjustbox}{width={\textwidth}}
\begin{threeparttable}
\caption{Current Labor Market Conditions and Perceived Income Risks}
\label{macro_corr_he}
\begin{tabular}{ccccccl}
\toprule
{} &  mean:var &  mean:iqr & mean:rvar & median:var & median:iqr & median:rvar \\
\midrule
0 &   -0.28** &  -0.42*** &  -0.48*** &      -0.16 &      -0.16 &    -0.53*** \\
1 &  -0.44*** &  -0.54*** &  -0.51*** &      -0.02 &      -0.02 &    -0.53*** \\
2 &  -0.39*** &  -0.44*** &  -0.43*** &      -0.05 &        0.0 &    -0.45*** \\
3 &  -0.44*** &  -0.47*** &  -0.41*** &      -0.09 &      -0.06 &     -0.5*** \\
4 &   -0.29** &  -0.38*** &  -0.32*** &      -0.19 &      -0.14 &     -0.5*** \\
\bottomrule
\end{tabular}
\begin{tablenotes}
\item *** p$<$0.001, ** p$<$0.01 and * p$<$0.05.
\item This table reports correlation coefficients between different perceived income moments(inc for nominal
and rinc for real) at time
$t$ and the quarterly growth rate in hourly earning at $t,t-1,...,t-k$.
\end{tablenotes}
\end{threeparttable}
\end{adjustbox}
\end{table}
\clearpage

	
	\begin{table}[ht]
		\centering
		\begin{adjustbox}{width=0.9\textwidth}
			\begin{threeparttable}
			\caption{Average Perceived Risks and State Labor Market}
			\label{macro_corr_he_state}
			\begin{tabular}{lllll}
					\hline 
				& (1)                & (2)                & (3)               & (4)               \\
				& log perceived risk & log perceived risk & log perceived iqr & log perceived iqr \\
				\hline 
				Wage Growth (Median) & -0.05***           &                    & -0.03***          &                   \\
				& (0.01)             &                    & (0.01)            &                   \\
				&                    &                    &                   &                   \\
				UE (Median)          &                    & 0.04*              &                   & 0.04***           \\
				&                    & (0.02)             &                   & (0.01)            \\
				&                    &                    &                   &                   \\
					\hline 
				Observations         & 3589               & 3589               & 3596              & 3596              \\
				R-squared            & 0.021              & 0.019              & 0.025             & 0.027             \\
				\hline      
			\end{tabular}
			
				\begin{tablenotes}
					\item *** p$<$0.001, ** p$<$0.01 and * p$<$0.05.
					\item This table reports regression coefficient of the average perceived income risk of each state in different times on current labor market indicators, i.e. wage growth and unemployment rate. Montly state wage series is from Local Area Unemployment Statistics (LAUS) of BLS. Quarterly state unemployment rate is from Quarterly Census of Employment and Wage (QCEW) of BLS. 
				\end{tablenotes}
			\end{threeparttable}
		\end{adjustbox}
	\end{table}
\clearpage

\begin{table}[p]
\centering
\begin{adjustbox}{width=\textwidth}
\begin{threeparttable}
\caption{Perceived Income Risks, Experienced Volatility and Individual Characteristics}
\label{micro_reg}\begin{tabular}{lllllll}
\toprule
{} & incvar I & incvar II & incvar III & incvar IIII & incvar IIIII & incvar IIIIII \\
                    &          &           &            &             &              &               \\
\midrule
IdExpVol            &  4.58*** &   2.23*** &    2.69*** &     2.75*** &      2.95*** &       2.94*** \\
                    &   (0.33) &    (0.36) &     (0.39) &      (0.39) &       (0.38) &        (0.39) \\
AgExpVol            &     0.04 &   0.28*** &    0.34*** &     0.32*** &      0.18*** &       0.20*** \\
                    &   (0.04) &    (0.04) &     (0.05) &      (0.05) &       (0.05) &        (0.05) \\
AgExpUE             &  0.14*** &   0.08*** &     0.05** &       0.05* &        0.04* &        0.05** \\
                    &   (0.02) &    (0.02) &     (0.02) &      (0.02) &       (0.02) &        (0.02) \\
age                 &          &  -0.02*** &   -0.02*** &    -0.02*** &     -0.02*** &      -0.02*** \\
                    &          &    (0.00) &     (0.00) &      (0.00) &       (0.00) &        (0.00) \\
gender=male         &          &           &   -0.36*** &    -0.35*** &     -0.32*** &      -0.30*** \\
                    &          &           &     (0.02) &      (0.02) &       (0.02) &        (0.02) \\
nlit\_gr=low nlit    &          &           &    0.09*** &     0.09*** &      0.10*** &       0.09*** \\
                    &          &           &     (0.02) &      (0.02) &       (0.02) &        (0.02) \\
parttime=yes        &          &           &            &             &        -0.01 &         -0.02 \\
                    &          &           &            &             &       (0.02) &        (0.02) \\
selfemp=yes         &          &           &            &             &      1.25*** &      -0.00*** \\
                    &          &           &            &             &       (0.03) &        (0.00) \\
UEprobAgg           &          &           &            &             &              &       0.02*** \\
                    &          &           &            &             &              &        (0.00) \\
UEprobInd           &          &           &            &             &              &       0.02*** \\
                    &          &           &            &             &              &        (0.00) \\
HHinc\_gr=low income &          &           &            &             &      0.16*** &       0.16*** \\
                    &          &           &            &             &       (0.02) &        (0.02) \\
educ\_gr=high school &          &           &            &    -0.10*** &     -0.13*** &      -0.09*** \\
                    &          &           &            &      (0.02) &       (0.02) &        (0.02) \\
educ\_gr=hs dropout  &          &           &            &        0.08 &         0.11 &       0.29*** \\
                    &          &           &            &      (0.11) &       (0.11) &        (0.11) \\
N                   &    41422 &     41422 &      34833 &       34833 &        33480 &         29687 \\
R2                  &     0.01 &      0.02 &       0.04 &        0.04 &         0.11 &          0.06 \\
\bottomrule
\end{tabular}
\begin{tablenotes}\item Standard errors are clustered by household. *** p$<$0.001, ** p$<$0.01 and * p$<$0.05. 
\item This table reports results associated a regression of looged perceived income risks (incvar) on logged indiosyncratic($\text{IdExpVol}$), aggregate experienced volatility($\text{AgExpVol}$), experienced unemployment rate (AgExpUE), and a list of household specific variables such as age, income, education, gender, job type and other economic expectations.
\end{tablenotes}
\end{threeparttable}
\end{adjustbox}
\end{table}
\clearpage

\begin{table}[p]
\centering
\begin{adjustbox}{width={0.9\textwidth}}
\begin{threeparttable}
\caption{Perceived Income Risks and Household Spending}
\label{spending_reg}\begin{tabular}{cccccc}
\toprule
{} & spending I & spending II & spending III & spending IIII & spending IIIII \\
          &            &             &              &               &                \\
\midrule
incexp    &   39.11*** &             &              &               &                \\
          &     (8.47) &             &              &               &                \\
incvar    &            &     1.86*** &              &               &                \\
          &            &      (0.46) &              &               &                \\
rincvar   &            &             &      2.49*** &               &                \\
          &            &             &       (0.35) &               &                \\
incskew   &            &             &              &          0.19 &                \\
          &            &             &              &        (0.45) &                \\
UEprobAgg &            &             &              &               &          0.44* \\
          &            &             &              &               &         (0.25) \\
N         &      53455 &       53171 &        49986 &         52751 &          76531 \\
R2        &       0.00 &        0.00 &         0.00 &          0.00 &           0.00 \\
\bottomrule
\end{tabular}
\begin{tablenotes}\item Standard errors are clustered by household. *** p$<$0.001, ** p$<$0.01 and * p$<$0.05. 
\item This table reports regression results of expected spending growth on perceived income risks (incvar for nominal, rincvar for real).
\end{tablenotes}
\end{threeparttable}
\end{adjustbox}
\end{table}       % Add a bibliography block to the postdoc
    
    
\bibliographystyle{apalike}
\bibliography{PerceivedIncomeRisk}

    
\end{document}
