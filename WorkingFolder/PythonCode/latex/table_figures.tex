                                                                                                                                                                        \newpage 
   
  \section*{Tables and Figures} 
    
    
    % figures 
    

        \begin{figure}[!ht]
        		\caption{Distribution of Individual Moments}
        	\label{fig:histmoms}
    	\begin{center}
    		\adjustimage{max size={0.4\linewidth}{0.3\paperheight}}{../../Graphs/ind/hist_incexp.jpg}
    		\adjustimage{max size={0.4\linewidth}{0.3\paperheight}}{../../Graphs/ind/hist_rincexp.jpg} \\
    		\adjustimage{max size={0.4\linewidth}{0.3\paperheight}}{../../Graphs/ind/hist_incvar.jpg}
    		\adjustimage{max size={0.4\linewidth}{0.3\paperheight}}{../../Graphs/ind/hist_rincvar.jpg}    
\end{center}
    	\floatfoot{Note: this figure plots the expected nominal and real income growth and perceived risks.}
    \end{figure}
    
    
    \begin{figure}[!ht]
    	\caption{Perceived Income by Age}
    	\label{fig:ts_incvar_age}
    	\begin{center}\adjustimage{max size={0.7\linewidth}}{../../Graphs/ind/ts/ts_incvar_age_g_mean.png}\end{center}
    	\floatfoot{Note: this figure plots average perceived income risks of different age groups over time.}
    \end{figure}
    
    
    \begin{figure}[!ht]
    	\caption{Perceived Income by Income}
    	\label{fig:barplot_byinc}
    	\begin{center}\adjustimage{max size={0.7\linewidth}}{../../Graphs/ind/boxplot_var_stata.png}\end{center}
    	\floatfoot{Note: this figure plots average perceived income risks by the range of household income.}
    \end{figure}
    
    
    \begin{figure}[!ht]
      \caption{Recent Labor Market Outcome and Perceived Risks}
    \label{fig:tshe}
    	\begin{center}\adjustimage{max size={\linewidth}}{../../Graphs/pop/tsMeanvar_he.jpg}\end{center}
    	\floatfoot{Note: recent labor market outcome is measured by hourly earning growth (YoY).}
    \end{figure}

 
\begin{figure}[!ht]
	\caption{Experienced Volatility and Perceived Income Risk}
	\label{fig:var_experience_var_data}
	\begin{center}\adjustimage{max size={0.9\linewidth}{0.4\paperheight}}{../../Graphs/ind/experience_var_var_data.png}\end{center}
	\floatfoot{Note: experienced volatility is the mean squred error(MSE) of income regression based on a particular year-cohort sample. The perceived income risk is the average across all individuals from the cohort in that year.}
\end{figure}



\begin{figure}[!ht]
	\caption{Attribution and Parameter Uncertainty}
	\label{fig:corr_var}
	\begin{center}\adjustimage{max size={0.9\linewidth}{0.4\paperheight}}{../../Graphs/theory/corr_var.jpg}\end{center}
	\floatfoot{Note: this figure illustrates how parameter uncertainty changes with the subjective correlation of one's own income and others'.}
\end{figure}


\begin{figure}[!ht]
	\caption{Experienced Volatility and Perceived Risk}
	\label{fig:var_experience_var}
	\begin{center}\adjustimage{max size={0.9\linewidth}{0.4\paperheight}}{../../Graphs/theory/var_experience_var.jpg}\end{center}
	\floatfoot{Note: this figure illustrates the relationship between experienced volatility and perceived income income risk under different attributions.}
\end{figure}


\begin{figure}[!ht]
	\caption{Attribution Function}
	\label{fig:theta_corr}
	\begin{center}\adjustimage{max size={0.9\linewidth}{0.4\paperheight}}{../../Graphs/theory/theta_corr.jpg}\end{center}
	\floatfoot{Note: this figure illustrates the parameterized attribution function under different degree of attribution error governed by $\theta$.}
\end{figure}


\begin{figure}[!ht]
	\caption{Current Income and Perceived Risk}
	\label{fig:var_recent}
	\begin{center}\adjustimage{max size={0.9\linewidth}{0.4\paperheight}}{../../Graphs/theory/var_recent.jpg}\end{center}
	\floatfoot{Note: this figure plots the theoretical prediction of the relationship between current income and perceived income risks.}
\end{figure}



\begin{figure}[!ht]
	\caption{Simulated Income Profile of Perceived Risk}
	\label{fig:var_recent_sim}
	\begin{center}\adjustimage{max size={0.9\linewidth}{0.4\paperheight}}{../../Graphs/theory/var_recent_sim.jpg}\end{center}
	\floatfoot{Note: this figure plots the simulated relationship between current income and perceived income risks under the theory.}
\end{figure}

    
    
    \begin{figure}[!ht]
    	\caption{Simulated Age Profile of Perceived Risk}
    	\label{fig:var_age_sim}
    	\begin{center}\adjustimage{max size={0.9\linewidth}{0.4\paperheight}}{../../Graphs/theory/var_age_sim.jpg}\end{center}
    	\floatfoot{Note: this figure plots the simulated relationship between age and perceived income risks.}
    \end{figure}
    
    
    \begin{figure}[!ht]
    	\caption{Simulatd Average Labor Market and Perceived Risk}
    	\label{fig:recent_change_var_sim1}
    	\begin{center}   		
    		\adjustimage{max size={0.9\linewidth}{0.4\paperheight}}{../../Graphs/theory/var_recent_change_sim.jpg} \\
    	\adjustimage{max size={0.9\linewidth}{0.4\paperheight}}{../../Graphs/theory/var_recent_change_sim2.jpg}
\end{center}
    	\floatfoot{Note: this figure plots the simulated relationship between average perceived risks and average income changes with/without attribution errors (the upper panel) and under aggregate/idiosyncratic risks (the bottom panel). }
    \end{figure}
    
    
   \newpage 
   
   % tables   
\begin{table}[ht]
\centering
\begin{adjustbox}{width={\textwidth}}
\begin{threeparttable}
\caption{Current Labor Market Conditions and Perceived Income Risks}
\label{macro_corr_he}
\begin{tabular}{ccccccl}
\toprule
{} &  mean:var &  mean:iqr & mean:rvar & median:var & median:iqr & median:rvar \\
\midrule
0 &   -0.24** &  -0.36*** &  -0.47*** &      -0.15 &      -0.15 &    -0.53*** \\
1 &  -0.44*** &  -0.53*** &  -0.56*** &      -0.01 &       0.01 &    -0.55*** \\
2 &  -0.37*** &  -0.42*** &  -0.43*** &      -0.09 &      -0.07 &    -0.45*** \\
3 &   -0.4*** &  -0.43*** &  -0.41*** &      -0.09 &      -0.09 &    -0.46*** \\
4 &   -0.28** &  -0.38*** &   -0.31** &    -0.27** &     -0.24* &    -0.49*** \\
\bottomrule
\end{tabular}
\begin{tablenotes}
\item *** p$<$0.001, ** p$<$0.01 and * p$<$0.05.
\item This table reports correlation coefficients between different perceived income moments(inc for nominal
and rinc for real) at time
$t$ and the quarterly growth rate in hourly earning at $t,t-1,...,t-k$.
\end{tablenotes}
\end{threeparttable}
\end{adjustbox}
\end{table}
	
	\begin{table}[ht]
		\centering
		\begin{adjustbox}{width=0.5\textwidth}
			\begin{threeparttable}
			\caption{Average Perceived Risks and State Labor Market}
			\label{macro_corr_he_state}
			\begin{tabular}{lllll}
				\hline 
				& (1)                & (2)                & (3)               & (4)               \\
				\hline 
				& log(var) & log(risk) & log(iqr) & log(iqr) \\
				\hline 
				wage growth & -0.05***           &                    & -0.03***          &                   \\
				
				& (0.01)             &                    & (0.01)            &                   \\
				unemp rate &                    & 0.04*              &                   & 0.04***           \\
				&                    & (0.02)             &                   & (0.01)            \\
				\hline 
				Observations      & 3529               & 3529               & 3546              & 3546              \\
				R-squared         & 0.023              & 0.020              & 0.025             & 0.028            \\
				\hline 
			\end{tabular}
			
				\begin{tablenotes}
					\item *** p$<$0.001, ** p$<$0.01 and * p$<$0.05.
					\item This table reports regression coefficient of the average perceived income risk of each state in different times on current labor market indicators, i.e. wage growth and unemployment rate. 
				\end{tablenotes}
			\end{threeparttable}
		\end{adjustbox}
	\end{table}
\begin{table}[p]
\centering
\begin{adjustbox}{width=\textwidth}
\begin{threeparttable}
\caption{Perceived Income Risks, Experienced Volatility and Individual Characteristics}
\label{micro_reg}\begin{tabular}{llllll}
\toprule
{} & incvar I & incvar II & incvar III & incvar IIII & incvar IIIII \\
                 &          &           &            &             &              \\
\midrule
expvol           &  6.31*** &   2.92*** &    3.56*** &     3.56*** &      6.15*** \\
                 &   (0.40) &    (0.83) &     (0.91) &      (0.91) &       (0.92) \\
age\_gr=30-39     &          &  -0.33*** &   -0.34*** &    -0.34*** &     -0.38*** \\
                 &          &    (0.03) &     (0.03) &      (0.03) &       (0.03) \\
age\_gr=40-48     &          &  -0.51*** &   -0.53*** &    -0.53*** &     -0.61*** \\
                 &          &    (0.03) &     (0.03) &      (0.03) &       (0.03) \\
age\_gr=49-57     &          &  -0.61*** &   -0.59*** &    -0.59*** &     -0.65*** \\
                 &          &    (0.03) &     (0.03) &      (0.03) &       (0.03) \\
age\_gr=>57       &          &  -0.48*** &   -0.48*** &    -0.48*** &     -0.58*** \\
                 &          &    (0.04) &     (0.05) &      (0.05) &       (0.05) \\
HHinc\_gr=low inc &          &           &    0.20*** &     0.20*** &      0.15*** \\
                 &          &           &     (0.02) &      (0.02) &       (0.02) \\
educ\_gr=low educ &          &           &   -0.11*** &    -0.11*** &     -0.08*** \\
                 &          &           &     (0.02) &      (0.02) &       (0.02) \\
gender=male      &          &           &   -0.38*** &    -0.38*** &     -0.31*** \\
                 &          &           &     (0.02) &      (0.02) &       (0.02) \\
parttime=yes     &          &           &            &             &        -0.03 \\
                 &          &           &            &             &       (0.02) \\
selfemp=yes      &          &           &            &             &      0.00*** \\
                 &          &           &            &             &       (0.00) \\
UEprobAgg        &          &           &            &             &      0.00*** \\
                 &          &           &            &             &       (0.00) \\
UEprobInd        &          &           &            &             &      0.00*** \\
                 &          &           &            &             &       (0.00) \\
N                &    40529 &     40529 &      34101 &       34101 &        28898 \\
R2               &     0.01 &      0.02 &       0.04 &        0.04 &         0.05 \\
\bottomrule
\end{tabular}
\begin{tablenotes}\item Standard errors are clustered by household. *** p$<$0.001, ** p$<$0.01 and * p$<$0.05. 
\item This table reports results associated a regression of looged perceived income risks (incvar) on logged experienced volatility ($\text{expvol}$) and a list of household specific variables such as age, income, education, gender, job type and unemployment expectations.
\end{tablenotes}
\end{threeparttable}
\end{adjustbox}
\end{table}
\begin{table}[p]
\centering
\begin{adjustbox}{width={0.9\textwidth}}
\begin{threeparttable}
\caption{Perceived Income Risks and Household Spending}
\label{spending_reg}\begin{tabular}{cccccc}
\toprule
{} & spending I & spending II & spending III & spending IIII & spending IIIII \\
          &            &             &              &               &                \\
\midrule
incexp    &    0.39*** &             &              &               &                \\
          &     (0.08) &             &              &               &                \\
incvar    &            &     0.58*** &              &               &                \\
          &            &      (0.13) &              &               &                \\
rincvar   &            &             &      1.08*** &               &                \\
          &            &             &       (0.15) &               &                \\
incskew   &            &             &              &          0.19 &                \\
          &            &             &              &        (0.45) &                \\
UEprobAgg &            &             &              &               &          0.44* \\
          &            &             &              &               &         (0.25) \\
N         &      53455 &       53171 &        49986 &         52751 &          76531 \\
R2        &       0.00 &        0.00 &         0.00 &          0.00 &           0.00 \\
\bottomrule
\end{tabular}
\begin{tablenotes}\item Standard errors are clustered by household. *** p$<$0.001, ** p$<$0.01 and * p$<$0.05. 
\item This table reports regression results of expected spending growth on perceived income risks (incvar for nominal, rincvar for real).
\end{tablenotes}
\end{threeparttable}
\end{adjustbox}
\end{table}