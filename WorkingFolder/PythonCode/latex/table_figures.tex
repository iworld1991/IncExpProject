                                                                                                                                                                                                                                                                                                                                                                                                                                                                                                                                 \newpage 
   
  \section*{Tables and Figures} 
    
    
    % figures 
    

        \begin{figure}[!ht]
        		\caption{Distribution of Individual Moments}
        	\label{fig:histmoms}
    	\begin{center}
    		\adjustimage{max size={0.4\linewidth}{0.3\paperheight}}{../../Graphs/ind/hist_incvar.jpg}
    		\adjustimage{max size={0.4\linewidth}{0.3\paperheight}}{../../Graphs/ind/hist_rincvar.jpg}    
\end{center}
    	\floatfoot{Note: this figure plots the perceived risks of nominal and real income, measured by the subjective variance of log income changes one year from the time of the survey. Real risk is the sum of the perceived risk of nominal income and inflation uncertainty.}
    \end{figure}
    
    \clearpage
    
    \begin{figure}[!ht]
    	\caption{Perceived Income by Age}
    	\label{fig:ts_incvar_age}
    	\begin{center}\adjustimage{max size={0.7\linewidth}}{../../Graphs/ind/ts/ts_incvar_age_g_mean.png}\end{center}
    	\floatfoot{Note: this figure plots average perceived income risks of different age groups over time.}
    \end{figure}

\clearpage
\begin{figure}[!ht]
	\caption{Realized and Perceived Age Profile of Income Risks}
	\label{fig:log_wage_shk_gr_by_age_compare}
	\begin{center}\adjustimage{max size={0.7\linewidth}}{../../Graphs/psid/real_log_wage_shk_gr_by_age_compare.png}\end{center}
	\floatfoot{Note: this figure plots average realized and perceived income risks of different ages. Realized income risks is defined as the age-specific income volatility estimated as the standard-deviation of change in unexplained income residuals from PSID income panel. Perceived income risk is obtained from SCE.}
\end{figure}
    
    \clearpage
    \begin{figure}[!ht]
    	\caption{Perceived Income by Income}
    	\label{fig:barplot_byinc}
    	\begin{center}\adjustimage{max size={0.7\linewidth}}{../../Graphs/ind/boxplot_var_HHinc_stata.png}\end{center}
    	\floatfoot{Note: this figure plots average perceived income risks by the range of household income.}
    \end{figure}
    
    \clearpage
    \begin{figure}[!ht]
      \caption{Recent Labor Market Outcome and Perceived Risks}
    \label{fig:tshe}
    	\begin{center}\adjustimage{max size={\linewidth}}{../../Graphs/pop/tsMean3mvvar_he.jpg}\end{center}
    	\floatfoot{Note: recent labor market outcome is measured by hourly wage growth (YoY). The 3-month moving average is plotted for both series.}
    \end{figure}

 \clearpage
\begin{figure}[!ht]
	\caption{Experience and Perceived Income Risk}
	\label{fig:var_experience_data}
	\begin{center}
		\adjustimage{max size={0.7\linewidth}{0.4\paperheight}}{../../Graphs/ind/experience_gr_var_data.png}
	\adjustimage{max size={0.7\linewidth}{0.4\paperheight}}{../../Graphs/ind/experience_var_var_data.png}
\end{center}
	\floatfoot{Note: experienced growth is the average growth of unexplained income residual and the experienced volatility is its cross-sectional variance within each corresponding cohort. The latter is essentially computed as the mean squred error(MSE) from an income regression on observable individual characteristics including age, age-squred, time, education and gender. The perceived income risk is the average across all individuals from the cohort in that year. Cohorts are time/year-of-birth/education-specific and all cohort sized 30 or smaller are excluded.}
\end{figure}

\clearpage
\begin{figure}[!ht]
	\caption{Experience and Perceived Income Risk: Permanent and Transitory}
	\label{fig:experience_var_per_tran_var_data}
	\begin{center}
		\adjustimage{max size={0.6\linewidth}{0.3\paperheight}}{../../Graphs/ind/experience_var_permanent_var_data.png}
		\adjustimage{max size={0.6\linewidth}{0.3\paperheight}}{../../Graphs/ind/experience_var_transitory_var_data.png}
		\adjustimage{max size={0.6\linewidth}{0.3\paperheight}}{../../Graphs/ind/experience_var_ratio_var_data.png}
\end{center}
	\floatfoot{Note: experienced permanent (transitory) volatility is average of the estimated risks of the permanent (transitory) component of a particular year-cohort sample. The perceived income risk is the average across all individuals from the cohort in that year. }
\end{figure}

\clearpage

\begin{figure}[!ht]
	\caption{Experience and Perceived Income Risk: Aggregate and Idiosyncratic}
	\label{fig:experience_id_ag_data}
	\begin{center}\adjustimage{max size={0.4\linewidth}{0.3\paperheight}}{../../Graphs/ind/experience_id_gr_var_data.png} 
		\adjustimage{max size={0.4\linewidth}{0.3\paperheight}}{../../Graphs/ind/experience_var_id_var_data.png}
	\adjustimage{max size={0.4\linewidth}{0.3\paperheight}}{../../Graphs/ind/experience_ag_gr_var_data.png}
	\adjustimage{max size={0.4\linewidth}{0.3\paperheight}}{../../Graphs/ind/experience_var_ag_var_data.png}
\adjustimage{max size={0.4\linewidth}{0.3\paperheight}}{../../Graphs/ind/experience_ue_var_data.png}
\adjustimage{max size={0.4\linewidth}{0.3\paperheight}}{../../Graphs/ind/experience_ue_var_var_data.png}
\end{center}
	\floatfoot{Note: experienced idiosyncratic income shocks are approximated as the cohort-specific average unexplained income residuals from a regression controlling time-fixed and education/time fixed effect.  Aggregate shock is approixmated as the average income change explained by the two effects.  The perceived income risk is the average across all individuals from the cohort in that year.}
\end{figure}
   
   % tables   
\clearpage

\begin{table}[ht]
\centering
\begin{adjustbox}{width={\textwidth}}
\begin{threeparttable}
\caption{Current Labor Market Conditions and Perceived Income Risks}
\label{macro_corr_he}
\begin{tabular}{lllll}
\toprule
{} &  mean:var &  mean:iqr & mean:rvar & mean:skew \\
\midrule
0 &   -0.28** &  -0.42*** &  -0.48*** &     -0.02 \\
1 &  -0.42*** &  -0.53*** &  -0.51*** &      0.12 \\
2 &  -0.43*** &  -0.48*** &  -0.44*** &     -0.01 \\
3 &  -0.43*** &  -0.48*** &  -0.42*** &      -0.1 \\
4 &  -0.31*** &  -0.41*** &  -0.32*** &    -0.21* \\
\bottomrule
\end{tabular}
\begin{tablenotes}
\item *** p$<$0.001, ** p$<$0.01 and * p$<$0.05.
\item This table reports correlation coefficients between different perceived income moments(inc for nominal
and rinc for real) at time
$t$ and the quarterly growth rate in hourly earning at $t,t-1,...,t-k$.
\end{tablenotes}
\end{threeparttable}
\end{adjustbox}
\end{table}
\clearpage

	
	\begin{table}[ht]
		\centering
		\begin{adjustbox}{width=0.9\textwidth}
			\begin{threeparttable}
			\caption{Average Perceived Risks and State Labor Market}
			\label{macro_corr_he_state}
			\begin{tabular}{lllll}
					\hline 
				& (1)                & (2)                & (3)               & (4)               \\
				& log perceived risk & log perceived risk & log perceived iqr & log perceived iqr \\
				\hline 
				Wage Growth (Median) & -0.05***           &                    & -0.03***          &                   \\
				& (0.01)             &                    & (0.01)            &                   \\
				&                    &                    &                   &                   \\
				UE (Median)          &                    & 0.04*              &                   & 0.04***           \\
				&                    & (0.02)             &                   & (0.01)            \\
				&                    &                    &                   &                   \\
					\hline 
				Observations         & 3589               & 3589               & 3596              & 3596              \\
				R-squared            & 0.021              & 0.019              & 0.025             & 0.027             \\
				\hline      
			\end{tabular}
			
				\begin{tablenotes}
					\item *** p$<$0.001, ** p$<$0.01 and * p$<$0.05.
					\item This table reports regression coefficient of the average perceived income risk of each state in different times on current labor market indicators, i.e. wage growth and unemployment rate. Montly state wage series is from Local Area Unemployment Statistics (LAUS) of BLS. Quarterly state unemployment rate is from Quarterly Census of Employment and Wage (QCEW) of BLS. 
				\end{tablenotes}
			\end{threeparttable}
		\end{adjustbox}
	\end{table}
\clearpage

\begin{table}[p]
\centering
\begin{adjustbox}{width=\textwidth}
\begin{threeparttable}
\caption{Perceived Income Risks, Experienced Volatility and Individual Characteristics}
\label{micro_reg}\begin{tabular}{lllllll}
\toprule
{} &  incvar I & incvar II & incvar III & incvar IIII & incvar IIIII & incvar IIIIII \\
                    &           &           &            &             &              &               \\
\midrule
IdExpVol            &   0.42*** &     -0.08 &      -0.05 &      0.64** &      0.78*** &       0.85*** \\
                    &    (0.13) &    (0.14) &     (0.15) &      (0.29) &       (0.28) &        (0.29) \\
AgExpVol            &  -0.20*** &   0.21*** &    0.24*** &     0.23*** &       0.09** &        0.11** \\
                    &    (0.03) &    (0.04) &     (0.04) &      (0.04) &       (0.04) &        (0.04) \\
AgExpUE             &   0.30*** &   0.13*** &    0.10*** &     0.09*** &      0.09*** &       0.11*** \\
                    &    (0.02) &    (0.02) &     (0.02) &      (0.02) &       (0.02) &        (0.02) \\
age                 &           &  -0.02*** &   -0.02*** &    -0.02*** &     -0.02*** &      -0.02*** \\
                    &           &    (0.00) &     (0.00) &      (0.00) &       (0.00) &        (0.00) \\
gender=male         &           &           &   -0.36*** &    -0.35*** &     -0.33*** &      -0.30*** \\
                    &           &           &     (0.02) &      (0.02) &       (0.02) &        (0.02) \\
nlit\_gr=low nlit    &           &           &    0.08*** &     0.08*** &      0.08*** &       0.08*** \\
                    &           &           &     (0.02) &      (0.02) &       (0.02) &        (0.02) \\
parttime=yes        &           &           &            &             &        -0.01 &         -0.02 \\
                    &           &           &            &             &       (0.02) &        (0.02) \\
selfemp=yes         &           &           &            &             &      1.25*** &      -0.00*** \\
                    &           &           &            &             &       (0.03) &        (0.00) \\
UEprobAgg           &           &           &            &             &              &       0.01*** \\
                    &           &           &            &             &              &        (0.00) \\
UEprobInd           &           &           &            &             &              &       0.01*** \\
                    &           &           &            &             &              &        (0.00) \\
HHinc\_gr=low income &           &           &            &             &      0.16*** &       0.17*** \\
                    &           &           &            &             &       (0.02) &        (0.02) \\
educ\_gr=high school &           &           &            &    -0.11*** &     -0.17*** &      -0.13*** \\
                    &           &           &            &      (0.03) &       (0.03) &        (0.03) \\
educ\_gr=hs dropout  &           &           &            &       -0.25 &       -0.31* &         -0.17 \\
                    &           &           &            &      (0.19) &       (0.19) &        (0.19) \\
N                   &     41422 &     41422 &      34833 &       34833 &        33480 &         29687 \\
R2                  &      0.01 &      0.02 &       0.03 &        0.04 &         0.10 &          0.06 \\
\bottomrule
\end{tabular}
\begin{tablenotes}\item Standard errors are clustered by household. *** p$<$0.001, ** p$<$0.01 and * p$<$0.05. 
\item This table reports results from a regression of looged perceived income risks (incvar) on logged indiosyncratic($\text{IdExpVol}$), aggregate experienced volatility($\text{AgExpVol}$), experienced unemployment rate (AgExpUE), and a list of household specific variables such as age, income, education, gender, job type and other economic expectations.
\end{tablenotes}
\end{threeparttable}
\end{adjustbox}
\end{table}
\clearpage

\begin{table}[ht]
\centering
\begin{adjustbox}{width={\textwidth}}
\begin{threeparttable}
\caption{Correlation between Perceived Income Risks and Stock Market Return}
\label{macro_corr}
\begin{tabular}{ccccccl}
\toprule
{} & median:var & median:iqr & median:rvar & mean:var & mean:iqr & mean:rvar \\
\midrule
0  &       -0.1 &       -0.1 &       -0.04 &    -0.08 &    -0.06 &     -0.06 \\
1  &      -0.06 &      -0.05 &        0.07 &      0.0 &     0.02 &      0.07 \\
2  &      -0.17 &      -0.13 &        0.02 &      0.0 &    -0.02 &     0.19* \\
3  &      -0.03 &       0.03 &       -0.02 &    -0.07 &    -0.09 &      0.15 \\
4  &       0.09 &       0.12 &       -0.18 &   -0.21* &   -0.22* &     -0.16 \\
5  &        0.0 &       0.02 &       -0.2* &    -0.19 &   -0.19* &    -0.23* \\
6  &      -0.06 &      -0.03 &     -0.23** &    -0.13 &    -0.16 &     -0.18 \\
7  &     -0.22* &      -0.2* &     -0.26** &    -0.11 &    -0.16 &     -0.09 \\
8  &      -0.07 &      -0.03 &       -0.2* &    -0.17 &   -0.23* &     -0.13 \\
9  &       0.03 &       0.04 &       -0.05 &   -0.22* &   -0.23* &     -0.19 \\
10 &       0.12 &       0.14 &       -0.02 &  -0.26** &   -0.23* &     -0.08 \\
11 &       0.05 &       0.05 &       -0.03 &    -0.07 &    -0.09 &      0.01 \\
12 &       0.08 &       0.07 &       -0.07 &    -0.06 &    -0.07 &     -0.05 \\
\bottomrule
\end{tabular}
\begin{tablenotes}
\item *** p$<$0.001, ** p$<$0.01 and * p$<$0.05.
\item This table reports correlation coefficients between different perceived income moments(inc for nominal
and rinc for real) at time
$t$ and the monthly s\&p500 return by the end of $t+k$ for $k=0,1,..,11$.
\end{tablenotes}
\end{threeparttable}
\end{adjustbox}
\end{table}
\clearpage

\begin{table}[p]
\centering
\begin{adjustbox}{width={0.9\textwidth}}
\begin{threeparttable}
\caption{Perceived Income Risks and Household Spending}
\label{spending_reg}\begin{tabular}{cccccc}
\toprule
{} & spending I & spending II & spending III & spending IIII & spending IIIII \\
          &            &             &              &               &                \\
\midrule
incexp    &   36.24*** &             &              &               &                \\
          &     (7.46) &             &              &               &                \\
incvar    &            &     0.63*** &              &               &                \\
          &            &      (0.12) &              &               &                \\
rincvar   &            &             &      1.07*** &               &                \\
          &            &             &       (0.13) &               &                \\
incskew   &            &             &              &          0.19 &                \\
          &            &             &              &        (0.40) &                \\
UEprobAgg &            &             &              &               &           0.16 \\
          &            &             &              &               &         (0.13) \\
N         &      61422 &       62959 &        59057 &         60560 &          87775 \\
R2        &       0.00 &        0.00 &         0.00 &          0.00 &           0.00 \\
\bottomrule
\end{tabular}
\begin{tablenotes}\item Standard errors are clustered by household. *** p$<$0.001, ** p$<$0.01 and * p$<$0.05. 
\item This table reports regression results of expected spending growth on perceived income risks (incvar for nominal, rincvar for real).
\end{tablenotes}
\end{threeparttable}
\end{adjustbox}
\end{table}