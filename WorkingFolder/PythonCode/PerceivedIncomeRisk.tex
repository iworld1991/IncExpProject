
% Default to the notebook output style

    


% Inherit from the specified cell style.




    
\documentclass[12pt,notitlepage,onecolumn,aps,pra]{revtex4-1}

    
    
\usepackage[T1]{fontenc}
\usepackage{graphicx}
% We will generate all images so they have a width \maxwidth. This means
% that they will get their normal width if they fit onto the page, but
% are scaled down if they would overflow the margins.
\makeatletter
\def\maxwidth{\ifdim\Gin@nat@width>\linewidth\linewidth
\else\Gin@nat@width\fi}
\makeatother
\let\Oldincludegraphics\includegraphics
% Set max figure width to be 80% of text width, for now hardcoded.
\renewcommand{\includegraphics}[1]{\Oldincludegraphics[width=.8\maxwidth]{#1}}
% Ensure that by default, figures have no caption (until we provide a
% proper Figure object with a Caption API and a way to capture that
% in the conversion process - todo).
\usepackage{caption}
\DeclareCaptionLabelFormat{nolabel}{}
\captionsetup{labelformat=nolabel}

\usepackage{adjustbox} % Used to constrain images to a maximum size
\usepackage{xcolor} % Allow colors to be defined
\usepackage{enumerate} % Needed for markdown enumerations to work
\usepackage{geometry} % Used to adjust the document margins
\usepackage{amsmath} % Equations
\usepackage{amssymb} % Equations
\usepackage{textcomp} % defines textquotesingle
% Hack from http://tex.stackexchange.com/a/47451/13684:
\AtBeginDocument{%
    \def\PYZsq{\textquotesingle}% Upright quotes in Pygmentized code
}
\usepackage{upquote} % Upright quotes for verbatim code
\usepackage{eurosym} % defines \euro
\usepackage[mathletters]{ucs} % Extended unicode (utf-8) support
\usepackage[utf8x]{inputenc} % Allow utf-8 characters in the tex document
\usepackage{fancyvrb} % verbatim replacement that allows latex
\usepackage{grffile} % extends the file name processing of package graphics
                     % to support a larger range
% The hyperref package gives us a pdf with properly built
% internal navigation ('pdf bookmarks' for the table of contents,
% internal cross-reference links, web links for URLs, etc.)
\usepackage{hyperref}
\usepackage{natbib}
\usepackage{booktabs}  % table support for pandoc > 1.12.2
\usepackage[inline]{enumitem} % IRkernel/repr support (it uses the enumerate* environment)
\usepackage[normalem]{ulem} % ulem is needed to support strikethroughs (\sout)
                            % normalem makes italics be italics, not underlines
\usepackage{braket}


    
    
    % Colors for the hyperref package
    \definecolor{urlcolor}{rgb}{0,.145,.698}
    \definecolor{linkcolor}{rgb}{.71,0.21,0.01}
    \definecolor{citecolor}{rgb}{.12,.54,.11}

    % ANSI colors
    \definecolor{ansi-black}{HTML}{3E424D}
    \definecolor{ansi-black-intense}{HTML}{282C36}
    \definecolor{ansi-red}{HTML}{E75C58}
    \definecolor{ansi-red-intense}{HTML}{B22B31}
    \definecolor{ansi-green}{HTML}{00A250}
    \definecolor{ansi-green-intense}{HTML}{007427}
    \definecolor{ansi-yellow}{HTML}{DDB62B}
    \definecolor{ansi-yellow-intense}{HTML}{B27D12}
    \definecolor{ansi-blue}{HTML}{208FFB}
    \definecolor{ansi-blue-intense}{HTML}{0065CA}
    \definecolor{ansi-magenta}{HTML}{D160C4}
    \definecolor{ansi-magenta-intense}{HTML}{A03196}
    \definecolor{ansi-cyan}{HTML}{60C6C8}
    \definecolor{ansi-cyan-intense}{HTML}{258F8F}
    \definecolor{ansi-white}{HTML}{C5C1B4}
    \definecolor{ansi-white-intense}{HTML}{A1A6B2}
    \definecolor{ansi-default-inverse-fg}{HTML}{FFFFFF}
    \definecolor{ansi-default-inverse-bg}{HTML}{000000}

    % commands and environments needed by pandoc snippets
    % extracted from the output of `pandoc -s`
    \providecommand{\tightlist}{%
      \setlength{\itemsep}{0pt}\setlength{\parskip}{0pt}}
    \DefineVerbatimEnvironment{Highlighting}{Verbatim}{commandchars=\\\{\}}
    % Add ',fontsize=\small' for more characters per line
    \newenvironment{Shaded}{}{}
    \newcommand{\KeywordTok}[1]{\textcolor[rgb]{0.00,0.44,0.13}{\textbf{{#1}}}}
    \newcommand{\DataTypeTok}[1]{\textcolor[rgb]{0.56,0.13,0.00}{{#1}}}
    \newcommand{\DecValTok}[1]{\textcolor[rgb]{0.25,0.63,0.44}{{#1}}}
    \newcommand{\BaseNTok}[1]{\textcolor[rgb]{0.25,0.63,0.44}{{#1}}}
    \newcommand{\FloatTok}[1]{\textcolor[rgb]{0.25,0.63,0.44}{{#1}}}
    \newcommand{\CharTok}[1]{\textcolor[rgb]{0.25,0.44,0.63}{{#1}}}
    \newcommand{\StringTok}[1]{\textcolor[rgb]{0.25,0.44,0.63}{{#1}}}
    \newcommand{\CommentTok}[1]{\textcolor[rgb]{0.38,0.63,0.69}{\textit{{#1}}}}
    \newcommand{\OtherTok}[1]{\textcolor[rgb]{0.00,0.44,0.13}{{#1}}}
    \newcommand{\AlertTok}[1]{\textcolor[rgb]{1.00,0.00,0.00}{\textbf{{#1}}}}
    \newcommand{\FunctionTok}[1]{\textcolor[rgb]{0.02,0.16,0.49}{{#1}}}
    \newcommand{\RegionMarkerTok}[1]{{#1}}
    \newcommand{\ErrorTok}[1]{\textcolor[rgb]{1.00,0.00,0.00}{\textbf{{#1}}}}
    \newcommand{\NormalTok}[1]{{#1}}
    
    % Additional commands for more recent versions of Pandoc
    \newcommand{\ConstantTok}[1]{\textcolor[rgb]{0.53,0.00,0.00}{{#1}}}
    \newcommand{\SpecialCharTok}[1]{\textcolor[rgb]{0.25,0.44,0.63}{{#1}}}
    \newcommand{\VerbatimStringTok}[1]{\textcolor[rgb]{0.25,0.44,0.63}{{#1}}}
    \newcommand{\SpecialStringTok}[1]{\textcolor[rgb]{0.73,0.40,0.53}{{#1}}}
    \newcommand{\ImportTok}[1]{{#1}}
    \newcommand{\DocumentationTok}[1]{\textcolor[rgb]{0.73,0.13,0.13}{\textit{{#1}}}}
    \newcommand{\AnnotationTok}[1]{\textcolor[rgb]{0.38,0.63,0.69}{\textbf{\textit{{#1}}}}}
    \newcommand{\CommentVarTok}[1]{\textcolor[rgb]{0.38,0.63,0.69}{\textbf{\textit{{#1}}}}}
    \newcommand{\VariableTok}[1]{\textcolor[rgb]{0.10,0.09,0.49}{{#1}}}
    \newcommand{\ControlFlowTok}[1]{\textcolor[rgb]{0.00,0.44,0.13}{\textbf{{#1}}}}
    \newcommand{\OperatorTok}[1]{\textcolor[rgb]{0.40,0.40,0.40}{{#1}}}
    \newcommand{\BuiltInTok}[1]{{#1}}
    \newcommand{\ExtensionTok}[1]{{#1}}
    \newcommand{\PreprocessorTok}[1]{\textcolor[rgb]{0.74,0.48,0.00}{{#1}}}
    \newcommand{\AttributeTok}[1]{\textcolor[rgb]{0.49,0.56,0.16}{{#1}}}
    \newcommand{\InformationTok}[1]{\textcolor[rgb]{0.38,0.63,0.69}{\textbf{\textit{{#1}}}}}
    \newcommand{\WarningTok}[1]{\textcolor[rgb]{0.38,0.63,0.69}{\textbf{\textit{{#1}}}}}
    
    
    % Define a nice break command that doesn't care if a line doesn't already
    % exist.
    \def\br{\hspace*{\fill} \\* }
    % Math Jax compatibility definitions
    \def\gt{>}
    \def\lt{<}
    \let\Oldtex\TeX
    \let\Oldlatex\LaTeX
    \renewcommand{\TeX}{\textrm{\Oldtex}}
    \renewcommand{\LaTeX}{\textrm{\Oldlatex}}
    % Document parameters
    % Document title
    
    
    
    

    % Pygments definitions
    
\makeatletter
\def\PY@reset{\let\PY@it=\relax \let\PY@bf=\relax%
    \let\PY@ul=\relax \let\PY@tc=\relax%
    \let\PY@bc=\relax \let\PY@ff=\relax}
\def\PY@tok#1{\csname PY@tok@#1\endcsname}
\def\PY@toks#1+{\ifx\relax#1\empty\else%
    \PY@tok{#1}\expandafter\PY@toks\fi}
\def\PY@do#1{\PY@bc{\PY@tc{\PY@ul{%
    \PY@it{\PY@bf{\PY@ff{#1}}}}}}}
\def\PY#1#2{\PY@reset\PY@toks#1+\relax+\PY@do{#2}}

\expandafter\def\csname PY@tok@w\endcsname{\def\PY@tc##1{\textcolor[rgb]{0.73,0.73,0.73}{##1}}}
\expandafter\def\csname PY@tok@c\endcsname{\let\PY@it=\textit\def\PY@tc##1{\textcolor[rgb]{0.25,0.50,0.50}{##1}}}
\expandafter\def\csname PY@tok@cp\endcsname{\def\PY@tc##1{\textcolor[rgb]{0.74,0.48,0.00}{##1}}}
\expandafter\def\csname PY@tok@k\endcsname{\let\PY@bf=\textbf\def\PY@tc##1{\textcolor[rgb]{0.00,0.50,0.00}{##1}}}
\expandafter\def\csname PY@tok@kp\endcsname{\def\PY@tc##1{\textcolor[rgb]{0.00,0.50,0.00}{##1}}}
\expandafter\def\csname PY@tok@kt\endcsname{\def\PY@tc##1{\textcolor[rgb]{0.69,0.00,0.25}{##1}}}
\expandafter\def\csname PY@tok@o\endcsname{\def\PY@tc##1{\textcolor[rgb]{0.40,0.40,0.40}{##1}}}
\expandafter\def\csname PY@tok@ow\endcsname{\let\PY@bf=\textbf\def\PY@tc##1{\textcolor[rgb]{0.67,0.13,1.00}{##1}}}
\expandafter\def\csname PY@tok@nb\endcsname{\def\PY@tc##1{\textcolor[rgb]{0.00,0.50,0.00}{##1}}}
\expandafter\def\csname PY@tok@nf\endcsname{\def\PY@tc##1{\textcolor[rgb]{0.00,0.00,1.00}{##1}}}
\expandafter\def\csname PY@tok@nc\endcsname{\let\PY@bf=\textbf\def\PY@tc##1{\textcolor[rgb]{0.00,0.00,1.00}{##1}}}
\expandafter\def\csname PY@tok@nn\endcsname{\let\PY@bf=\textbf\def\PY@tc##1{\textcolor[rgb]{0.00,0.00,1.00}{##1}}}
\expandafter\def\csname PY@tok@ne\endcsname{\let\PY@bf=\textbf\def\PY@tc##1{\textcolor[rgb]{0.82,0.25,0.23}{##1}}}
\expandafter\def\csname PY@tok@nv\endcsname{\def\PY@tc##1{\textcolor[rgb]{0.10,0.09,0.49}{##1}}}
\expandafter\def\csname PY@tok@no\endcsname{\def\PY@tc##1{\textcolor[rgb]{0.53,0.00,0.00}{##1}}}
\expandafter\def\csname PY@tok@nl\endcsname{\def\PY@tc##1{\textcolor[rgb]{0.63,0.63,0.00}{##1}}}
\expandafter\def\csname PY@tok@ni\endcsname{\let\PY@bf=\textbf\def\PY@tc##1{\textcolor[rgb]{0.60,0.60,0.60}{##1}}}
\expandafter\def\csname PY@tok@na\endcsname{\def\PY@tc##1{\textcolor[rgb]{0.49,0.56,0.16}{##1}}}
\expandafter\def\csname PY@tok@nt\endcsname{\let\PY@bf=\textbf\def\PY@tc##1{\textcolor[rgb]{0.00,0.50,0.00}{##1}}}
\expandafter\def\csname PY@tok@nd\endcsname{\def\PY@tc##1{\textcolor[rgb]{0.67,0.13,1.00}{##1}}}
\expandafter\def\csname PY@tok@s\endcsname{\def\PY@tc##1{\textcolor[rgb]{0.73,0.13,0.13}{##1}}}
\expandafter\def\csname PY@tok@sd\endcsname{\let\PY@it=\textit\def\PY@tc##1{\textcolor[rgb]{0.73,0.13,0.13}{##1}}}
\expandafter\def\csname PY@tok@si\endcsname{\let\PY@bf=\textbf\def\PY@tc##1{\textcolor[rgb]{0.73,0.40,0.53}{##1}}}
\expandafter\def\csname PY@tok@se\endcsname{\let\PY@bf=\textbf\def\PY@tc##1{\textcolor[rgb]{0.73,0.40,0.13}{##1}}}
\expandafter\def\csname PY@tok@sr\endcsname{\def\PY@tc##1{\textcolor[rgb]{0.73,0.40,0.53}{##1}}}
\expandafter\def\csname PY@tok@ss\endcsname{\def\PY@tc##1{\textcolor[rgb]{0.10,0.09,0.49}{##1}}}
\expandafter\def\csname PY@tok@sx\endcsname{\def\PY@tc##1{\textcolor[rgb]{0.00,0.50,0.00}{##1}}}
\expandafter\def\csname PY@tok@m\endcsname{\def\PY@tc##1{\textcolor[rgb]{0.40,0.40,0.40}{##1}}}
\expandafter\def\csname PY@tok@gh\endcsname{\let\PY@bf=\textbf\def\PY@tc##1{\textcolor[rgb]{0.00,0.00,0.50}{##1}}}
\expandafter\def\csname PY@tok@gu\endcsname{\let\PY@bf=\textbf\def\PY@tc##1{\textcolor[rgb]{0.50,0.00,0.50}{##1}}}
\expandafter\def\csname PY@tok@gd\endcsname{\def\PY@tc##1{\textcolor[rgb]{0.63,0.00,0.00}{##1}}}
\expandafter\def\csname PY@tok@gi\endcsname{\def\PY@tc##1{\textcolor[rgb]{0.00,0.63,0.00}{##1}}}
\expandafter\def\csname PY@tok@gr\endcsname{\def\PY@tc##1{\textcolor[rgb]{1.00,0.00,0.00}{##1}}}
\expandafter\def\csname PY@tok@ge\endcsname{\let\PY@it=\textit}
\expandafter\def\csname PY@tok@gs\endcsname{\let\PY@bf=\textbf}
\expandafter\def\csname PY@tok@gp\endcsname{\let\PY@bf=\textbf\def\PY@tc##1{\textcolor[rgb]{0.00,0.00,0.50}{##1}}}
\expandafter\def\csname PY@tok@go\endcsname{\def\PY@tc##1{\textcolor[rgb]{0.53,0.53,0.53}{##1}}}
\expandafter\def\csname PY@tok@gt\endcsname{\def\PY@tc##1{\textcolor[rgb]{0.00,0.27,0.87}{##1}}}
\expandafter\def\csname PY@tok@err\endcsname{\def\PY@bc##1{\setlength{\fboxsep}{0pt}\fcolorbox[rgb]{1.00,0.00,0.00}{1,1,1}{\strut ##1}}}
\expandafter\def\csname PY@tok@kc\endcsname{\let\PY@bf=\textbf\def\PY@tc##1{\textcolor[rgb]{0.00,0.50,0.00}{##1}}}
\expandafter\def\csname PY@tok@kd\endcsname{\let\PY@bf=\textbf\def\PY@tc##1{\textcolor[rgb]{0.00,0.50,0.00}{##1}}}
\expandafter\def\csname PY@tok@kn\endcsname{\let\PY@bf=\textbf\def\PY@tc##1{\textcolor[rgb]{0.00,0.50,0.00}{##1}}}
\expandafter\def\csname PY@tok@kr\endcsname{\let\PY@bf=\textbf\def\PY@tc##1{\textcolor[rgb]{0.00,0.50,0.00}{##1}}}
\expandafter\def\csname PY@tok@bp\endcsname{\def\PY@tc##1{\textcolor[rgb]{0.00,0.50,0.00}{##1}}}
\expandafter\def\csname PY@tok@fm\endcsname{\def\PY@tc##1{\textcolor[rgb]{0.00,0.00,1.00}{##1}}}
\expandafter\def\csname PY@tok@vc\endcsname{\def\PY@tc##1{\textcolor[rgb]{0.10,0.09,0.49}{##1}}}
\expandafter\def\csname PY@tok@vg\endcsname{\def\PY@tc##1{\textcolor[rgb]{0.10,0.09,0.49}{##1}}}
\expandafter\def\csname PY@tok@vi\endcsname{\def\PY@tc##1{\textcolor[rgb]{0.10,0.09,0.49}{##1}}}
\expandafter\def\csname PY@tok@vm\endcsname{\def\PY@tc##1{\textcolor[rgb]{0.10,0.09,0.49}{##1}}}
\expandafter\def\csname PY@tok@sa\endcsname{\def\PY@tc##1{\textcolor[rgb]{0.73,0.13,0.13}{##1}}}
\expandafter\def\csname PY@tok@sb\endcsname{\def\PY@tc##1{\textcolor[rgb]{0.73,0.13,0.13}{##1}}}
\expandafter\def\csname PY@tok@sc\endcsname{\def\PY@tc##1{\textcolor[rgb]{0.73,0.13,0.13}{##1}}}
\expandafter\def\csname PY@tok@dl\endcsname{\def\PY@tc##1{\textcolor[rgb]{0.73,0.13,0.13}{##1}}}
\expandafter\def\csname PY@tok@s2\endcsname{\def\PY@tc##1{\textcolor[rgb]{0.73,0.13,0.13}{##1}}}
\expandafter\def\csname PY@tok@sh\endcsname{\def\PY@tc##1{\textcolor[rgb]{0.73,0.13,0.13}{##1}}}
\expandafter\def\csname PY@tok@s1\endcsname{\def\PY@tc##1{\textcolor[rgb]{0.73,0.13,0.13}{##1}}}
\expandafter\def\csname PY@tok@mb\endcsname{\def\PY@tc##1{\textcolor[rgb]{0.40,0.40,0.40}{##1}}}
\expandafter\def\csname PY@tok@mf\endcsname{\def\PY@tc##1{\textcolor[rgb]{0.40,0.40,0.40}{##1}}}
\expandafter\def\csname PY@tok@mh\endcsname{\def\PY@tc##1{\textcolor[rgb]{0.40,0.40,0.40}{##1}}}
\expandafter\def\csname PY@tok@mi\endcsname{\def\PY@tc##1{\textcolor[rgb]{0.40,0.40,0.40}{##1}}}
\expandafter\def\csname PY@tok@il\endcsname{\def\PY@tc##1{\textcolor[rgb]{0.40,0.40,0.40}{##1}}}
\expandafter\def\csname PY@tok@mo\endcsname{\def\PY@tc##1{\textcolor[rgb]{0.40,0.40,0.40}{##1}}}
\expandafter\def\csname PY@tok@ch\endcsname{\let\PY@it=\textit\def\PY@tc##1{\textcolor[rgb]{0.25,0.50,0.50}{##1}}}
\expandafter\def\csname PY@tok@cm\endcsname{\let\PY@it=\textit\def\PY@tc##1{\textcolor[rgb]{0.25,0.50,0.50}{##1}}}
\expandafter\def\csname PY@tok@cpf\endcsname{\let\PY@it=\textit\def\PY@tc##1{\textcolor[rgb]{0.25,0.50,0.50}{##1}}}
\expandafter\def\csname PY@tok@c1\endcsname{\let\PY@it=\textit\def\PY@tc##1{\textcolor[rgb]{0.25,0.50,0.50}{##1}}}
\expandafter\def\csname PY@tok@cs\endcsname{\let\PY@it=\textit\def\PY@tc##1{\textcolor[rgb]{0.25,0.50,0.50}{##1}}}

\def\PYZbs{\char`\\}
\def\PYZus{\char`\_}
\def\PYZob{\char`\{}
\def\PYZcb{\char`\}}
\def\PYZca{\char`\^}
\def\PYZam{\char`\&}
\def\PYZlt{\char`\<}
\def\PYZgt{\char`\>}
\def\PYZsh{\char`\#}
\def\PYZpc{\char`\%}
\def\PYZdl{\char`\$}
\def\PYZhy{\char`\-}
\def\PYZsq{\char`\'}
\def\PYZdq{\char`\"}
\def\PYZti{\char`\~}
% for compatibility with earlier versions
\def\PYZat{@}
\def\PYZlb{[}
\def\PYZrb{]}
\makeatother


    % Exact colors from NB
    \definecolor{incolor}{rgb}{0.0, 0.0, 0.5}
    \definecolor{outcolor}{rgb}{0.545, 0.0, 0.0}



    
    % Prevent overflowing lines due to hard-to-break entities
    \sloppy 
    % Setup hyperref package
    \hypersetup{
      breaklinks=true,  % so long urls are correctly broken across lines
      colorlinks=true,
      urlcolor=urlcolor,
      linkcolor=linkcolor,
      citecolor=citecolor,
      }
    % Slightly bigger margins than the latex defaults
    
    \geometry{verbose,tmargin=1in,bmargin=1in,lmargin=1in,rmargin=1in}
    
    

    \begin{document}
    
    
    \title{Perceived Income Risks}\author{Tao Wang}\affiliation{Johns Hopkins University}

\date{\today}
\maketitle


    
    

    
    \hypertarget{the-research-question}{%
\section{The research question}\label{the-research-question}}

``The devil is in higher moments.'' Even if two people share identical
expected income and homogeneous preferences, different degrees of income
risks still lead to starkly different decisions such as
saving/consumption and portfolio choices. This is well understood in
models in which agents are inter-temporally risk-averse, or prudent, and
the risks associated with future marginal utility motivate precautionary
motives. The same logic carries through to models in which capital
income and portfolio returns are stochastic, and the risks of returns
naturally become the center of asset pricing. Such behavioral
regularities equipped with market incompleteness due to reasons such as
imperfect insurance and credit constraints have also been the
cornerstone assumptions used in the literature on heterogeneous-agent
macroeconomics.

Economists have long utilized cross-sectional distributions of realized
micro data to estimate the stochastic environments relevant for the
agents' decision, such as income process. And then in modeling the
estimated risk profile is taken as parametric inputs and the individual
shocks are simply drawn from the shared distributions.(See
\cite{blundell_consumption_2008} as an example.) But one assumption
implicitly made when doing this is that the agents in the model
perfectly understand thus agree on the income risk profile imposed on
them. As shown by the actively developing literature on expectation
formation, in particular, the mounting evidence on heterogeneity in
economic expectations held by micro agents, this assumption seems be too
stringent. To the extent that agents make decisions based on their
\emph{respective} perceptions, understanding the \emph{perceived} income
risk profile and its correlation structure with other macro variables
are the keys to explaining their behavior patterns.

This paper's goal is to understand the question discussed above by
directly shedding light on the subjective income profile using the
recently available density forecasts of labor income surveyed by New
York Fed's Survey of Consumer Expectation (SCE). What is special about
this survey is that agents are asked to provide histogram-type forecasts
of their earning growth over the next 12 months together with a set of
expectational questions about macroeconomy. It is at monthly frequency
and has a panel structure allowing for concesecutive observations of the
same household over a horizon of 12 months. When the indiviudal density
foreacst is available, a parametric density estimation can be made to
obtain the individual-specific subjetive distribution. And higher
moments reflecting the perceived income risks such as variance, as well
as the assymmetry of the distributon such as skewness allow me to
directly chracterize the perceived risk profile without relying on
external estimates from cross-sectinal micro data. This provides the
first-hand measured perceptions on income risks that are truely relevant
to individual decisions.

Empirically, I can immediately ask following questions.

\begin{itemize}
\item
  How much heterogeneity is there across workers' perceived income
  risks? What factors, i.e.~household income, demographics, and other
  expectations, are correlated with the subjective risks in both
  individual and macro level?
\item
  To what extent to which this heterogeneity in perceptions align with
  the true income risks facing different population group, or at least
  partly attributed to perceptive differences due to heterogeneity in
  information and information processing, as discussed in many models?

  \begin{itemize}
  \tightlist
  \item
    For instance, if we treat the income risks identified from
    cross-sectional inequality by econometricians as a benchmark, to
    what extent are the risks perceived by the agents?

    \begin{itemize}
    \tightlist
    \item
      If agents know more than econometricians about their individual
      earnings, the perceived risks may be lower than econometrician's
      estimates?
    \item
      Or actually, agents, due to inattention or other reasons, tend to
      think the overall risk is higher?
    \end{itemize}
  \end{itemize}
\item
  If the subjetive income risk can be decomposed into composents of
  varying persistence (i.e.~permanent vs transitory) based on assumed
  income process, it is possible to charaterize potential deviations of
  perceptive income process from some well defined rational benchmark.

  \begin{itemize}
  \tightlist
  \item
    For instance, if agents overestimate their permanent income risks?
  \item
    If agents overestimate the persistence of the income process?
    \cite{rozsypal_overpersistence_2017}
  \item
    One step back, if the perceived income process is really log normal.
    Or it has skewness? This can be jointly tested using higher moments
    of the density forecasts.
  \end{itemize}
\item
  Finally, not just the process of earning itself, but also its
  covariance with macro-environment, risky asset returns, matter a great
  deal. For instance, if perceived income risks are counter-cyclical, it
  has important labor supply and portfolio implications.
  \cite{catherine_countercyclical_2019}
\end{itemize}

Theoretically, once I can document robustly some patterns of the
perceived income risks profile, it can ben incorporated into an
otherwise standard life-cycle models involving consumption/portfolio
decisions to explore its macro implications. Ex ante, one may conjecture
a few of the following scenarios.

\begin{itemize}
\item
  If the subjetive risks or skewness is found to be negatively
  correlated with the risky market return or business cycles, this
  exposes agents to more risks than a non-state-dependent income
  profile.
\item
  If according to the subjetive risk profile, the downside risks are
  highly persistent than typically assumed, then it is in line with the
  rare disaster idea.
\item
  The peceptive differences lead to differences in MPCs.
\end{itemize}

\hypertarget{relevant-literature-and-potential-contribution}{%
\subsection{Relevant literature and potential
contribution}\label{relevant-literature-and-potential-contribution}}

This paper is relevant to four lines of literature. First, the idea of
this paper echoes with an old problem in the consumption insurance
literature: `insurance or information' (\cite{pistaferri_superior_2001},
\cite{kaufmann_disentangling_2009}). In any empirical tests of
consumption insurance or consumption response to income, there is always
a worry that what is interpreted as the shock has actually already
entered the agents' information set or exactly the opposite. For
instance, the notion of excessive sensitivity, namely households
consumption highly responsive to anticipated income shock, maybe simply
because agents have not incorporated the recently realized shocks that
econometricians assume so. My paper shares a similar spirit with the few
papers above in the sense that we try to tackle the identification
problem in the same approach: directly using the expectation data and
explicitly controling what are truely conditional expectations of the
agents making the decision. This helps economists avoid making
assumptions on what is exactly in the agents' information set. What
differentiates my work from these few papers is that I focus on higher
moments, i.e.~income risks and skewness by utilizing the recently
available density forecasts of labor income. Part of my exercise can be
also seen as an extension of the New York Fed
\href{https://libertystreeteconomics.newyorkfed.org/2017/11/understanding-permanent-and-temporary-income-shocks.html}{blog}
from the first moment to the second moments.

Second, this paper is inspired by an old but recently reviving interest
in studying consumption/saving behaviors in models incorporating
imperfect expectations and perceptions. For instance,
\cite{rozsypal_overpersistence_2017} found that households' expectation
of income exhibits an over-persistent bias using both expected and
realized household income from Michigan household survey. The paper also
shows that incorporating such bias affects the aggregate consumption
function by distorting the cross-sectional distributions of marginal
propensity to consume(MPCs) across the population.
\cite{carroll_sticky_2018} reconciles the low micro-MPC and high
macro-MPCs by introducing to the model an information rigidity of
households in learning about macro news while being updated about micro
news. \cite{Lian_imperfect_2019} shows that an imperfect perception of
wealth accounts such phenomenon as excess sensitivity to current income
and higher MPCs out of wealth than current income and so forth.
\_\textbf{My paper has a similar flavor to all of these works by
exploring the behavioral implications of households' perceptive
imperfection. But unlike their work, my paper dominantly focues on the
implications of heterogeneity in perceived higher moments such as risks
and skewness.}

This paper also contributes to the literature studying expectation
formation using subjective surveys. There has been a long list of
`'irrational expectation'' theories developed in recent decades on how
agents deviate from full-information rationality benchmark, such as
sticky expectation, noisy signal extraction, least-square learning, etc.
Also, empirical work has been devoted to testing these theories. But it
is fair to say that thus far, relatively little work has been done on
individual variables such as labor income, which may well be more
relevant to individual economic decisions. Therefore, understanding
expectation formation of the individual variables, in particular,
concerning both mean and higher moments, will prove fruitful insights
for macroeconomic modeling assumptions.

Lastly, the paper is indirectly related to the research that advocated
for eliciting probabilistic questions measuring subjective uncertainty
in economic surveys (\cite{manski_measuring_2004},
\cite{delavande2011measuring}, \cite{manski_survey_2018}). Although the
initial suspicion concerning to people's ability in understanding, using
and answering probabilistic questions is understandable,
\cite{bertrand_people_2001} and other works have shown respondents have
the consistent ability and willingness to assign a probability (or
``percent chance'') to future events. \cite{armantier_overview_2017}
have a thorough discussion on designing, experimenting and implementing
the consumer expectation surveys to ensure the quality of the responses.
Broadly speaking, the advocators have argued that going beyond the
revealed preference approach, availability to survey data provides
economists with direct information on agents' expectations and helps
avoids imposing arbitrary assumptions. This insight holds for not only
point forecast but also and even more importantly, for uncertainty,
because for any economic decision made by a risk-averse agent, not only
the expectation but also the perceived risks matter a great deal.

\hypertarget{a-broader-motivation}{%
\subsection{A broader motivation}\label{a-broader-motivation}}

The approach that I am proposing here is a natural development from the
existing literature of expectation formation. One of the common
practices in this literature is to compare the measured expectations
from surveys with the law of the systems independently identified by
econometricians and interpret all deviations of the former to the latter
as the evidence for irrationality. It is true that this has proved to be
fruitful and refreshing compared to the earlier macroeconomic tradition
that solely relies on the stringent assumption of rationality. But such
practices implicitly assume the process discovered by econometricians is
the ``true'' one. It does not recognize at all the use of large-sized
surveys of expectations in discovering the law of the system besides
making a case about how expectations are not rational. Therefore, a
rather obvious reconciliation building upon the existing literature is
to utilize jointly the realized data and expectations to understand the
``true'' process, while allowing for the partial rationality of the
modelers and agents in the model. - The advantage of doing this is - One
does not need to make a stringent assumption about either agents' full
rationality or econometricians' correctness of model specification. -
Utilize the information from expectations to understand the true law of
the system. - Once we take this step, it is natural to incorporate
specific mechanisms of expectation formation into a full-fledged
structural model that contains optimizing decisions and general
equilibrium forces.


    \hypertarget{data-variables-and-density-estimation}{%
\section{Data, variables and density
estimation}\label{data-variables-and-density-estimation}}

\hypertarget{data}{%
\subsection{Data}\label{data}}

The data used for this paper is from the core module of Survey of
Consumer Expectation(SCE) conducted by the New York Fed, a monthly
online survey for a rotating panel of around 1,300 household heads. The
sample period in my paper spans from June 2013 to June 2018, in total of
60 months. This makes about 79000 household-year observations, among
which around 53,000 observations provide non-empty answers to the
density question on earning growth.

Particular relevant for my purpose, the questionnaire asks each
respondent to fill perceived probabilities of their same-job-hour
earning growth to pre-defined non-overlapping bins. The question is
framed as ``suppose that 12 months from now, you are working in the
exact same {[}``main'' if Q11\textgreater{}1{]} job at the same place
you currently work and working the exact same number of hours. In your
view, what would you say is the percent chance that 12 months from now:
increased by x\% or more?''.

As a special feature of the online questionnaire, the survey only moves
on to the next question if the probabilities filled in all bins add up
to one. This ensures the basic probabilistic consistency of the answers
crucial for any further analysis. Besides, the earning growth
expectation regarding exactly the same position, same hours and same
location has two important implications for my analysis. First, the
requirements help make sure the comparability of the answers across time
and also excludes the potential changes in earnings driven by endogenous
labor supply decisions, i.e.~working for longer hours. Second, the
earning expectations, risks and tail risks measured here are only
conditional. It excludes either unemployment, i.e.~likely a zero
earning, or an upward movement in the job ladder, i.e.~a different
earning growth rate. Therefore, it does not fully reflect the labor
income risk profile relevant to each individual.

In so far as we want to tease out the earning changes from voluntary
decisions, moving to a different job position should actually not be
included in earning expectations. Therefore only the exclusion of an
unemployment scenario is relevant for my purpose in characterizing the
labor income risks. But what is assuring me is that the bias due to
omission of unemployment risk is unambiguous. We could interpret the
moments of same-job-hour earning growth as an upper bound for the level
of growth rate and a lower bound for the income risk. To put it in
another way, the expected earning growth conditional on employment is
higher than the unconditional one, and the conditional earning risk is
lower than the unconditional one. At the same time, since SCE separately
elicits the perceived probability of losing the current job for each
respondent, I could adjust the measured labor income moments taking into
account the unemployment risk.

\hypertarget{density-estimation-and-variables}{%
\subsection{Density estimation and
variables}\label{density-estimation-and-variables}}

With the histogram answers for each individual in hand, I follow
\cite{engelberg_comparing_2009} to fit each of them with a parametric
distribution accordingly for three following cases. In the first case
when there are three or more intervales filled with positive
probabilities, it will be fitted with a generalized beta distribution.
In particular, if there is no open-ended bin on the left or right, then
2-parameter beta distribution is sufficient. If there is either
open-ended bin with positive probability, since the lower bound or upper
bound of the support needs to be determined, a 4-parameter beta
distribution is estimated. In the second case, in which there are
exactly 2 adjacent intervals with positive probabilities, it is fitted
with an isosceles triangular distribution. In the third case, if there
is only one positive-probability of interval only, i.e.~equal to one, it
is fitted with a uniform distribution.

I have a reason to discuss at length the exact procedures for density
distribution. It is important for this paper's purpose because I need to
make sure the estimation assumptions of the density distribution do not
mechanically distort my cross-sectional patterns of the estimated
moments. This is the most obviously seen in the tail risk measure,
skewness. The assumption of log normality of income process, common in
the literature (See again \cite{blundell_consumption_2008}), implicitly
assume zero skewness, i.e.~that the income increase and decrease are
equally likely. This may not be the case in our surveyed density for
many individuals. In order to account for this possibility, the assumed
density distribution should be flexible enough to allow for different
shapes of subjective distribution. Beta distribution fits this purpose
well. Of course, in the case of uniform and isosceles triangular
distribution, the skewness is zero by default. For those of you who may
wonder, the fractions of the density answers fitted with the beta,
uniform, and triangular distributions are, respectively, xxx, xxx, xxx
in our sample.

Since the microdata provided in the SCE website already includes the
estimated mean, variance and IQR by the staff economists following the
exact same approach, I directly use their estimates for these moments.
At the same time, for the measure of tail-risk, i.e.~skewness, as not
provided, I use my own estimates. I also confirm that my estimates and
theirs for the first two moments are correlated with a coefficient of
0.9.

For all the moment's estimates, there are inevitably extreme values.
This could be due to the idiosyncratic answers provided by the original
respondent, or some non-convergence of the numeric estimation program.
Therefore, for each moment of the analysis, I exclude top and bottom
\(5\%\) observations, leading to a sample size of around 45,000.

I also recognize what is really relevant to many economic decisions such
as consumption, real income instead of nominal income is truely
relevant. Thanks to the availability of inflation expectation and
inflation uncertainty (also estimated from density question) can be used
to convert nominal earning growth moments to real terms. In particular,
the real earning growth rate is expected nominal growth minus inflation
expectation.

\begin{eqnarray}
\overline {\Delta y^{r}}_{i,t} = \overline\Delta y_{i,t} - \overline \pi_{i,t}
\end{eqnarray}

The variance associated with real earning growth, if we treat inflation
and nominal earning growth as two independent stochastical variables, is
equal to the summed variance of the two.

\begin{eqnarray}
\overline{var^{r}}_{i,t} = \overline{var}_{i,t} + \overline{var}_{i,t}(\pi_{t})
\end{eqnarray}

Not enough information is available for the same kind of transformation
of IQR and skewness from nominal to real, so I only use nominal
variables. Besides, as there are extreme values on inflation
expectations and uncertainty, I also exclude top and bottom \(5\%\) of
the observations. This further shrinks the sample, when using real
moments, to 36,000.

    \hypertarget{perceived-income-risks-basic-facts}{%
\section{Perceived income risks: basic
facts}\label{perceived-income-risks-basic-facts}}

\hypertarget{cross-sectional-heterogeneity}{%
\subsection{\texorpdfstring{\href{MacroRiskProfile.ipynb}{Cross-sectional
heterogeneity}}{Cross-sectional heterogeneity}}\label{cross-sectional-heterogeneity}}

This section inspects some basic cross-sectional patterns of the subject
moments of labor income. In the figure below, I plot the histograms of
\(\overline\Delta y_{i,t}\),
\(\overline{var}_{i,t}\),\(\overline{iqr}_{i,t}\) and
\(\overline {skw}_{i,t}\), \(\overline {\Delta y^{r}}_{i,t}\),
\(\overline{var^{r}}_{i,t}\).

First, expected income growth across the population exhibits a
dispersion ranging from a decrease of \(2-3\%\) to around an increase of
\(10\%\) in nominal terms. Given the well-known downward wage rigidity,
it is not surprising that most of the people expect a positive earning
growth. At the same time, the distribution of expected income growth is
right-skewed, meaning that more workers expect a smaller than larger
wage growth. What is interesting is that this cross-sectional
right-skewness in nominal earning disappears in expected real terms.
Expected earnings growth adjusted by individual inflation expectation
becomes symmetric around zero, ranging from a decrease of \(10\%\) to an
increase of \(10\%\). Real labor income increase and decrease are
approximately equally likely.

Second, as the primary focus of this paper, perceived income risks also
have a notable cross-sectional dispersion. For both measures of risks
variance and iqr, and in terms of both nominal and real terms, the
distribution is right-skewed with a long tail. Specifically, most of the
workers have perceived a variance of nominal earning growth ranging from
zero to \(20\) (a standard-deviation equivalence of \(4-4.5\%\) income
growth a year). But in the tail, some of the workers perceive risks to
be as high as \(7-8\%\) standard deviation a year. To have a better
sense of how large the risk is, consider a median individual in our
sample, who has an expected earning growth of \(2.4\%\), and a perceived
risk of \(1\%\) standard deviation. This implies by no means negligible
earning risk.

Third, the subjective skewness, an indicator of symmetry of the
perceived density, are distributed across populations symmetrically
around zero. It ranges from a left-skewness of the negative skewness of
0.6 to the same size of positive skewness or right-skewness. Although
one may think, based on cross-sectional distribution of the earning
growth, a right-skewness is more common it turns out that approximately
equal proportion of the sample has left and right tails of their
individual earning growth expectation. It is important to note here that
this pattern is not particularly due to our density estimation
assumptions. Both uniform and isosceles triangular distribution deliver
a skewness of zero. (This is also why we can observe a clear cluster of
the skewness at zero.) Therefore, the non-zero skewness estimates in our
sample are both from the beta distribution cases, which is flexible
enough to allow both.


    \begin{figure}
        \begin{center}\adjustimage{max size={0.9\linewidth}{0.4\paperheight}}{PerceivedIncomeRisk_files/PerceivedIncomeRisk_4_0.png}\end{center}
        \caption{Distribution of Inidivudal Momnets}
        \label{fig:histmoms}
    \end{figure}
    
    \hypertarget{correlation-with-asset-returns}{%
\subsection{\texorpdfstring{\href{MacroRiskProfile.ipynb}{Correlation
with asset
returns}}{Correlation with asset returns}}\label{correlation-with-asset-returns}}

It is not only the labor income risk profile per se but also the macro
risk profile, i.e.~how the labor income is correlated with risky asset
return and the business cycle, that is important for household
decisions. Since the short time period of my sample (2013M6-2018M5) has
not seen a single one business cycle, at least as defined by the NBER
recession committee, it poses a challenge for me to examine the
correlation between perceived risks and macroeconomic cycles. Therefore,
as the first stage of the analysis, I only focus on the correlation
between perceived risks and stock market returns.

Of course, there is a rationale in the first place to study stock market
return and labor income, as it bears critical implications for household
consumption insurance, equity premium, and participation puzzle. For
instance, a negative correlation of income risk and risky asset return
means households will be faced with higher risks of their total income
by investing in the stock market. Or a negative correlation between
skewness and stock market return, meaning a bigger income increase is
less likely in low-return times will also deter households from
participating in the stock market.

Following the most common practice in the finance literature, I use the
monthly return of the S\&P 500, computed from the beginning to the end
of the month, as an approximate of the stock market return. Over the
sample period, there are exactly two-thirds of the time marking a
positive return.

For a population summary statistic of individual moments of perceived
income growth, I take the median and mean across all respondents in the
survey for each point of the time. One may worry about the seasonality
of the monthly series of this kind. For instance, it is possible that
workers tend to learn news about their future earnings at a particular
month of the year, i.e.~end of the fiscal year when the wage contracts
are renegotiated and updated. Reasons of this kind may result in
seasonal patterns of the expected earning growth, variance and other
moments. Because my time series is too short in sample size to perform a
trustful seasonal adjustment, I check the seasonality by inspecting the
auto-correlation of each time series at different lags. As seen in the
figure of the appendix, although it seems that the average or median
earning growth per se has some seasonal patterns, there is no evidence
for higher moments, such as variance and skewness.

\href{appendix/acfpot.jpg}{figure in appendix on acf}

There are two econometric crucial considerations we examine the
correlation between the subjective moments of earning growth and stock
return.

The first is the time-average or time-aggregation problem documented in
both empirical asset pricing and consumption insurance literature
\cite{working_note_1960}, \cite{jagannathan_lazy_2007}, (Crawley, 2019).
Variables such as consumption and earning are interval measures,
reported as an average over a period, while the stock return is a spot
measure computed between two points of the time. As a result, if the
unit of the time for the underlying income process is at a higher
frequency than the measured interval (an extreme case being the
continuous-time), the measured variable will exhibit upward biased
autocorrelation and correlation with other underlying random walk series
in the same frequency. In my context, such a problem can be partly
mitigated by the availability of monthly frequency of earning
expectations, if we assume the unit of time of the underlying stochastic
process is a month. Then the directly observed monthly correlation of
the two cannot be driven by the time aggregation problem. What also
becomes immediately clear from this considration is that I should not
examine the correlation of the two series in moving average terms,
because it will cause the time aggregation problem. This point will be
discussed in greater detail in the next section when I decompose the
perceived income risks to different components of varying persistence.

The second issue regards which of the following, lagged, contemporaneous
or forward is the correct correlation one should look at. Considering
what is relevant to an individual making decisions are unrealized
stochastic shocks to both income and asset return, one should examine
the 1-year-ahead earning growth and its risks with the realized return
over the impeding 12 months at each point of the time.

With these considerations, in the figure below, I plot the median
perceived risk and skewness of both nominal and real earning along with
the contemporaneous stock market returns by the end of each month (also
true for the mean, see appendix). In order to account for the fact that
the survey is undertaken in the middle of the month while the return is
computed at the end of the month, I take the lag the income moments by 1
or 2 months when calculating the correlation coefficient. Although a
Pearson test of the correlation coefficients is only significant for a
2-month lag, overall, the income risks measured by variance and IQR for
both nominal and real earning post a negative correlation with the
realized stock return a few months ahead. The subjective skewness has
also a negative associated with the realized stock return in the near
future.

\href{../Graphs/pop/ts.jpg}{figures of stock market return and income
risks}

\href{../Tables/latex/macro_corr.tex}{table on correlation coefficient
of different lags}

More caution is needed when interpreting the observed negative
association between perceived earning risks/skewness with stock market
returns. First, my sample period is short and has mostly posted positive
returns. Second, the pattern is based on a population median and mean of
the perceived income risks, and it does not account for any
household-specific characteristics. As we have seen in the
cross-sectional pattern, there are substantial variations across
individuals in their perceived income risks and skewness. Third, the
risk profile we consider here is only relevant for marginal
consumers/investors who at least have access to the stock market in the
first place. Therefore, it is worth exploring the correlation above
conditional on more individual characteristics.


    \begin{figure*}
        \begin{center}\adjustimage{max size={0.9\linewidth}{0.8\paperheight}}{PerceivedIncomeRisk_files/PerceivedIncomeRisk_6_0.png}\end{center}
        \caption{Perceived Income Risks and Stock Market Return}
        \label{fig:tssp500}
    \end{figure*}
    

\begin{Verbatim}[commandchars=\\\{\}]
{\color{outcolor}Out[{\color{outcolor}17}]:}     Unnamed: 0     0     1        2        3      4        5
         0   median:var -0.08 -0.14  -0.33**    -0.17  -0.08    -0.15
         1   median:iqr  -0.1 -0.13  -0.33**    -0.15  -0.08    -0.19
         2  median:rvar -0.07 -0.01    -0.17    -0.09  -0.13  -0.33**
         3    mean:skew                                              
         4     mean:var -0.03 -0.09   -0.24*    -0.08  -0.13     -0.2
         5     mean:iqr -0.05 -0.07  -0.29**    -0.16  -0.12    -0.18
         6    mean:rvar -0.09 -0.01    -0.08    -0.12  -0.12   -0.23*
         7    mean:skew -0.07 -0.12     -0.0  -0.27**  0.24*    -0.02
\end{Verbatim}
            
    \hypertarget{role-of-individual-characteristics}{%
\subsection{\texorpdfstring{\href{MicroRiskProfile.ipynb}{Role of
individual
characteristics}}{Role of individual characteristics}}\label{role-of-individual-characteristics}}

What factors are associated with subjective riskiness of labor income?
This section inspects the question by regressing the perceived income
risks at individual level on three major blocks of variables:
job-specific characteristics, household demographics and other
macroeconomic expectations held by the respondent.

In a general form, the regression is specified as followed, where the
dependent variable is one of the individual subjetive moments that
represent perceived income risks for either nominal or real earning.

\begin{eqnarray}
\{\overline{var}_{i,t}, \overline{var}^2_{i,t}, \overline{iqr}_{i,t}\} = \alpha + \beta_0 \textrm{HH}_{i,t} + \beta_1 \textrm{JobType}_{i,t} + \beta_2 \textrm{Exp}_{i,t} + \beta_3 \textrm{Month}_t + \epsilon_{i,t}
\end{eqnarray}

The first block of factors, as called \(\textit{Jobtype}_{i,t}\)
includes dummy variables indicating if the job is part-time or if the
work is for others or self-employed. Since the earning growth is
specifically asked regarding the current job of the individual, I can
directly test if a part-time job and the self-employed job is associated
with higher perceived risks.

The second type of factors denoted \(\textit{HH}_{i,t}\) represents
household-specific demographics such as the household income level,
education, and gender of the respondent.

Third, \(\textit{Exp}_{i,t}\) represents other subjective expectations
held by the same individual. As far as this paper is concerned, I
include the perceived probability of unemployment herself, the
probability of stock market rise over and the probability of a higher
nationwide unemployment rate.

\(\textit{Month}_t\) is meant to control possible seasonal or
month-of-the-year fixed effects. It may well be the case that at a
certain point of the time of the year, workers are more likely to learn
about news to their future earnings. But as I will show in the following
section, such evidence is limited particularly for the higher moments of
earnings growth expectations.

Besides, since many of the regressors are time-invariant household
characteristics, I choose not to control household fixed effects in
these regressions (\(\omega_i\)). Throughout all specifications, I
cluster standard errors at the household level because of the concern of
unobservable household heterogeneity. The regression results are
presented in the table below for three measures of perceived income
risks, nominal growth variance, nominal growth iqr, and real growth
variance.

\href{../Table/mom_ind_reg.cvs}{Table of Individual Moments Regression}

The regression results are rather intuitive. It confirms that
self-employed jobs, workers from low-income households and lower
education have higher perceived income risks. In our sample, there are
around \(15\%\) (6000) of the individuals who report themselves to be
self-employed instead of working for others. In the table shown in the
appendix, this group of people also has higher expected earnings growth.
The effects are statistically and economically significant. Whether a
part-time job is associated with higher perceived risk is ambiguous
depending on if we control household demographics. At first sight,
part-time jobs may be thought of as more unstable. But the exact nature
of part-time job varies across different types and populations. It is
possible, for instance, that the part-time jobs available to high-income
and educated workers bear lower risks than those by the low-income and
low-education groups.

The negative correlation between perceived risks and household income is
significant and robust throughout all specifications. In contrast, there
is no such correlation between expected earning growth per se and
household income. Although SCE asks the respondent to report an income
range instead of the accurate monetary value, the 11-group breakdown is
sufficiently granular to examine if the high-income/low risks
association is monotonic. As implied by the size of the coefficient of
each income group dummy in the table, this pattern is monotonically
negative until the top income group (\$200k or above). I also plot the
mean and median of income risks by income group in the figure below.

\href{'../Graphs/ind/riskbyincome.jpg'}{Figure on Perceived Income Risks
by Income Group}

Besides household income, there is also statistical correlation between
perceived risks and other demographic variables. In particular, higher
eduation, being a male versus female, being a middle-aged worker
compared to a young, are all associated with lower perceived income
risks. To keep a sufficiently large sample size, I run regressions of
this set of variables without controling the rest regressors. Although
the sample size shrink substantially by including these demographics,
the relationships are statistically significant and consistent across
all measures of earning risks.

Higher perceived the probability of losing the current job, which I call
individual unemployment risk, \(\textit{IndUE}\) is associated with
higher earning risks of the current job. The perceived chance that the
nationwide unemployment rate going up next year, which I call aggregate
unemployment risk, \(\textit{AggUE}\) has a similar correlation with
perceived earning risks. Such a positive correlation is important
because this implies that a more comprehensively measured income risk
facing the individual that incorporates not only the current job's
earning risks but also the risk of unemployment is actually higher.
Moreover, the perceived risk is higher for those whose perceptions of
the earning risk and unemployment risk are more correlated than those
less correlated.

Lastly, what is ambiguous from the regression is the correlation between
stock market expectations and perceived income risks. Although a more
positive stock market expectation is associated with higher expected
earnings growth in both real and nominal terms, it is positively
correlated with nominal earning risks but negatively correlated with
real earning risks. As the real earning risk is the summation of the
perceived risk of nominal earning and inflation uncertainty, the sign
difference has to be driven by a negative correlation of expectation
stock market and inflation uncertainty. In order to reach more
conclusive statements, in the next section, I will examine how perceived
labor income risks correlate with the realized stock market returns and
indicators of business cycles.

To summerize, a few questions arise from the patterns discussed above.
First, what drives the differences in subjective earning risks across
different workers? To what extent these perceptive differences reflect
the true heterogeneity of the income risks facing by these individuals?
Or they can be attributed to perceptive heterogeneity independent from
the true risk profile. Second, how are individual earning risk is
correlated with asset return expectations and broadly the macro economic
environment? This will be the focus of the coming sections.


    \begin{figure*}
        \begin{center}\adjustimage{max size={0.9\linewidth}{0.8\paperheight}}{PerceivedIncomeRisk_files/PerceivedIncomeRisk_9_0.png}\end{center}
        \caption{Perceived Income by Group}
        \label{fig:boxplotbygroup}
    \end{figure*}
    


\begin{Verbatim}[commandchars=\\\{\}]
{\color{outcolor}Out[{\color{outcolor}10}]:}             Unnamed: 0 Q24\_var I Q24\_var II Q24\_var III Q24\_var IIII  \textbackslash{}
         0     C(HHinc)[T.10.0]                         -4.32***                
         1                                                (0.54)                
         2     C(HHinc)[T.11.0]                         -2.23***                
         3                                                (0.55)                
         4      C(HHinc)[T.2.0]                          -1.25**                
         5                                                (0.59)                
         6      C(HHinc)[T.3.0]                         -1.63***                
         7                                                (0.55)                
         8      C(HHinc)[T.4.0]                         -2.27***                
         9                                                (0.55)                
         10     C(HHinc)[T.5.0]                         -3.61***                
         11                                               (0.54)                
         12     C(HHinc)[T.6.0]                         -3.70***                
         13                                               (0.54)                
         14     C(HHinc)[T.7.0]                         -4.15***                
         15                                               (0.53)                
         16     C(HHinc)[T.8.0]                         -4.24***                
         17                                               (0.53)                
         18     C(HHinc)[T.9.0]                         -4.47***                
         19                                               (0.53)                
         20   C(gender)[T.male]                                      -0.88***   
         21                                                            (0.29)   
         22  C(parttime)[T.yes]     -0.19      -0.01    -0.81***                
         23                        (0.13)     (0.13)      (0.15)                
         24   C(selfemp)[T.yes]   7.36***   -0.00***     0.00***                
         25                        (0.15)     (0.00)      (0.00)                
         26                        (0.06)     (0.11)      (0.52)       (0.49)   
         27             Stkprob               0.01**     0.01***                
         28                                   (0.00)      (0.00)                
         29           UEprobInd              0.03***     0.03***                
         30                                   (0.00)      (0.00)                
         31                educ                                       -0.25**   
         32                                                            (0.10)   
         33                   N     46027      39868       34475         6872   
         34                  R2      0.05       0.00        0.01         0.00   
         
            Q24\_rvar I Q24\_rvar II Q24\_rvar III Q24\_rvar IIII  
         0                            -13.64***                
         1                               (1.08)                
         2                            -11.70***                
         3                               (1.10)                
         4                             -3.52***                
         5                               (1.18)                
         6                              -2.67**                
         7                               (1.09)                
         8                             -4.56***                
         9                               (1.09)                
         10                            -8.25***                
         11                              (1.08)                
         12                            -8.55***                
         13                              (1.08)                
         14                           -10.43***                
         15                              (1.06)                
         16                           -11.33***                
         17                              (1.05)                
         18                           -12.91***                
         19                              (1.05)                
         20                                           4.47***  
         21                                            (0.57)  
         22    0.91***     1.28***     -1.53***                
         23     (0.25)      (0.27)       (0.29)                
         24    6.16***     0.00***     -0.00***                
         25     (0.29)      (0.00)       (0.00)                
         26     (0.11)      (0.23)       (1.04)        (0.97)  
         27               -0.05***     -0.04***                
         28                 (0.00)       (0.00)                
         29                0.06***      0.05***                
         30                 (0.01)       (0.01)                
         31                                          -2.09***  
         32                                            (0.19)  
         33      43237       37397        32614          6165  
         34       0.01        0.01         0.04          0.03  
\end{Verbatim}
            
    \hypertarget{perceived-income-risks-and-decision-preliminary}{%
\subsection{Perceived income risks and decision
(preliminary)}\label{perceived-income-risks-and-decision-preliminary}}

Need to be careful with the bias toward zero due to the noisiness of
subjective risk measure. But if it is negative sign, the bias goes
against me. Therefore, less of a concern.

\begin{itemize}
\tightlist
\item
  Higher income risks may lead to lower household spending.
\item
  Particularly so for durable good.
\item
  Higher income risks is also associated with higher chance to
  voluntarily leave the job.
\end{itemize}

    \hypertarget{perceived-income-risks-and-persistence}{%
\section{Perceived income risks and
persistence}\label{perceived-income-risks-and-persistence}}

\hypertarget{an-illustration-of-the-idea-in-a-permanent-transitory-income-process}{%
\subsection{An illustration of the idea in a permanent-transitory income
process}\label{an-illustration-of-the-idea-in-a-permanent-transitory-income-process}}

The income process of individual \(i\) is the following

\begin{equation}
\begin{split}
y_{i,t} = P_{i,t} + \epsilon_{i,t} \\
P_{i,t} = P_{i,t-1} + \theta_{i,t} \\
\theta_{i,t} \sim N(0,\sigma_{\theta,t}) \\
\epsilon_{i,t} \sim N(0,\sigma_{\epsilon,t})
\end{split}
\end{equation}

Notice transitory and permanent risks are time-varying. For now, we do
not break down the individual into different cohorts, i.e.
\(\sigma_{\theta,t}\) and \(\sigma_{\epsilon,t}\) are not cohort
specific. But we can do this exercise for any defined cohort.

Income growth is

\begin{equation}
\begin{split}
\Delta y_{i,t+1} = y_{i,t+1} - y_{i,t} \\
 = P_{i,t+1} + \epsilon_{i,t+1} - P_{i,t} - \epsilon_{i,t} \\
 = \theta_{i,t+1} + \Delta \epsilon_{i,t+1}
\end{split}
\end{equation}

Assuming the agent knows perfectly the income process, then standing at
time \(t\), the conditional variance of income growth for next period is

\begin{equation}
Var^*_{i,t}(\Delta y_{i,t+1}) = \tilde \sigma^2_{\theta,t+1} + \tilde \sigma^2_{\epsilon,t+1} \quad \forall i
\end{equation}

where we use \(\tilde{}\) supscript to denote the perceived risks.
Because of rational expectation, the agent learns about the realization
of \(\sigma_{\epsilon,t}\), therefore it does not show up in her
uncertainty.

In the same time, the cross-cetional variance of the expected income
growth at time \(t\) about income growth reflects the different views of
the risks.

\begin{equation}
\overline {Var}^*_{t}(E_{i}(\Delta y_{i,t+1})) = \tilde \sigma^2_{\theta,t+1} +\tilde \sigma^2_{\epsilon,t}+ \tilde \sigma^2_{\epsilon,t+1}
\end{equation}

The autocovariance of expected income growth in consecutive two periods
is as follows.

\begin{equation}
\overline {Cov}^*_{t+1|t}(E_{i,t}(\Delta y_{i,t+1}),E_{i,t+1}(\Delta y_{i,t+2}) ) = - \tilde \sigma^2_{\epsilon,t+1}
\end{equation}

The three moments exactly identify the perceived income risks in each
period. One way to think about these risks is that they are revealed by
people's forecasts.

These moments restrictions exactly mirrors the problem faced by
econometricians who have only access to the realized earnings in a panel
structure.

What is available to econometricians is the realized cross-sectional
variance of income growth (no subscript \(i\)) shown below. It is
different from uncertainty faced with individuals.

\begin{equation}
Var (\Delta y_{i,t+1}) =  \sigma^2_{\theta,t+1} +\sigma^2_{\epsilon,t}+ \sigma^2_{\epsilon,t+1}
\end{equation}

Taking the differences of the population's analogue of the first
equation and the second above recover variance of transitory risks
\(\sigma_{\epsilon,t}\). Recursively using the panel structure, we could
recover all the transitory and permanent income risks.

Besides, econometricians also use the following moments.

\begin{equation}
Cov (\Delta y_{i,t}, \Delta y_{i,t+1}) =  -\sigma^2_{\epsilon,t}
\end{equation}

This exercise is based on the assumption that individuals across the
population or one defined cohort share the same income process. And also
it is rational expectation in the sense that on average individuals get
the income process right.

Once we recover permanent and transitory volatilities from above
exercise, we can compare them with estimates from only realized income
serieses.

\hypertarget{other-moments-from-rational-expectation}{%
\subsection{Other moments from rational
expectation}\label{other-moments-from-rational-expectation}}

Besises, econometricians have utilized another moment restrictions: auto
correlation of income growth across two periods are \begin{equation}
Cov^*_{t}( \Delta y_t, \Delta y_{t+1} ) = \\
 = Cov^*_{t}(\theta_t + \epsilon_t - \epsilon_{t-1}, \theta_{t+1} + \epsilon_{t+1} - \epsilon_{t}) \\
 = 0 
\end{equation}

This is, again, different to an econometrician, for whom the covariance
is \(-\sigma^2_{\epsilon,t}\). The rational agent in the model learns
about \(\sigma_{\epsilon,t}\).

The serial covariance of expeced income growth across two periods are
\begin{equation}
Cov^*( E_{t-1}(\Delta y_t), E_t(\Delta y_{t+1}) ) = \\
= Cov^*(E_{t-1}(\theta_t +\epsilon_t - \epsilon_{t-1}), E_{t}(\theta_{t+1} + \epsilon_{t+1} - \epsilon_t)) \\
= 0
\end{equation}

    \hypertarget{time-aggregation-problem}{%
\subsection{Time aggregation problem}\label{time-aggregation-problem}}

\begin{itemize}
\tightlist
\item
  The earning growth asked is from \(m\) to \(m+12\).
\item
  The survey is asked each month.
\end{itemize}

    \hypertarget{a-simple-example-with-half-year-as-the-unit-of-the-time}{%
\subsubsection{A simple example with half-year as the unit of the
time}\label{a-simple-example-with-half-year-as-the-unit-of-the-time}}

Earning in year \(t\) is a summation of half-year earning.

\begin{equation}
y_t = y_{t_2}+ y_{t_2} 
\end{equation}

The YoY growth of income is below

\begin{equation}
\begin{split}
\Delta y_{t_2+1} = y_{(t+1)_1}+ y_{(t+1)_2} - y_{t_1 } - y_{t_2}  \\
 = p_{(t+1)_1} + \epsilon_{(t+1)_2} + p_{(t+1)_2} + \epsilon_{(t+1)_2} - p_{t_1} - \epsilon_{t_1} - p_{t_1} - \epsilon_{(t)_2 } \\
 = \theta_{(t)_2} + \theta_{(t+1)_1} + \theta_{(t+1)_2} + \theta_{(t+1)_1} + \epsilon_{(t+1)_1} + \epsilon_{(t+1)_2} - \epsilon_{t_1} - \epsilon_{t_2} \\
 =  \theta_{t_2} + 2\theta_{(t+1)_1} + \theta_{(t+1)_2} + \epsilon_{(t+1)_1} + \epsilon_{(t+1)_2} - \epsilon_{t_1} - \epsilon_{t_2} 
\end{split}
\end{equation}

The middle-year-on-middle-year income growth is

\begin{equation}
\begin{split}
\Delta y_{(t+1)_1+1} = y_{(t+1)_2}+ y_{(t+2)_1} - y_{(t+1)_1} - y_{t_2}  \\
 = p_{(t+1)_2} + \epsilon_{(t+1)_2} + p_{(t+2)_1} + \epsilon_{(t+2)_1} - p_{(t+1)_1} - \epsilon_{(t+1)_1} - p_{t_2} - \epsilon_{t_2 } \\
 = \theta_{(t+1)_2} + \theta_{(t+1)_1} + \theta_{(t+1)_2} + \theta_{(t+2)_1} + \epsilon_{(t+1)_2} + \epsilon_{(t+2)_1} - \epsilon_{(t+1)_1} - \epsilon_{t_2 } \\
 = 2\theta_{(t+1)_2} + \theta_{(t+1)_1} + \theta_{(t+2)_1} + \epsilon_{(t+1)_2} + \epsilon_{(t+2)_1} - \epsilon_{(t+1)_1} - \epsilon_{t_2 }
\end{split}
\end{equation}

Then for each individual \(i\) at \(t''\) and \((t+1)'\) are
respectively:

\begin{equation}
Var^*_{i,t_2}(\Delta y_{i,t_2+1}) =  2\sigma^2_{\theta,(t+1)_1} + \sigma^2_{\theta,(t+1)_2} + \sigma^2_{\epsilon,(t+1)_1} + \sigma^2_{\epsilon,(t+1)_2}
\end{equation}

\begin{equation}
Var^*_{i,(t+1)_1}(\Delta y_{i,(t+1)_1+1}) =  2\sigma^2_{\theta,(t+1)_2} + \sigma^2_{\theta,(t+2)_1} + \sigma^2_{\epsilon,(t+1)_2} + \sigma^2_{\epsilon,(t+2)_1}
\end{equation}

From end of \(t_2\) (end of year \(t\)) to the end of \((t+1)_1\)
(middle of the year \(t+1\)), the realization of \(\theta_{(t+1)_1}\)
and \(\epsilon_{(t+1)_1}\) reduces the variance.

Besides, the econometricians have access to following two
cross-sectional moments.

\begin{equation}
Var (\Delta y_{i,t_2+1}) =  \sigma^2_{\theta,t_2} + 2\sigma^2_{\theta,(t+1)_1} + \sigma^2_{\theta,(t+1)_2} + \sigma^2_{\epsilon,(t+1)_1} + \sigma^2_{\epsilon,(t+1)_2} + \sigma^2_{\epsilon,t_1} + \sigma^2_{\epsilon,t_2} 
\end{equation}

\begin{equation}
Var (\Delta y_{i,(t+1)_1+1}) =  2\sigma^2_{\theta,(t+1)_2} + \sigma^2_{\theta,(t+1)_1} + \sigma^2_{\theta,(t+2)_1} + \sigma^2_{\epsilon,(t+1)_2} + \sigma^2_{\epsilon,(t+2)_1} + \sigma^2_{\epsilon,(t+1)_1} + \sigma^2_{\epsilon,t_2}
\end{equation}

\begin{equation}
\begin{split}
Cov ( \Delta y_{i,(t-1)_2+1},\Delta y_{i,t_1+1}) = Cov(\theta_{(t-1)_2} + 2\theta_{t_1} + \theta_{t_2} + \epsilon_{t_1} + \epsilon_{t_2} - \epsilon_{(t-1)_1} - \epsilon_{(t-1)_2} , \\
2\theta_{t_2} + \theta_{t_1} + \theta_{(t+1)_1} + \epsilon_{t_2} + \epsilon_{(t+1)_1} - \epsilon_{t_1} - \epsilon_{(t-1)_2 } ) \\
= 2\sigma^2_{\theta,t_1} + 2\sigma^2_{\theta,t_2} - \sigma^2_{\epsilon,t_1} + \sigma^2_{\epsilon,t_2} + \sigma^2_{\epsilon,(t-1)_2}
\end{split}
\end{equation}

\begin{equation}
\begin{split}
Cov ( \Delta y_{i,(t-1)_2+1},\Delta y_{i,t_2+1}) = Cov(\theta_{(t-1)_2} + 2\theta_{t_1} + \theta_{t_2} + \epsilon_{t_1} + \epsilon_{t_2} - \epsilon_{(t-1)_1} - \epsilon_{(t-1)_2} , \\
\theta_{t_2} + 2\theta_{(t+1)_1} + \theta_{(t+1)_2} + \epsilon_{(t+1)_1} + \epsilon_{(t+1)_2} - \epsilon_{t_1} - \epsilon_{t_2} ) \\
= \sigma^2_{\theta,t_2}-(\sigma^2_{\epsilon,(t+1)_1} + \sigma^2_{\epsilon,t_2})
\end{split}
\end{equation}

\begin{equation}
\begin{split}
Cov ( \Delta y_{i,t_2+1},\Delta y_{i,(t+1)_2}) = \sigma^2_{\theta,(t+1)_2}-(\sigma^2_{\epsilon,(t+2)_1} + \sigma^2_{\epsilon,(t+1)_2})
\end{split}
\end{equation}

The rational expectation assumption also gives following moment
restrictions

\begin{equation}
Cov^*_{t_2}(\Delta y_t, \Delta y_{t+1}) = 0
\end{equation}

Standing at any point of the time, for the rational agent, the
\(\Delta y_t\) is realizated already. So it should have zero covariance
with income growth in future.

This is again, different from the econometrician's problem.

    \hypertarget{model-in-progress}{%
\section{Model (in progress)}\label{model-in-progress}}

    \hypertarget{summary}{%
\section{Summary}\label{summary}}

    \hypertarget{reference}{%
\section{Reference}\label{reference}}


    % Add a bibliography block to the postdoc
    
    
\bibliographystyle{apalike}
\bibliography{PerceivedIncomeRisk}

    
    \end{document}
