\documentclass[12pt,notitlepage,onecolumn,aps,pra]{article}


    
\usepackage[T1]{fontenc}
\usepackage{graphicx}
% We will generate all images so they have a width \maxwidth. This means
% that they will get their normal width if they fit onto the page, but
% are scaled down if they would overflow the margins.
\makeatletter
\def\maxwidth{\ifdim\Gin@nat@width>\linewidth\linewidth
\else\Gin@nat@width\fi}
\makeatother
\let\Oldincludegraphics\includegraphics
% Set max figure width to be 80% of text width, for now hardcoded.
\renewcommand{\includegraphics}[1]{\Oldincludegraphics[width=.8\maxwidth]{#1}}
% Ensure that by default, figures have no caption (until we provide a
% proper Figure object with a Caption API and a way to capture that
% in the conversion process - todo).
\usepackage{caption}
\usepackage{adjustbox} % Used to constrain images to a maximum size
\usepackage{xcolor} % Allow colors to be defined
\usepackage{enumerate} % Needed for markdown enumerations to work
\usepackage{geometry} % Used to adjust the document margins
\usepackage{amsmath} % Equations
\usepackage{amssymb} % Equations
\usepackage{textcomp} % defines textquotesingle
% Hack from http://tex.stackexchange.com/a/47451/13684:
\AtBeginDocument{%
    \def\PYZsq{\textquotesingle}% Upright quotes in Pygmentized code
}
\usepackage{upquote} % Upright quotes for verbatim code
\usepackage{eurosym} % defines \euro
\usepackage[mathletters]{ucs} % Extended unicode (utf-8) support
\usepackage[utf8x]{inputenc} % Allow utf-8 characters in the tex document
\usepackage{fancyvrb} % verbatim replacement that allows latex
\usepackage{grffile} % extends the file name processing of package graphics
                     % to support a larger range
% The hyperref package gives us a pdf with properly built
% internal navigation ('pdf bookmarks' for the table of contents,
% internal cross-reference links, web links for URLs, etc.)
\usepackage{hyperref}
\usepackage{natbib}
\usepackage{booktabs}  % table support for pandoc > 1.12.2
\usepackage[inline]{enumitem} % IRkernel/repr support (it uses the enumerate* environment)
\usepackage[normalem]{ulem} % ulem is needed to support strikethroughs (\sout)
                            % normalem makes italics be italics, not underlines
\usepackage{braket}

\usepackage{rotating}
\usepackage{threeparttable}
\usepackage{subcaption}



    
    % Colors for the hyperref package
    \definecolor{urlcolor}{rgb}{0,.145,.698}
    \definecolor{linkcolor}{rgb}{.71,0.21,0.01}
    \definecolor{citecolor}{rgb}{.12,.54,.11}

    % ANSI colors
    \definecolor{ansi-black}{HTML}{3E424D}
    \definecolor{ansi-black-intense}{HTML}{282C36}
    \definecolor{ansi-red}{HTML}{E75C58}
    \definecolor{ansi-red-intense}{HTML}{B22B31}
    \definecolor{ansi-green}{HTML}{00A250}
    \definecolor{ansi-green-intense}{HTML}{007427}
    \definecolor{ansi-yellow}{HTML}{DDB62B}
    \definecolor{ansi-yellow-intense}{HTML}{B27D12}
    \definecolor{ansi-blue}{HTML}{208FFB}
    \definecolor{ansi-blue-intense}{HTML}{0065CA}
    \definecolor{ansi-magenta}{HTML}{D160C4}
    \definecolor{ansi-magenta-intense}{HTML}{A03196}
    \definecolor{ansi-cyan}{HTML}{60C6C8}
    \definecolor{ansi-cyan-intense}{HTML}{258F8F}
    \definecolor{ansi-white}{HTML}{C5C1B4}
    \definecolor{ansi-white-intense}{HTML}{A1A6B2}
    \definecolor{ansi-default-inverse-fg}{HTML}{FFFFFF}
    \definecolor{ansi-default-inverse-bg}{HTML}{000000}

    % commands and environments needed by pandoc snippets
    % extracted from the output of `pandoc -s`
    \providecommand{\tightlist}{%
      \setlength{\itemsep}{0pt}\setlength{\parskip}{0pt}}
    \DefineVerbatimEnvironment{Highlighting}{Verbatim}{commandchars=\\\{\}}
    % Add ',fontsize=\small' for more characters per line
    \newenvironment{Shaded}{}{}
    \newcommand{\KeywordTok}[1]{\textcolor[rgb]{0.00,0.44,0.13}{\textbf{{#1}}}}
    \newcommand{\DataTypeTok}[1]{\textcolor[rgb]{0.56,0.13,0.00}{{#1}}}
    \newcommand{\DecValTok}[1]{\textcolor[rgb]{0.25,0.63,0.44}{{#1}}}
    \newcommand{\BaseNTok}[1]{\textcolor[rgb]{0.25,0.63,0.44}{{#1}}}
    \newcommand{\FloatTok}[1]{\textcolor[rgb]{0.25,0.63,0.44}{{#1}}}
    \newcommand{\CharTok}[1]{\textcolor[rgb]{0.25,0.44,0.63}{{#1}}}
    \newcommand{\StringTok}[1]{\textcolor[rgb]{0.25,0.44,0.63}{{#1}}}
    \newcommand{\CommentTok}[1]{\textcolor[rgb]{0.38,0.63,0.69}{\textit{{#1}}}}
    \newcommand{\OtherTok}[1]{\textcolor[rgb]{0.00,0.44,0.13}{{#1}}}
    \newcommand{\AlertTok}[1]{\textcolor[rgb]{1.00,0.00,0.00}{\textbf{{#1}}}}
    \newcommand{\FunctionTok}[1]{\textcolor[rgb]{0.02,0.16,0.49}{{#1}}}
    \newcommand{\RegionMarkerTok}[1]{{#1}}
    \newcommand{\ErrorTok}[1]{\textcolor[rgb]{1.00,0.00,0.00}{\textbf{{#1}}}}
    \newcommand{\NormalTok}[1]{{#1}}
    
    % Additional commands for more recent versions of Pandoc
    \newcommand{\ConstantTok}[1]{\textcolor[rgb]{0.53,0.00,0.00}{{#1}}}
    \newcommand{\SpecialCharTok}[1]{\textcolor[rgb]{0.25,0.44,0.63}{{#1}}}
    \newcommand{\VerbatimStringTok}[1]{\textcolor[rgb]{0.25,0.44,0.63}{{#1}}}
    \newcommand{\SpecialStringTok}[1]{\textcolor[rgb]{0.73,0.40,0.53}{{#1}}}
    \newcommand{\ImportTok}[1]{{#1}}
    \newcommand{\DocumentationTok}[1]{\textcolor[rgb]{0.73,0.13,0.13}{\textit{{#1}}}}
    \newcommand{\AnnotationTok}[1]{\textcolor[rgb]{0.38,0.63,0.69}{\textbf{\textit{{#1}}}}}
    \newcommand{\CommentVarTok}[1]{\textcolor[rgb]{0.38,0.63,0.69}{\textbf{\textit{{#1}}}}}
    \newcommand{\VariableTok}[1]{\textcolor[rgb]{0.10,0.09,0.49}{{#1}}}
    \newcommand{\ControlFlowTok}[1]{\textcolor[rgb]{0.00,0.44,0.13}{\textbf{{#1}}}}
    \newcommand{\OperatorTok}[1]{\textcolor[rgb]{0.40,0.40,0.40}{{#1}}}
    \newcommand{\BuiltInTok}[1]{{#1}}
    \newcommand{\ExtensionTok}[1]{{#1}}
    \newcommand{\PreprocessorTok}[1]{\textcolor[rgb]{0.74,0.48,0.00}{{#1}}}
    \newcommand{\AttributeTok}[1]{\textcolor[rgb]{0.49,0.56,0.16}{{#1}}}
    \newcommand{\InformationTok}[1]{\textcolor[rgb]{0.38,0.63,0.69}{\textbf{\textit{{#1}}}}}
    \newcommand{\WarningTok}[1]{\textcolor[rgb]{0.38,0.63,0.69}{\textbf{\textit{{#1}}}}}
    
    
    % Define a nice break command that doesn't care if a line doesn't already
    % exist.
    \def\br{\hspace*{\fill} \\* }
    % Math Jax compatibility definitions
    \def\gt{>}
    \def\lt{<}
    \let\Oldtex\TeX
    \let\Oldlatex\LaTeX
    \renewcommand{\TeX}{\textrm{\Oldtex}}
    \renewcommand{\LaTeX}{\textrm{\Oldlatex}}
    % Document parameters
    % Document title
    
    
    
    
% Pygments definitions
\makeatletter
\def\PY@reset{\let\PY@it=\relax \let\PY@bf=\relax%
    \let\PY@ul=\relax \let\PY@tc=\relax%
    \let\PY@bc=\relax \let\PY@ff=\relax}
\def\PY@tok#1{\csname PY@tok@#1\endcsname}
\def\PY@toks#1+{\ifx\relax#1\empty\else%
    \PY@tok{#1}\expandafter\PY@toks\fi}
\def\PY@do#1{\PY@bc{\PY@tc{\PY@ul{%
    \PY@it{\PY@bf{\PY@ff{#1}}}}}}}
\def\PY#1#2{\PY@reset\PY@toks#1+\relax+\PY@do{#2}}

\expandafter\def\csname PY@tok@w\endcsname{\def\PY@tc##1{\textcolor[rgb]{0.73,0.73,0.73}{##1}}}
\expandafter\def\csname PY@tok@c\endcsname{\let\PY@it=\textit\def\PY@tc##1{\textcolor[rgb]{0.25,0.50,0.50}{##1}}}
\expandafter\def\csname PY@tok@cp\endcsname{\def\PY@tc##1{\textcolor[rgb]{0.74,0.48,0.00}{##1}}}
\expandafter\def\csname PY@tok@k\endcsname{\let\PY@bf=\textbf\def\PY@tc##1{\textcolor[rgb]{0.00,0.50,0.00}{##1}}}
\expandafter\def\csname PY@tok@kp\endcsname{\def\PY@tc##1{\textcolor[rgb]{0.00,0.50,0.00}{##1}}}
\expandafter\def\csname PY@tok@kt\endcsname{\def\PY@tc##1{\textcolor[rgb]{0.69,0.00,0.25}{##1}}}
\expandafter\def\csname PY@tok@o\endcsname{\def\PY@tc##1{\textcolor[rgb]{0.40,0.40,0.40}{##1}}}
\expandafter\def\csname PY@tok@ow\endcsname{\let\PY@bf=\textbf\def\PY@tc##1{\textcolor[rgb]{0.67,0.13,1.00}{##1}}}
\expandafter\def\csname PY@tok@nb\endcsname{\def\PY@tc##1{\textcolor[rgb]{0.00,0.50,0.00}{##1}}}
\expandafter\def\csname PY@tok@nf\endcsname{\def\PY@tc##1{\textcolor[rgb]{0.00,0.00,1.00}{##1}}}
\expandafter\def\csname PY@tok@nc\endcsname{\let\PY@bf=\textbf\def\PY@tc##1{\textcolor[rgb]{0.00,0.00,1.00}{##1}}}
\expandafter\def\csname PY@tok@nn\endcsname{\let\PY@bf=\textbf\def\PY@tc##1{\textcolor[rgb]{0.00,0.00,1.00}{##1}}}
\expandafter\def\csname PY@tok@ne\endcsname{\let\PY@bf=\textbf\def\PY@tc##1{\textcolor[rgb]{0.82,0.25,0.23}{##1}}}
\expandafter\def\csname PY@tok@nv\endcsname{\def\PY@tc##1{\textcolor[rgb]{0.10,0.09,0.49}{##1}}}
\expandafter\def\csname PY@tok@no\endcsname{\def\PY@tc##1{\textcolor[rgb]{0.53,0.00,0.00}{##1}}}
\expandafter\def\csname PY@tok@nl\endcsname{\def\PY@tc##1{\textcolor[rgb]{0.63,0.63,0.00}{##1}}}
\expandafter\def\csname PY@tok@ni\endcsname{\let\PY@bf=\textbf\def\PY@tc##1{\textcolor[rgb]{0.60,0.60,0.60}{##1}}}
\expandafter\def\csname PY@tok@na\endcsname{\def\PY@tc##1{\textcolor[rgb]{0.49,0.56,0.16}{##1}}}
\expandafter\def\csname PY@tok@nt\endcsname{\let\PY@bf=\textbf\def\PY@tc##1{\textcolor[rgb]{0.00,0.50,0.00}{##1}}}
\expandafter\def\csname PY@tok@nd\endcsname{\def\PY@tc##1{\textcolor[rgb]{0.67,0.13,1.00}{##1}}}
\expandafter\def\csname PY@tok@s\endcsname{\def\PY@tc##1{\textcolor[rgb]{0.73,0.13,0.13}{##1}}}
\expandafter\def\csname PY@tok@sd\endcsname{\let\PY@it=\textit\def\PY@tc##1{\textcolor[rgb]{0.73,0.13,0.13}{##1}}}
\expandafter\def\csname PY@tok@si\endcsname{\let\PY@bf=\textbf\def\PY@tc##1{\textcolor[rgb]{0.73,0.40,0.53}{##1}}}
\expandafter\def\csname PY@tok@se\endcsname{\let\PY@bf=\textbf\def\PY@tc##1{\textcolor[rgb]{0.73,0.40,0.13}{##1}}}
\expandafter\def\csname PY@tok@sr\endcsname{\def\PY@tc##1{\textcolor[rgb]{0.73,0.40,0.53}{##1}}}
\expandafter\def\csname PY@tok@ss\endcsname{\def\PY@tc##1{\textcolor[rgb]{0.10,0.09,0.49}{##1}}}
\expandafter\def\csname PY@tok@sx\endcsname{\def\PY@tc##1{\textcolor[rgb]{0.00,0.50,0.00}{##1}}}
\expandafter\def\csname PY@tok@m\endcsname{\def\PY@tc##1{\textcolor[rgb]{0.40,0.40,0.40}{##1}}}
\expandafter\def\csname PY@tok@gh\endcsname{\let\PY@bf=\textbf\def\PY@tc##1{\textcolor[rgb]{0.00,0.00,0.50}{##1}}}
\expandafter\def\csname PY@tok@gu\endcsname{\let\PY@bf=\textbf\def\PY@tc##1{\textcolor[rgb]{0.50,0.00,0.50}{##1}}}
\expandafter\def\csname PY@tok@gd\endcsname{\def\PY@tc##1{\textcolor[rgb]{0.63,0.00,0.00}{##1}}}
\expandafter\def\csname PY@tok@gi\endcsname{\def\PY@tc##1{\textcolor[rgb]{0.00,0.63,0.00}{##1}}}
\expandafter\def\csname PY@tok@gr\endcsname{\def\PY@tc##1{\textcolor[rgb]{1.00,0.00,0.00}{##1}}}
\expandafter\def\csname PY@tok@ge\endcsname{\let\PY@it=\textit}
\expandafter\def\csname PY@tok@gs\endcsname{\let\PY@bf=\textbf}
\expandafter\def\csname PY@tok@gp\endcsname{\let\PY@bf=\textbf\def\PY@tc##1{\textcolor[rgb]{0.00,0.00,0.50}{##1}}}
\expandafter\def\csname PY@tok@go\endcsname{\def\PY@tc##1{\textcolor[rgb]{0.53,0.53,0.53}{##1}}}
\expandafter\def\csname PY@tok@gt\endcsname{\def\PY@tc##1{\textcolor[rgb]{0.00,0.27,0.87}{##1}}}
\expandafter\def\csname PY@tok@err\endcsname{\def\PY@bc##1{\setlength{\fboxsep}{0pt}\fcolorbox[rgb]{1.00,0.00,0.00}{1,1,1}{\strut ##1}}}
\expandafter\def\csname PY@tok@kc\endcsname{\let\PY@bf=\textbf\def\PY@tc##1{\textcolor[rgb]{0.00,0.50,0.00}{##1}}}
\expandafter\def\csname PY@tok@kd\endcsname{\let\PY@bf=\textbf\def\PY@tc##1{\textcolor[rgb]{0.00,0.50,0.00}{##1}}}
\expandafter\def\csname PY@tok@kn\endcsname{\let\PY@bf=\textbf\def\PY@tc##1{\textcolor[rgb]{0.00,0.50,0.00}{##1}}}
\expandafter\def\csname PY@tok@kr\endcsname{\let\PY@bf=\textbf\def\PY@tc##1{\textcolor[rgb]{0.00,0.50,0.00}{##1}}}
\expandafter\def\csname PY@tok@bp\endcsname{\def\PY@tc##1{\textcolor[rgb]{0.00,0.50,0.00}{##1}}}
\expandafter\def\csname PY@tok@fm\endcsname{\def\PY@tc##1{\textcolor[rgb]{0.00,0.00,1.00}{##1}}}
\expandafter\def\csname PY@tok@vc\endcsname{\def\PY@tc##1{\textcolor[rgb]{0.10,0.09,0.49}{##1}}}
\expandafter\def\csname PY@tok@vg\endcsname{\def\PY@tc##1{\textcolor[rgb]{0.10,0.09,0.49}{##1}}}
\expandafter\def\csname PY@tok@vi\endcsname{\def\PY@tc##1{\textcolor[rgb]{0.10,0.09,0.49}{##1}}}
\expandafter\def\csname PY@tok@vm\endcsname{\def\PY@tc##1{\textcolor[rgb]{0.10,0.09,0.49}{##1}}}
\expandafter\def\csname PY@tok@sa\endcsname{\def\PY@tc##1{\textcolor[rgb]{0.73,0.13,0.13}{##1}}}
\expandafter\def\csname PY@tok@sb\endcsname{\def\PY@tc##1{\textcolor[rgb]{0.73,0.13,0.13}{##1}}}
\expandafter\def\csname PY@tok@sc\endcsname{\def\PY@tc##1{\textcolor[rgb]{0.73,0.13,0.13}{##1}}}
\expandafter\def\csname PY@tok@dl\endcsname{\def\PY@tc##1{\textcolor[rgb]{0.73,0.13,0.13}{##1}}}
\expandafter\def\csname PY@tok@s2\endcsname{\def\PY@tc##1{\textcolor[rgb]{0.73,0.13,0.13}{##1}}}
\expandafter\def\csname PY@tok@sh\endcsname{\def\PY@tc##1{\textcolor[rgb]{0.73,0.13,0.13}{##1}}}
\expandafter\def\csname PY@tok@s1\endcsname{\def\PY@tc##1{\textcolor[rgb]{0.73,0.13,0.13}{##1}}}
\expandafter\def\csname PY@tok@mb\endcsname{\def\PY@tc##1{\textcolor[rgb]{0.40,0.40,0.40}{##1}}}
\expandafter\def\csname PY@tok@mf\endcsname{\def\PY@tc##1{\textcolor[rgb]{0.40,0.40,0.40}{##1}}}
\expandafter\def\csname PY@tok@mh\endcsname{\def\PY@tc##1{\textcolor[rgb]{0.40,0.40,0.40}{##1}}}
\expandafter\def\csname PY@tok@mi\endcsname{\def\PY@tc##1{\textcolor[rgb]{0.40,0.40,0.40}{##1}}}
\expandafter\def\csname PY@tok@il\endcsname{\def\PY@tc##1{\textcolor[rgb]{0.40,0.40,0.40}{##1}}}
\expandafter\def\csname PY@tok@mo\endcsname{\def\PY@tc##1{\textcolor[rgb]{0.40,0.40,0.40}{##1}}}
\expandafter\def\csname PY@tok@ch\endcsname{\let\PY@it=\textit\def\PY@tc##1{\textcolor[rgb]{0.25,0.50,0.50}{##1}}}
\expandafter\def\csname PY@tok@cm\endcsname{\let\PY@it=\textit\def\PY@tc##1{\textcolor[rgb]{0.25,0.50,0.50}{##1}}}
\expandafter\def\csname PY@tok@cpf\endcsname{\let\PY@it=\textit\def\PY@tc##1{\textcolor[rgb]{0.25,0.50,0.50}{##1}}}
\expandafter\def\csname PY@tok@c1\endcsname{\let\PY@it=\textit\def\PY@tc##1{\textcolor[rgb]{0.25,0.50,0.50}{##1}}}
\expandafter\def\csname PY@tok@cs\endcsname{\let\PY@it=\textit\def\PY@tc##1{\textcolor[rgb]{0.25,0.50,0.50}{##1}}}

\def\PYZbs{\char`\\}
\def\PYZus{\char`\_}
\def\PYZob{\char`\{}
\def\PYZcb{\char`\}}
\def\PYZca{\char`\^}
\def\PYZam{\char`\&}
\def\PYZlt{\char`\<}
\def\PYZgt{\char`\>}
\def\PYZsh{\char`\#}
\def\PYZpc{\char`\%}
\def\PYZdl{\char`\$}
\def\PYZhy{\char`\-}
\def\PYZsq{\char`\'}
\def\PYZdq{\char`\"}
\def\PYZti{\char`\~}
% for compatibility with earlier versions
\def\PYZat{@}
\def\PYZlb{[}
\def\PYZrb{]}
\makeatother


    % For linebreaks inside Verbatim environment from package fancyvrb. 
    \makeatletter
        \newbox\Wrappedcontinuationbox 
        \newbox\Wrappedvisiblespacebox 
        \newcommand*\Wrappedvisiblespace {\textcolor{red}{\textvisiblespace}} 
        \newcommand*\Wrappedcontinuationsymbol {\textcolor{red}{\llap{\tiny$\m@th\hookrightarrow$}}} 
        \newcommand*\Wrappedcontinuationindent {3ex } 
        \newcommand*\Wrappedafterbreak {\kern\Wrappedcontinuationindent\copy\Wrappedcontinuationbox} 
        % Take advantage of the already applied Pygments mark-up to insert 
        % potential linebreaks for TeX processing. 
        %        {, <, #, %, $, ' and ": go to next line. 
        %        _, }, ^, &, >, - and ~: stay at end of broken line. 
        % Use of \textquotesingle for straight quote. 
        \newcommand*\Wrappedbreaksatspecials {% 
            \def\PYGZus{\discretionary{\char`\_}{\Wrappedafterbreak}{\char`\_}}% 
            \def\PYGZob{\discretionary{}{\Wrappedafterbreak\char`\{}{\char`\{}}% 
            \def\PYGZcb{\discretionary{\char`\}}{\Wrappedafterbreak}{\char`\}}}% 
            \def\PYGZca{\discretionary{\char`\^}{\Wrappedafterbreak}{\char`\^}}% 
            \def\PYGZam{\discretionary{\char`\&}{\Wrappedafterbreak}{\char`\&}}% 
            \def\PYGZlt{\discretionary{}{\Wrappedafterbreak\char`\<}{\char`\<}}% 
            \def\PYGZgt{\discretionary{\char`\>}{\Wrappedafterbreak}{\char`\>}}% 
            \def\PYGZsh{\discretionary{}{\Wrappedafterbreak\char`\#}{\char`\#}}% 
            \def\PYGZpc{\discretionary{}{\Wrappedafterbreak\char`\%}{\char`\%}}% 
            \def\PYGZdl{\discretionary{}{\Wrappedafterbreak\char`\$}{\char`\$}}% 
            \def\PYGZhy{\discretionary{\char`\-}{\Wrappedafterbreak}{\char`\-}}% 
            \def\PYGZsq{\discretionary{}{\Wrappedafterbreak\textquotesingle}{\textquotesingle}}% 
            \def\PYGZdq{\discretionary{}{\Wrappedafterbreak\char`\"}{\char`\"}}% 
            \def\PYGZti{\discretionary{\char`\~}{\Wrappedafterbreak}{\char`\~}}% 
        } 
        % Some characters . , ; ? ! / are not pygmentized. 
        % This macro makes them "active" and they will insert potential linebreaks 
        \newcommand*\Wrappedbreaksatpunct {% 
            \lccode`\~`\.\lowercase{\def~}{\discretionary{\hbox{\char`\.}}{\Wrappedafterbreak}{\hbox{\char`\.}}}% 
            \lccode`\~`\,\lowercase{\def~}{\discretionary{\hbox{\char`\,}}{\Wrappedafterbreak}{\hbox{\char`\,}}}% 
            \lccode`\~`\;\lowercase{\def~}{\discretionary{\hbox{\char`\;}}{\Wrappedafterbreak}{\hbox{\char`\;}}}% 
            \lccode`\~`\:\lowercase{\def~}{\discretionary{\hbox{\char`\:}}{\Wrappedafterbreak}{\hbox{\char`\:}}}% 
            \lccode`\~`\?\lowercase{\def~}{\discretionary{\hbox{\char`\?}}{\Wrappedafterbreak}{\hbox{\char`\?}}}% 
            \lccode`\~`\!\lowercase{\def~}{\discretionary{\hbox{\char`\!}}{\Wrappedafterbreak}{\hbox{\char`\!}}}% 
            \lccode`\~`\/\lowercase{\def~}{\discretionary{\hbox{\char`\/}}{\Wrappedafterbreak}{\hbox{\char`\/}}}% 
            \catcode`\.\active
            \catcode`\,\active 
            \catcode`\;\active
            \catcode`\:\active
            \catcode`\?\active
            \catcode`\!\active
            \catcode`\/\active 
            \lccode`\~`\~ 	
        }
    \makeatother

    \let\OriginalVerbatim=\Verbatim
    \makeatletter
    \renewcommand{\Verbatim}[1][1]{%
        %\parskip\z@skip
        \sbox\Wrappedcontinuationbox {\Wrappedcontinuationsymbol}%
        \sbox\Wrappedvisiblespacebox {\FV@SetupFont\Wrappedvisiblespace}%
        \def\FancyVerbFormatLine ##1{\hsize\linewidth
            \vtop{\raggedright\hyphenpenalty\z@\exhyphenpenalty\z@
                \doublehyphendemerits\z@\finalhyphendemerits\z@
                \strut ##1\strut}%
        }%
        % If the linebreak is at a space, the latter will be displayed as visible
        % space at end of first line, and a continuation symbol starts next line.
        % Stretch/shrink are however usually zero for typewriter font.
        \def\FV@Space {%
            \nobreak\hskip\z@ plus\fontdimen3\font minus\fontdimen4\font
            \discretionary{\copy\Wrappedvisiblespacebox}{\Wrappedafterbreak}
            {\kern\fontdimen2\font}%
        }%
        
        % Allow breaks at special characters using \PYG... macros.
        \Wrappedbreaksatspecials
        % Breaks at punctuation characters . , ; ? ! and / need catcode=\active 	
        \OriginalVerbatim[#1,codes*=\Wrappedbreaksatpunct]%
    }
    \makeatother

    % Exact colors from NB
    \definecolor{incolor}{HTML}{303F9F}
    \definecolor{outcolor}{HTML}{D84315}
    \definecolor{cellborder}{HTML}{CFCFCF}
    \definecolor{cellbackground}{HTML}{F7F7F7}
    
    % prompt
    \makeatletter
    \newcommand{\boxspacing}{\kern\kvtcb@left@rule\kern\kvtcb@boxsep}
    \makeatother
    \newcommand{\prompt}[4]{
        \ttfamily\llap{{\color{#2}[#3]:\hspace{3pt}#4}}\vspace{-\baselineskip}
    }
    

    
    % Prevent overflowing lines due to hard-to-break entities
    \sloppy 
    % Setup hyperref package
    \hypersetup{
      breaklinks=true,  % so long urls are correctly broken across lines
      colorlinks=true,
      urlcolor=urlcolor,
      linkcolor=linkcolor,
      citecolor=citecolor,
      }
    % Slightly bigger margins than the latex defaults
    
    \geometry{verbose,tmargin=1in,bmargin=1in,lmargin=1in,rmargin=1in}
    
    

\begin{document}
    
    \title{Perceived Income Risks}\author{Tao Wang}

\date{\today}
\maketitle\begin{abstract}What econometricians have assumed to be the labor income risks facing agents based on estimates from cross-sectional inequality may not necessarily be consistent with what is truely perceived.  This work in progress studies individual-specific perceived income risks utilizing a density survey of labor income. Empirically, I have found the earners who are female, from low-income households and low education has higher perceived risks. And both perceived variance and tail risk measure skewness are negatively correlated with stock market returns. The ongoing work includes characterizing how the subjective differences are driven by both the true income risk profile and perceptual heterogeneity from alternative mechanisms of expectation formation deviating from full-informational rationality. I will also incorporate empirical findings in an otherwise standard life-cycle model of consumption and portfolio choice to explore their implications on consumption insurance and asset pricing.\end{abstract}


    
    

    
    \hypertarget{the-research-question}{%
\section{The research question}\label{the-research-question}}

``The devil is in higher moments.'' Even if two people share identical
expected income and homogeneous preferences, different degrees of income
risks still lead to starkly different decisions such as
saving/consumption and portfolio choices. This is well understood in
models in which agents are inter-temporally risk-averse, or prudent, and
the risks associated with future marginal utility motivate precautionary
motives. The same logic carries through to models in which capital
income and portfolio returns are stochastic, and the risks of returns
naturally become the center of asset pricing. Such behavioral
regularities equipped with market incompleteness due to reasons such as
imperfect insurance and credit constraints have also been the
cornerstone assumptions used in the literature on heterogeneous-agent
macroeconomics.

Economists have long utilized cross-sectional distributions of realized
microdata to estimate the stochastic environments relevant to the
agents' decision, such as the income process. And then in modeling the
estimated risk profile is taken as parametric inputs and the individual
shocks are simply drawn from the shared distributions. (See
\cite{blundell_consumption_2008} as an example.) But one assumption
implicitly made when doing this is that the agents in the model
perfectly understand thus agree on the income risk profile imposed on
them. As shown by the actively developing literature on expectation
formation, in particular, the mounting evidence on heterogeneity in
economic expectations held by micro agents, this assumption seems to be
too stringent. To the extent that agents make decisions based on their
\emph{respective} perceptions, understanding the \emph{perceived} income
risk profile and its correlation structure with other macro variables
are the keys to explaining their behavior patterns.

This paper's goal is to understand the question discussed above by
directly shedding light on the subjective income profile using the
recently available density forecasts of labor income surveyed by New
York Fed's Survey of Consumer Expectation (SCE). What is special about
this survey is that agents are asked to provide histogram-type forecasts
of their earning growth over the next 12 months together with a set of
expectational questions about the macroeconomy. It is at a monthly
frequency and has a panel structure allowing for consecutive
observations of the same household over a horizon of 12 months. When the
individual density forecast is available, a parametric density
estimation can be made to obtain the individual-specific subjective
distribution. And higher moments reflecting the perceived income risks
such as variance, as well as the asymmetry of the distribution such as
skewness allow me to directly characterize the perceived risk profile
without relying on external estimates from cross-sectional microdata.
This provides the first-hand measured perceptions on income risks that
are truly relevant to individual decisions.

Empirically, I can immediately ask the following questions.

\begin{itemize}
\item
  How much heterogeneity is there across workers' perceived income
  risks? What factors, i.e.~household income, demographics, and other
  expectations, are correlated with the subjective risks in both
  individual and macro level?
\item
  To what extent this heterogeneity in perceptions align with the true
  income risks facing different population group, or at least partly
  attributed to perceptive differences due to heterogeneity in
  information and information processing, as discussed in many models of
  expectation formation?

  \begin{itemize}
  \tightlist
  \item
    If we treat the income risks identified from cross-sectional
    inequality by econometricians as a benchmark, to what extent are the
    risks perceived by the agents?

    \begin{itemize}
    \tightlist
    \item
      If agents know more than econometricians about their individual
      earnings, should the perceived risks be lower than the
      econometrician's estimates?
    \item
      Or actually, do agents, due to inattention or other information
      rigidity in learning about recently realized shocks, perceive the
      overall risk to be higher?
    \end{itemize}
  \end{itemize}
\item
  If the subjective income risk can be decomposed into components of
  varying persistence (i.e.~permanent vs transitory) based on assumed
  income process, it is possible to characterize potential deviations of
  perceptive income process from some well defined rational benchmark.

  \begin{itemize}
  \tightlist
  \item
    For instance, if agents overestimate their permanent income risks?
  \item
    If agents overestimate the persistence of the income process?
    \cite{rozsypal_overpersistence_2017}
  \item
    One step back, if the log-normality assumption of income progress
    consistent with the surveyed data. Or it has non-zero skewness? This
    can be jointly tested using higher moments of the density forecasts.
  \end{itemize}
\item
  Finally, not just the process of earning itself, but also its
  covariance with macro-environment, risky asset returns, matter a great
  deal. For instance, if perceived income risks are counter-cyclical, it
  has important labor supply and portfolio implications.
  (\cite{guvenen2014nature}, \cite{catherine_countercyclical_2019})
\end{itemize}

One of the key challenges when addressing these questions is to
separately account for the differences in perceived risks driven by
differences in underlying risk profiles, i.e.~the ``truth'', and the
rest driven by perceptive and informational heterogeneity. The most
straightforward way seems to be to compare econometrician's external
estimates of the income process using realized data and the perceived
from the subjective survey. But this approach implicitly assumes that
econometricians correctly specify the model of the income process and
ignores the likely superior information problem discussed above.
Therefore, in this paper, instead of simply assuming the external
estimate by econometricians is the true underlying income process, I
characterize the differences between perception and the true process by
jointly recovering the process using realized data and expectations
based on a particular well-defined theory of expectation formation. The
advantage of doing this is that one does not need to make a stringent
assumption about either agents' full rationality or econometricians'
correctness of model specification. It allows econometricians to utilize
the information from expectations to understand the true law of the
system. This is in a similar spirit to \cite{guvenen_inferring_2014},
although the author does not use expectation survey but the consumption
choice as the additional input for the joint estimation.

Theoretically, once I can document robustly some patterns of the
perceived income risks profile, it can ben incorporated into an
otherwise standard life-cycle model involving consumption/portfolio
decisions to explore its macro implications. Ex-ante, one may conjecture
a few of the following scenarios.

\begin{itemize}
\item
  If the subjective risks or skewness is found to be negatively
  correlated with the risky market return or business cycles, this
  exposes agents to more risks than a non-state-dependent income
  profile.
\item
  If according to the subjective risk profile, the downside risks are
  highly persistent than typically assumed, then it is in line with the
  rare disaster idea.
\item
  The perceptual differences lead to differences in MPCs, which is a
  different mechanism from credit-constraints and non-insurance of
  risks.
\end{itemize}

\hypertarget{relevant-literature-and-potential-contribution}{%
\subsection{Relevant literature and potential
contribution}\label{relevant-literature-and-potential-contribution}}

This paper is relevant to four lines of literature. First, the idea of
this paper echoes with an old problem in the consumption insurance
literature: `insurance or information' (\cite{pistaferri_superior_2001},
\cite{kaufmann_disentangling_2009},\cite{meghir2011earnings}). In any
empirical tests of consumption insurance or consumption response to
income, there is always a worry that what is interpreted as the shock
has actually already entered the agents' information set or exactly the
opposite. For instance, the notion of excessive sensitivity, namely
households consumption highly responsive to anticipated income shock,
maybe simply because agents have not incorporated the recently realized
shocks that econometricians assume so (\cite{flavin_excess_1988}). Also,
recently, in the New York Fed
\href{https://libertystreeteconomics.newyorkfed.org/2017/11/understanding-permanent-and-temporary-income-shocks.html}{blog},
the authors followed a similar approach to decompose the permanent and
transitory shocks. My paper shares a similar spirit with these studies
in the sense that I try to tackle the identification problem in the same
approach: directly using the expectation data and explicitly controlling
what are truly conditional expectations of the agents making the
decision. This helps economists avoid making assumptions on what is
exactly in the agents' information set. What differentiates my work from
other authors is that I focus on higher moments, i.e.~income risks and
skewness by utilizing the recently available density forecasts of labor
income. Previous work only focuses on the sizes of the realized shocks
and estimates the variance of the shocks using cross-sectional
distribution, while my paper directly studies the individual specific
variance of these shocks perceived by different individuals. This will
become clear in Section \ref{perceived-income-process-in-progress}.

Second, this paper is inspired by an old but recently reviving interest
in studying consumption/saving behaviors in models incorporating
imperfect expectations and perceptions. For instance,
\cite{rozsypal_overpersistence_2017} found that households' expectation
of income exhibits an over-persistent bias using both expected and
realized household income from Michigan household survey. The paper also
shows that incorporating such bias affects the aggregate consumption
function by distorting the cross-sectional distributions of marginal
propensity to consume(MPCs) across the population.
\cite{carroll_sticky_2018} reconciles the low micro-MPC and high
macro-MPCs by introducing to the model an information rigidity of
households in learning about macro news while being updated about micro
news. \cite{lian2019imperfect} shows that an imperfect perception of
wealth accounts for such phenomenon as excess sensitivity to current
income and higher MPCs out of wealth than current income and so forth.
My paper has a similar flavor to all of these works by exploring the
behavioral implications of households' perceptive imperfection. The
novelty of my paper lies in the primary focus on the implications of
heterogeneity in perceived higher moments such as risks and skewness.
Various theories of expectation formation have different predictions
about the cross-sectional and dynamic patterns of perceived risks. I
examine these predictions in this paper.

This paper also contributes to the literature studying expectation
formation using subjective surveys. There has been a long list of
``irrational expectation'' theories developed in recent decades on how
agents deviate from full-information rationality benchmark, such as
sticky expectation, noisy signal extraction, least-square learning, etc.
Also, empirical work has been devoted to testing these theories in a
comparable manner (\cite{coibion2012can}, \cite{fuhrer2018intrinsic}).
But it is fair to say that thus far, relatively little work has been
done on individual variables such as labor income, which may well be
more relevant to individual economic decisions. Therefore, understanding
expectation formation of the individual variables, in particular,
concerning both mean and higher moments, will provide fruitful insights
for macroeconomic modeling assumptions.

Lastly, the paper is indirectly related to the research that advocated
for eliciting probabilistic questions measuring subjective uncertainty
in economic surveys (\cite{manski_measuring_2004},
\cite{delavande2011measuring}, \cite{manski_survey_2018}). Although the
initial suspicion concerning to people's ability in understanding, using
and answering probabilistic questions is understandable,
\cite{bertrand_people_2001} and other works have shown respondents have
the consistent ability and willingness to assign a probability (or
``percent chance'') to future events. \cite{armantier_overview_2017}
have a thorough discussion on designing, experimenting and implementing
the consumer expectation surveys to ensure the quality of the responses.
Broadly speaking, the advocates have argued that going beyond the
revealed preference approach, availability to survey data provides
economists with direct information on agents' expectations and helps
avoids imposing arbitrary assumptions. This insight holds for not only
point forecast but also and even more importantly, for uncertainty,
because for any economic decision made by a risk-averse agent, not only
the expectation but also the perceived risks matter a great deal.


    \hypertarget{data-variables-and-density-estimation}{%
\section{Data, variables and density
estimation}\label{data-variables-and-density-estimation}}

\hypertarget{data}{%
\subsection{Data}\label{data}}

The data used for this paper is from the core module of Survey of
Consumer Expectation(SCE) conducted by the New York Fed, a monthly
online survey for a rotating panel of around 1,300 household heads. The
sample period in my paper spans from June 2013 to June 2018, in total of
60 months. This makes about 79000 household-year observations, among
which around 53,000 observations provide non-empty answers to the
density question on earning growth.

Particular relevant for my purpose, the questionnaire asks each
respondent to fill perceived probabilities of their same-job-hour
earning growth to pre-defined non-overlapping bins. The question is
framed as ``suppose that 12 months from now, you are working in the
exact same {[}``main'' if Q11\textgreater{}1{]} job at the same place
you currently work and working the exact same number of hours. In your
view, what would you say is the percent chance that 12 months from now:
increased by x\% or more?''.

As a special feature of the online questionnaire, the survey only moves
on to the next question if the probabilities filled in all bins add up
to one. This ensures the basic probabilistic consistency of the answers
crucial for any further analysis. Besides, the earning growth
expectation regarding exactly the same position, same hours and same
location has two important implications for my analysis. First, the
requirements help make sure the comparability of the answers across time
and also excludes the potential changes in earnings driven by endogenous
labor supply decisions, i.e.~working for longer hours. Second, the
earning expectations, risks and tail risks measured here are only
conditional. It excludes either unemployment, i.e.~likely a zero
earning, or an upward movement in the job ladder, i.e.~a different
earning growth rate. Therefore, it does not fully reflect the labor
income risk profile relevant to each individual.

In so far as we want to tease out the earning changes from voluntary
decisions, moving to a different job position should actually not be
included in earning expectations. Therefore only the exclusion of an
unemployment scenario is relevant for my purpose in characterizing the
labor income risks. But what is assuring me is that the bias due to
omission of unemployment risk is unambiguous. We could interpret the
moments of same-job-hour earning growth as an upper bound for the level
of growth rate and a lower bound for the income risk. To put it in
another way, the expected earning growth conditional on employment is
higher than the unconditional one, and the conditional earning risk is
lower than the unconditional one. At the same time, since SCE separately
elicits the perceived probability of losing the current job for each
respondent, I could adjust the measured labor income moments taking into
account the unemployment risk.

\hypertarget{density-estimation-and-variables}{%
\subsection{Density estimation and
variables}\label{density-estimation-and-variables}}

With the histogram answers for each individual in hand, I follow
\cite{engelberg_comparing_2009} to fit each of them with a parametric
distribution accordingly for three following cases. In the first case
when there are three or more intervals filled with positive
probabilities, it will be fitted with a generalized beta distribution.
In particular, if there is no open-ended bin on the left or right, then
2-parameter beta distribution is sufficient. If there is either
open-ended bin with positive probability, since the lower bound or upper
bound of the support needs to be determined, a 4-parameter beta
distribution is estimated. In the second case, in which there are
exactly 2 adjacent intervals with positive probabilities, it is fitted
with an isosceles triangular distribution. In the third case, if there
is only one positive-probability of interval only, i.e.~equal to one, it
is fitted with a uniform distribution.

I have a reason to discuss at length the exact procedures for density
distribution. It is important for this paper's purpose because I need to
make sure the estimation assumptions of the density distribution do not
mechanically distort my cross-sectional patterns of the estimated
moments. This is the most obviously seen in the tail risk measure,
skewness. The assumption of log normality of income process, common in
the literature (See again \cite{blundell_consumption_2008}), implicitly
assume zero skewness, i.e.~that the income increase and decrease are
equally likely. This may not be the case in our surveyed density for
many individuals. In order to account for this possibility, the assumed
density distribution should be flexible enough to allow for different
shapes of subjective distribution. Beta distribution fits this purpose
well. Of course, in the case of uniform and isosceles triangular
distribution, the skewness is zero by default. For those of you who may
wonder, the fractions of the density answers fitted with the beta,
uniform, and triangular distributions are, respectively, xxx, xxx, xxx
in our sample.

Since the microdata provided in the SCE website already includes the
estimated mean, variance and IQR by the staff economists following the
exact same approach, I directly use their estimates for these moments.
At the same time, for the measure of tail-risk, i.e.~skewness, as not
provided, I use my own estimates. I also confirm that my estimates and
theirs for the first two moments are correlated with a coefficient of
0.9.

For all the moment's estimates, there are inevitably extreme values.
This could be due to the idiosyncratic answers provided by the original
respondent, or some non-convergence of the numeric estimation program.
Therefore, for each moment of the analysis, I exclude top and bottom
\(5\%\) observations, leading to a sample size of around 45,000.

I also recognize what is really relevant to many economic decisions such
as consumption is real income instead of nominal income. Thanks to the
availability of inflation expectation and inflation uncertainty (also
estimated from density question) can be used to convert nominal earning
growth moments to real terms. In particular, the real earning growth
rate is expected nominal growth minus inflation expectation.

\begin{eqnarray}
\overline {\Delta y^{r}}_{i,t} = \overline\Delta y_{i,t} - \overline \pi_{i,t}
\end{eqnarray}

The variance associated with real earning growth, if we treat inflation
and nominal earning growth as two independent stochastic variables, is
equal to the summed variance of the two.

\begin{eqnarray}
\overline{var^{r}}_{i,t} = \overline{var}_{i,t} + \overline{var}_{i,t}(\pi_{t})
\end{eqnarray}

Not enough information is available for the same kind of transformation
of IQR and skewness from nominal to real, so I only use nominal
variables. Besides, as there are extreme values on inflation
expectations and uncertainty, I also exclude top and bottom \(5\%\) of
the observations. This further shrinks the sample, when using real
moments, to 36,000.

    \hypertarget{perceived-income-risks-basic-facts}{%
\section{Perceived income risks: basic
facts}\label{perceived-income-risks-basic-facts}}

\hypertarget{cross-sectional-heterogeneity}{%
\subsection{Cross-sectional
heterogeneity}\label{cross-sectional-heterogeneity}}

This section inspects some basic cross-sectional patterns of the subject
moments of labor income. In the Figure \ref{fig:histmoms} below, I plot
the histograms of \(\overline\Delta y_{i,t}\), \(\overline{var}_{i,t}\),
\(\overline {skw}_{i,t}\), \(\overline {\Delta y^{r}}_{i,t}\),
\(\overline{var^{r}}_{i,t}\).

First, expected income growth across the population exhibits a
dispersion ranging from a decrease of \(2-3\%\) to around an increase of
\(10\%\) in nominal terms. Given the well-known downward wage rigidity,
it is not surprising that most of the people expect a positive earning
growth. At the same time, the distribution of expected income growth is
right-skewed, meaning that more workers expect a smaller than larger
wage growth. What is interesting is that this cross-sectional
right-skewness in nominal earning disappears in expected real terms.
Expected earnings growth adjusted by individual inflation expectation
becomes symmetric around zero, ranging from a decrease of \(10\%\) to an
increase of \(10\%\). Real labor income increase and decrease are
approximately equally likely.

Second, as the primary focus of this paper, perceived income risks also
have a notable cross-sectional dispersion. For both measures of risks
variance and iqr, and in terms of both nominal and real terms, the
distribution is right-skewed with a long tail. Specifically, most of the
workers have perceived a variance of nominal earning growth ranging from
zero to \(20\) (a standard-deviation equivalence of \(4-4.5\%\) income
growth a year). But in the tail, some of the workers perceive risks to
be as high as \(7-8\%\) standard deviation a year. To have a better
sense of how large the risk is, consider a median individual in our
sample, who has an expected earning growth of \(2.4\%\), and a perceived
risk of \(1\%\) standard deviation. This implies by no means negligible
earning risk.

Third, the subjective skewness, an indicator of symmetry of the
perceived density or upper/lower tail risk, are distributed across
populations symmetrically around zero. It ranges from a left-skewness or
negative skewness of 0.6 to the same size of positive skewness or
right-skewness. Although one may think, based on the general knowledge
of the cross-sectional distribution of the earnings growth, a
right-skewness is more common, it turns out that approximately equal
proportion of the sample has left and right tails of their individual
earning growth expectation. It is important to note here that this
pattern is not particularly due to our density estimation assumptions.
Both uniform and isosceles triangular distribution deliver a skewness of
zero. (This is also why we can observe a clear cluster of the skewness
at zero.) Therefore, the non-zero skewness estimates in our sample are
both from the beta distribution cases, which is flexible enough to allow
both.


    \begin{figure}[!ht]
        \begin{center}\adjustimage{max size={0.9\linewidth}{0.4\paperheight}}{PerceivedIncomeRisk_files/PerceivedIncomeRisk_4_0.png}\end{center}
        \caption{Distribution of Individual Moments}
        \label{fig:histmoms}
    \end{figure}
    
    \hypertarget{correlation-with-asset-returns}{%
\subsection{Correlation with asset
returns}\label{correlation-with-asset-returns}}

It is not only the labor income risk profile per se but also the macro
risk profile, i.e.~how the labor income is correlated with risky asset
return and the business cycle, that is important for household
decisions. Since the short time period of my sample (2013M6-2018M5) has
not seen a single one business cycle, at least as defined by the NBER
recession committee, it poses a challenge for me to examine the
correlation between perceived risks and macroeconomic cycles. Therefore,
as the first stage of the analysis, I only focus on the correlation
between perceived risks and stock market returns.

Of course, there is a rationale in the first place to study stock market
return and labor income, as it bears critical implications for household
consumption insurance, equity premium, and participation puzzle. For
instance, a negative correlation of income risk and risky asset return
means households will be faced with higher risks of their total income
by investing in the stock market. Or a negative correlation between
skewness and stock market return, meaning a bigger income increase is
less likely in low-return times will also deter households from
participating in the stock market.

Following the most common practice in the finance literature, I use the
monthly return of the S\&P 500, computed from the beginning to the end
of the month, as an approximate of the stock market return. Over the
sample period, there are exactly two-thirds of the time marking a
positive return.

For a population summary statistic of individual moments of perceived
income growth, I take the median and mean across all respondents in the
survey for each point of the time. One may worry about the seasonality
of the monthly series of this kind. For instance, it is possible that
workers tend to learn news about their future earnings at a particular
month of the year, i.e.~end of the fiscal year when the wage contracts
are renegotiated and updated. Reasons of this kind may result in
seasonal patterns of the expected earning growth, variance and other
moments. Because my time series is too short in sample size to perform a
trustful seasonal adjustment, I check the seasonality by inspecting the
auto-correlation of each time series at different lags. As seen in the
figures in the appendix, although it seems that the average or median
earning growth per se has some seasonal patterns, there is no evidence
for higher moments, such as variance and skewness.

There are two crucial econometric considerations when we examine the
correlation between the subjective moments of earning growth and stock
return.

The first is the time-average or time-aggregation problem documented in
both empirical asset pricing and consumption insurance literature
(\cite{working_note_1960}, \cite{jagannathan_lazy_2007},
\cite{crawley_search_2019}). Variables such as consumption and earning
are interval measures, reported as an average over a period, while the
stock return is a spot measure computed between two points of the time.
As a result, if the unit of the time for the underlying income process
is at a higher frequency than the measured interval (an extreme case
being the continuous-time), the measured variable will exhibit upward
biased autocorrelation and correlation with other underlying random walk
series in the same frequency. In my context, such a problem can be
partly mitigated by the availability of monthly frequency of earning
expectations, if we assume the unit of time of the underlying stochastic
process is a month. Then the directly observed monthly correlation of
the two cannot be driven by the time aggregation problem. What also
becomes immediately clear from this consideration is that I should not
examine the correlation of the two series in moving average terms,
because it will cause the time aggregation problem. This point will be
discussed in greater detail in the next section when I decompose the
perceived income risks to different components of varying persistence.

The second issue regards which of the following, lagged, contemporaneous
or forward is the correct correlation one should look at. Considering
what is relevant to an individual making decisions are unrealized
stochastic shocks to both income and asset return, one should examine
the 1-year-ahead earning growth and its risks with the realized return
over the impeding 12 months at each point of the time.

With these considerations, in the Figure \ref{fig:tssp500}, I plot the
median perceived risk and skewness of both nominal and real earning
along with the contemporaneous stock market returns by the end of each
month (also true for the mean, see Figure \ref{micro_reg_exp} in
appendix.). In order to account for the fact that the survey is
undertaken in the middle of the month while the return is computed at
the end of the month, I take the lag the income moments by 1 or 2 months
when calculating the correlation coefficient. Table \ref{macro_corr}
reports correlation coefficients of between perceived risks and the
realized stock market return over the next 0-6 months. Although a
Pearson test of the correlation coefficients is only significant for a
2-month lag, overall, the income risks measured by variance and IQR for
both nominal and real earning post a negative correlation with the
realized stock return a few months ahead. The subjective skewness has
also a negative associated with the realized stock return in the near
future.

More caution is needed when interpreting the observed negative
association between perceived earning risks/skewness with stock market
returns. First, my sample period is short and has mostly posted positive
returns. Second, the pattern is based on a population median and mean of
the perceived income risks, and it does not account for any
household-specific characteristics. As we have seen in the
cross-sectional pattern, there are substantial variations across
individuals in their perceived income risks and skewness. Third, the
risk profile we consider here is only relevant for marginal
consumers/investors who at least have access to the stock market in the
first place. Therefore, it is worth exploring the correlation above
conditional on more individual characteristics.


    \begin{figure*}[!ht]
        \begin{center}\adjustimage{max size={0.9\linewidth}{0.8\paperheight}}{PerceivedIncomeRisk_files/PerceivedIncomeRisk_6_0.png}\end{center}
        \caption{Perceived Income Risks and Stock Market Return}
        \label{fig:tssp500}
    \end{figure*}
    





    \hypertarget{role-of-individual-characteristics}{%
\subsection{Role of individual
characteristics}\label{role-of-individual-characteristics}}

What factors are associated with subjective riskiness of labor income?
This section inspects the question by regressing the perceived income
risks at individual level on three major blocks of variables:
job-specific characteristics, household demographics and other
macroeconomic expectations held by the respondent.

In a general form, the regression is specified as followed, where the
dependent variable is one of the individual subjective moments that
represent perceived income risks for either nominal or real earning.

\begin{eqnarray}
\{\overline{var}_{i,t}, \overline{var}^r_{i,t}, \overline{iqr}_{i,t}\} = \alpha + \beta_0 \textrm{HH}_{i,t} + \beta_1 \textrm{JobType}_{i,t} + \beta_2 \textrm{Exp}_{i,t} + \beta_3 \textrm{Month}_t + \epsilon_{i,t}
\end{eqnarray}

The first block of factors, as called \(\textit{Jobtype}_{i,t}\)
includes dummy variables indicating if the job is part-time or if the
work is for others or self-employed. Since the earning growth is
specifically asked regarding the current job of the individual, I can
directly test if a part-time job and the self-employed job is associated
with higher perceived risks.

The second type of factors denoted \(\textit{HH}_{i,t}\) represents
household-specific demographics such as the household income level,
education, and gender of the respondent.

Third, \(\textit{Exp}_{i,t}\) represents other subjective expectations
held by the same individual. As far as this paper is concerned, I
include the perceived probability of unemployment herself, the
probability of stock market rise over the next year and the probability
of a higher nationwide unemployment rate.

\(\textit{Month}_t\) is meant to control possible seasonal or
month-of-the-year fixed effects. It may well be the case that at a
certain point of the time of the year, workers are more likely to learn
about news to their future earnings. But as I have shown in the previous
section, such evidence is limited particularly for the higher moments of
earnings growth expectations.

Besides, since many of the regressors are time-invariant household
characteristics, I choose not to control household fixed effects in
these regressions (\(\omega_i\)). Throughout all specifications, I
cluster standard errors at the household level because of the concern of
unobservable household heterogeneity. The regression results are
presented in Table \ref{micro_reg} below for three measures of perceived
income risks, nominal growth variance, nominal growth IQR, and real
growth variance.

The regression results are rather intuitive. It confirms that
self-employed jobs, workers from low-income households and lower
education have higher perceived income risks. In our sample, there are
around \(15\%\) (6000) of the individuals who report themselves to be
self-employed instead of working for others. In the Table
\ref{micro_reg_mean} shown in the appendix, this group of people also
has higher expected earnings growth. The effects are statistically and
economically significant. Whether a part-time job is associated with
higher perceived risk is ambiguous depending on if we control household
demographics. At first sight, part-time jobs may be thought of as more
unstable. But the exact nature of part-time job varies across different
types and populations. It is possible, for instance, that the part-time
jobs available to high-income and educated workers bear lower risks than
those by the low-income and low-education groups.

The negative correlation between perceived risks and household income is
significant and robust throughout all specifications. In contrast, there
is no such correlation between expected earning growth per se and
household income. Although SCE asks the respondent to report an income
range instead of the accurate monetary value, the 11-group breakdown is
sufficiently granular to examine if the high-income/low risks
association is monotonic. As implied by the size of the coefficient of
each income group dummy in the Table \ref{micro_reg}, this pattern is
monotonically negative until the top income group (\$200k or above). I
also plot the mean and median of income risks by income group in the
Figure \ref{fig:boxplotbygroup}.

Besides household income, there is also a statistical correlation
between perceived risks and other demographic variables. In particular,
higher education, being a male versus female, being a middle-aged worker
compared to a young, are all associated with lower perceived income
risks. To keep a sufficiently large sample size, I run regressions of
this set of variables without controlling the rest regressors. Although
the sample size shrinks substantially by including these demographics,
the relationships are statistically significant and consistent across
all measures of earning risks.

Higher perceived the probability of losing the current job, which I call
individual unemployment risk, \(\textit{IndUE}\) is associated with
higher earning risks of the current job. The perceived chance that the
nationwide unemployment rate going up next year, which I call aggregate
unemployment risk, \(\textit{AggUE}\) has a similar correlation with
perceived earning risks. Such a positive correlation is important
because this implies that a more comprehensively measured income risk
facing the individual that incorporates not only the current job's
earning risks but also the risk of unemployment is actually higher.
Moreover, the perceived risk is higher for those whose perceptions of
the earning risk and unemployment risk are more correlated than those
less correlated.

Lastly, what is ambiguous from the regression is the correlation between
stock market expectations and perceived income risks. Although a more
positive stock market expectation is associated with higher expected
earnings growth in both real and nominal terms, it is positively
correlated with nominal earning risks but negatively correlated with
real earning risks. As the real earning risk is the summation of the
perceived risk of nominal earning and inflation uncertainty, the sign
difference has to be driven by a negative correlation of expectation
stock market and inflation uncertainty. In order to reach more
conclusive statements, I will examine how perceived labor income risks
correlate with the realized stock market returns and indicators of
business cycles depending upon individual factors in the next step of
the analysis.

To summarize, a few questions arise from the patterns discussed above.
First, what drives the differences in subjective earning risks across
different workers? To what extent these perceptive differences reflect
the true heterogeneity of the income risks facing by these individuals?
Or they can be attributed to perceptive heterogeneity independent from
the true risk profile. Second, how are individual earning risk is
correlated with asset return expectations and broadly the macroeconomic
environment? This will be the focus of the coming sections.


    \begin{figure*}[!ht]
        \begin{center}\adjustimage{max size={0.9\linewidth}{0.8\paperheight}}{PerceivedIncomeRisk_files/PerceivedIncomeRisk_13_0.png}\end{center}
        \caption{Perceived Income by Group}
        \label{fig:boxplotbygroup}
    \end{figure*}
    


    \hypertarget{perceived-income-risks-and-decisions-in-progress}{%
\subsection{Perceived income risks and decisions (in
progress)}\label{perceived-income-risks-and-decisions-in-progress}}

This section investigates how individual-specific perceived risks are
correlated with household economic decisions such as consumption and
labor supply. I should note that the purpose of this exercise is not
primarily for causal inference at the current stage. Instead, it is
meant to check if the surveyed households demonstrate a certain degree
of in-survey consistency in terms of their perceptions and decision
inclinations.

In particular, I ask two questions based on the available survey answers
provided by the core module of the survey. First, are higher perceived
income risks associated with a lower anticipated household spending
growth? Second, are higher perceived income risks are associated with
actions of self-insurance such as seeking an alternative job. This can
be indirectly tested using the surveyed probability of voluntary
separation from the current job. In addition, supplementary modules of
SCE have also surveyed more detailed questions on spending decisions and
the labor market. (These I will examine in the next stage of the
analysis.)

There is one important econometric concern when I run regressions of the
decision variable on perceived risks due to the measurement error in the
regressor used here. In a typical OLS regression in which the regressor
has i.i.d. measurement errors, the coefficient estimate for the
imperfectly measured regressor will have a bias toward zero. For this
reason, if I find that willingness to consume is indeed negatively
correlated with perceived risks, taking into account the bias, it
implies that the correlation of the two is greater in the magnitude.

The empirical results will be reported in the next version of the draft.



    \hypertarget{perceived-income-process-in-progress}{%
\section{Perceived income process (in
progress)}\label{perceived-income-process-in-progress}}

The cross-sectional heterogeneity documented in the previous analysis in
perceived risks mask two aspects that are crutial to understanding the
real nature of the perceptual heterogeneity. First, to what extent these
subjetive risk profiles simply reflect the de facto differences in the
stochastic nature of the idiosyncratic labor income shocks? This is also
imperative if one wants to attribute part of the differences to any
deviations from full-information rational expectation. Second, do the
perceived risks contain different components that differ in terms of its
persistence? The risks associated with permanent and transitory shocks
have substantially different impacts on individual decisions. In order
to address both issues, I will need to specify a well-defined income
process.

\hypertarget{an-illustration-in-a-simple-two-component-income-process-and-rational-expectation}{%
\subsection{An illustration in a simple two-component income process and
rational
expectation}\label{an-illustration-in-a-simple-two-component-income-process-and-rational-expectation}}

The logged income of individual \(i\) (excluding the predictable
component given known information by agents at time \(t\)) follows a
following process (same as \cite{carroll1997nature}).

\begin{equation}
\begin{split}
y_{i,t} = p_{i,t} + \epsilon_{i,t} \\
p_{i,t} = p_{i,t-1} + \theta_{i,t} \\
\theta_{i,t} \sim N(0,\sigma_{\theta,t}) \\
\epsilon_{i,t} \sim N(0,\sigma_{\epsilon,t})
\end{split}
\end{equation}

where \(p_{i,t}\) is a random walk component with a permanent shock
\(\theta_{i,t}\) at each point of time, and \(\epsilon_{i,t}\) is the
transitory shock that is i.i.d.. Note that risks of both transitory and
permanent shocks indicated by their respective variance are
time-varying. For now, we do not break down individuals into different
cohorts, i.e. \(\sigma_{\theta,t}\) and \(\sigma_{\epsilon,t}\) are not
cohort-specific. But we can do this exercise for any defined cohort.

Realized income growth is

\begin{equation}
\begin{split}
\Delta y_{i,t+1} = y_{i,t+1} - y_{i,t} \\
 = p_{i,t+1} + \epsilon_{i,t+1} - p_{i,t} - \epsilon_{i,t} \\
 = \theta_{i,t+1} + \Delta \epsilon_{i,t+1}
\end{split}
\end{equation}

To recover the sizes of the income risks \(\sigma_{\theta,t}\) and
\(\sigma_{\epsilon,t}\), econometricians have conventionally relied upon
following three moment restrictions available from cross-sectional
distributions of realized income growth rates.

\begin{equation}
\label{VarUC}
Var (\Delta y_{i,t+1}) =  \sigma^2_{\theta,t+1} +\sigma^2_{\epsilon,t}+ \sigma^2_{\epsilon,t+1}
\end{equation}

\begin{equation}
\label{AvarUC1}
Cov (\Delta y_{i,t}, \Delta y_{i,t+1}) =  - \sigma^2_{\epsilon,t}
\end{equation}

\begin{equation}
\label{AvarUC2}
Cov (\Delta y_{i,t}, \Delta y_{i,t-1}) =  - \sigma^2_{\epsilon,t-1}
\end{equation}

Using the recursive structure from an income panel of length t+2 periods
(therefore, t+1 periods of \(\Delta y_{i,t}\) are available), the three
moments above could identify income risks \(\sigma_{\theta,t}\) and
\(\sigma_{\epsilon,t}\) for up to \(t\) periods.

The additional information from density forecasts is the conditional
variance that is specifically perceived by \(i\),
\(Var_{i,t}(\Delta y_{i,t+1})\). This is not the same as the
cross-sectional variance defined above (which has no subscript \(i\)).
Under the rational expectation, an agent \(i\) who knows perfectly about
her income process and standing at time \(t\) shall perceive her income
risks to be exactly equal to the conditional variance of income growth
over the next period given the realized shocks up to \(t\).

\begin{equation}
\label{VarCRE}
 Var^*_{i,t}(\Delta y_{i,t+1}) = \sigma^2_{\theta,t+1} + \sigma^2_{\epsilon,t+1} \quad \forall i
\end{equation}

where I use \({}^*\) superscript to represent the rational expectation
version of the moments. For the risks estimated from cross-sectional
distribution from above to truly align with what is perceived by
individual \(i\), we need two assumptions. First, the income shocks for
the entire population (or the defined cohort that shares the same
distribution of income shocks) are exactly as assumed by
econometricians. Second, the agent \(i\) has the rational expectation
and full information.

In order to reflect the observed heterogeneity in perceived risks from
the data, I assume the realized perceived risk has an i.i.d. component
on the top of rational expectation benchmark.

\begin{equation}
\label{VarCIRE}
 Var_{i,t}(\Delta y_{i,t+1}) = \sigma^2_{\theta,t+1} + \sigma^2_{\epsilon,t+1} + \xi_{i,t} \quad \forall i
\end{equation}

The third component \(\xi_t\) is an idiosyncratic shock to the level of
a perceived risk that can be either interpreted as a measurement error
from the point view of the econometricians or a perceptual change to
which econometricians have no access to. Rational expectation assumption
also implies \(\xi_{i,t}\) is mean zero, meaning the agents' perceived
risks on average converge to the summation of the one-period ahead
permanent and transitory risks.

The rational expectation also has following moment restrictions, which
although do not help identification of the risks, do have important
implications once I make alternative assumptions on expectation
formation.

\begin{equation}
Cov^*_{i,t}( \Delta y_{i,t}, \Delta y_{i,t+1} ) \\
 = Cov^*_{i,t}(\theta_{i,t} + \epsilon_{i,t} - \epsilon_{i,t-1}, \theta_{i,t+1} + \epsilon_{i,t+1} - \epsilon_{i,t}) \\
 = 0 
\end{equation}

Conditional on all the information and realized shocks up to \(t\), the
expected earning growth in further future should be orthogonal to the
next period. This is, again, different from an econometrician's problem,
for whom the covariance is \(-\sigma^2_{\epsilon,t}\). The rational
agent in the model learns about \(\sigma_{\epsilon,t}\).

In theory, with available data on cross-sectional realized income
growths and perceived income risks, by combining Equation \ref{VarUC},
\ref{AvarUC1}, \ref{AvarUC2}, and \ref{VarCIRE}, I can jointly estimate
the sizes of permanent and transitory risks as well as the dispersion of
the perceptual shock \(\sigma_{\xi,t}\). The last provides me with one
way to systematically characterize the deviation of the perceived risks
from what econometricians have estimated and used.

    \hypertarget{time-aggregation-problem-in-progress-and-very-preliminary}{%
\subsection{Time aggregation problem (in progress and very
preliminary)}\label{time-aggregation-problem-in-progress-and-very-preliminary}}

The stylized example in the previous section is only valid in the
absence of a time-aggregation problem. It hinges on the assumption that
the unit of the time at which the income process is defined is the same
as the frequency of the income observation is measured. This is,
however, not the case for SCE data. What is available as perceived
income risk is concerning the one-year-ahead income growth, while the
underlying income process is preferably defined at a higher frequency
(i.e.~in monthly and quarterly.) Assuming a higher frequency than annual
income is motivated by two considerations. First, it is more realistic.
Second, it is also meant to be compatible with the frequency at which
the survey is conducted, i.e.~monthly.

    \hypertarget{a-simple-example-with-half-year-as-the-unit-of-the-time}{%
\subsubsection{A simple example with half-year as the unit of the
time}\label{a-simple-example-with-half-year-as-the-unit-of-the-time}}

Earning in year \(t\) is a summation of half-year earning.

\begin{equation}
y_t = y_{t_1}+ y_{t_2} 
\end{equation}

The YoY growth of income is below

\begin{equation}
\begin{split}
\Delta y_{t_2+1} = y_{(t+1)_1}+ y_{(t+1)_2} - y_{t_1 } - y_{t_2}  \\
 = p_{(t+1)_1} + \epsilon_{(t+1)_2} + p_{(t+1)_2} + \epsilon_{(t+1)_2} - p_{t_1} - \epsilon_{t_1} - p_{t_1} - \epsilon_{(t)_2 } \\
 = \theta_{(t)_2} + \theta_{(t+1)_1} + \theta_{(t+1)_2} + \theta_{(t+1)_1} + \epsilon_{(t+1)_1} + \epsilon_{(t+1)_2} - \epsilon_{t_1} - \epsilon_{t_2} \\
 =  \theta_{t_2} + 2\theta_{(t+1)_1} + \theta_{(t+1)_2} + \epsilon_{(t+1)_1} + \epsilon_{(t+1)_2} - \epsilon_{t_1} - \epsilon_{t_2} 
\end{split}
\end{equation}

The middle-year-on-middle-year income growth is

\begin{equation}
\begin{split}
\Delta y_{(t+1)_1+1} = y_{(t+1)_2}+ y_{(t+2)_1} - y_{(t+1)_1} - y_{t_2}  \\
 = p_{(t+1)_2} + \epsilon_{(t+1)_2} + p_{(t+2)_1} + \epsilon_{(t+2)_1} - p_{(t+1)_1} - \epsilon_{(t+1)_1} - p_{t_2} - \epsilon_{t_2 } \\
 = \theta_{(t+1)_2} + \theta_{(t+1)_1} + \theta_{(t+1)_2} + \theta_{(t+2)_1} + \epsilon_{(t+1)_2} + \epsilon_{(t+2)_1} - \epsilon_{(t+1)_1} - \epsilon_{t_2 } \\
 = 2\theta_{(t+1)_2} + \theta_{(t+1)_1} + \theta_{(t+2)_1} + \epsilon_{(t+1)_2} + \epsilon_{(t+2)_1} - \epsilon_{(t+1)_1} - \epsilon_{t_2 }
\end{split}
\end{equation}

Then for each individual \(i\) at \(t''\) and \((t+1)'\) are
respectively:

\begin{equation}
Var^*_{i,t_2}(\Delta y_{i,t_2+1}) =  2\sigma^2_{\theta,(t+1)_1} + \sigma^2_{\theta,(t+1)_2} + \sigma^2_{\epsilon,(t+1)_1} + \sigma^2_{\epsilon,(t+1)_2}
\end{equation}

\begin{equation}
Var^*_{i,(t+1)_1}(\Delta y_{i,(t+1)_1+1}) =  2\sigma^2_{\theta,(t+1)_2} + \sigma^2_{\theta,(t+2)_1} + \sigma^2_{\epsilon,(t+1)_2} + \sigma^2_{\epsilon,(t+2)_1}
\end{equation}

From end of \(t_2\) (end of year \(t\)) to the end of \((t+1)_1\)
(middle of the year \(t+1\)), the realization of \(\theta_{(t+1)_1}\)
and \(\epsilon_{(t+1)_1}\) reduces the variance.

Besides, the econometricians have access to following two
cross-sectional moments.

\begin{equation}
Var (\Delta y_{i,t_2+1}) =  \sigma^2_{\theta,t_2} + 2\sigma^2_{\theta,(t+1)_1} + \sigma^2_{\theta,(t+1)_2} + \sigma^2_{\epsilon,(t+1)_1} + \sigma^2_{\epsilon,(t+1)_2} + \sigma^2_{\epsilon,t_1} + \sigma^2_{\epsilon,t_2} 
\end{equation}

\begin{equation}
Var (\Delta y_{i,(t+1)_1+1}) =  2\sigma^2_{\theta,(t+1)_2} + \sigma^2_{\theta,(t+1)_1} + \sigma^2_{\theta,(t+2)_1} + \sigma^2_{\epsilon,(t+1)_2} + \sigma^2_{\epsilon,(t+2)_1} + \sigma^2_{\epsilon,(t+1)_1} + \sigma^2_{\epsilon,t_2}
\end{equation}

\begin{equation}
\begin{split}
Cov ( \Delta y_{i,(t-1)_2+1},\Delta y_{i,t_1+1}) = Cov(\theta_{(t-1)_2} + 2\theta_{t_1} + \theta_{t_2} + \epsilon_{t_1} + \epsilon_{t_2} - \epsilon_{(t-1)_1} - \epsilon_{(t-1)_2} , \\
2\theta_{t_2} + \theta_{t_1} + \theta_{(t+1)_1} + \epsilon_{t_2} + \epsilon_{(t+1)_1} - \epsilon_{t_1} - \epsilon_{(t-1)_2 } ) \\
= 2\sigma^2_{\theta,t_1} + 2\sigma^2_{\theta,t_2} - \sigma^2_{\epsilon,t_1} + \sigma^2_{\epsilon,t_2} + \sigma^2_{\epsilon,(t-1)_2}
\end{split}
\end{equation}

\begin{equation}
\begin{split}
Cov ( \Delta y_{i,(t-1)_2+1},\Delta y_{i,t_2+1}) = Cov(\theta_{(t-1)_2} + 2\theta_{t_1} + \theta_{t_2} + \epsilon_{t_1} + \epsilon_{t_2} - \epsilon_{(t-1)_1} - \epsilon_{(t-1)_2} , \\
\theta_{t_2} + 2\theta_{(t+1)_1} + \theta_{(t+1)_2} + \epsilon_{(t+1)_1} + \epsilon_{(t+1)_2} - \epsilon_{t_1} - \epsilon_{t_2} ) \\
= \sigma^2_{\theta,t_2}-(\sigma^2_{\epsilon,(t+1)_1} + \sigma^2_{\epsilon,t_2})
\end{split}
\end{equation}

\begin{equation}
\begin{split}
Cov ( \Delta y_{i,t_2+1},\Delta y_{i,(t+1)_2}) = \sigma^2_{\theta,(t+1)_2}-(\sigma^2_{\epsilon,(t+2)_1} + \sigma^2_{\epsilon,(t+1)_2})
\end{split}
\end{equation}

The rational expectation assumption also gives following moment
restrictions

\begin{equation}
Cov^*_{t_2}(\Delta y_t, \Delta y_{t+1}) = 0
\end{equation}

Standing at any point of the time, for the rational agent, the
\(\Delta y_t\) is realizated already. So it should have zero covariance
with income growth in future.

This is again, different from the econometrician's problem.

    \hypertarget{model-in-progress}{%
\section{Model (in progress)}\label{model-in-progress}}

    \hypertarget{summary-and-agenda}{%
\section{Summary and agenda}\label{summary-and-agenda}}

A few robust empirical findings so far.

\begin{itemize}
\tightlist
\item
  Individuals' perceived income risks measured by variance, IQR and tail
  risk measure skewness all exhibit sizable dispersion across different
  individuals. This pattern also holds for real earning risk, when
  adjusted by inflation uncertainty.
\item
  Distributions of perceived risks are consistent with a number of
  intuitive patterns. For instance, earners who are males, from
  higher-income households, and with higher education have statistically
  significantly lower perceived risks.
\item
  The perceived risk is also positively associated with the perceived
  chance of unemployment.
\end{itemize}

Preliminary findings include the following.

\begin{itemize}
\tightlist
\item
  Perceived risk and skewness for future income are negatively
  correlated with stock market returns in the same future horizon. (This
  needs to be more disciplined by the empirical asset pricing
  literature.)
\end{itemize}

To-dos in the next stage of the work

\begin{itemize}
\item
  Empirically, investigate the correlation between perceived risk and
  high-frequency macro variables that can approximate business cycle
  dynamics.
\item
  To decompose the subjective income process by addressing the time
  aggregation problem and missing data on monthly earnings.
\item
  Theoretically, build a life cycle model of consumption/portfolio
  choice with the following features: - heterogeneous perceptions of
  income process, which is micro-founded in a manner in line with the
  empirical patterns. - with endogenous consumption and portfolio choice
  decisions. - with market incompleteness, i.e.~the idiosyncratic risks
  are uninsured.
\end{itemize}

    \hypertarget{appendix}{%
\section{Appendix}\label{appendix}}




       
\begin{table}[p]
\centering
\begin{adjustbox}{width={\textwidth}}
\begin{threeparttable}
\caption{Perceived Income Growth and Individual Characteristics}
\label{micro_reg_exp}\begin{tabular}{ccccccccc}
\toprule
{} &  incexp I & incexp II & incexp III & incexp IIII & rincexp I & rincexp II & rincexp III & rincexp IIII \\
\midrule
HHinc\_gr=low inc &           &           &      -0.03 &             &           &            &    -0.37*** &              \\
                 &           &           &     (0.02) &             &           &            &      (0.03) &              \\
educ\_gr=low educ &           &           &            &    -0.25*** &           &            &             &     -0.62*** \\
                 &           &           &            &      (0.02) &           &            &             &       (0.03) \\
gender=male      &           &           &            &    -0.33*** &           &            &             &     -0.78*** \\
                 &           &           &            &      (0.02) &           &            &             &       (0.03) \\
parttime=yes     &  -0.42*** &  -0.34*** &   -0.33*** &             &  -0.56*** &   -0.47*** &    -0.38*** &              \\
                 &    (0.03) &    (0.03) &     (0.03) &             &    (0.04) &     (0.04) &      (0.04) &              \\
selfemp=yes      &   0.86*** &  -0.00*** &   -0.00*** &             &   0.83*** &   -0.00*** &     0.00*** &              \\
                 &    (0.04) &    (0.00) &     (0.00) &             &    (0.05) &     (0.00) &      (0.00) &              \\
Intercept        &   2.82*** &   2.64*** &    2.65*** &     3.06*** &  -0.28*** &   -0.51*** &    -0.40*** &      0.23*** \\
                 &    (0.01) &    (0.03) &     (0.03) &      (0.02) &    (0.02) &     (0.04) &      (0.04) &       (0.02) \\
                 &      0.01 &      0.02 &       0.02 &        0.01 &      0.01 &       0.04 &        0.05 &         0.02 \\
Stkprob          &           &   0.01*** &    0.01*** &             &           &    0.02*** &     0.02*** &              \\
                 &           &    (0.00) &     (0.00) &             &           &     (0.00) &      (0.00) &              \\
UEprobAgg        &           &  -0.00*** &   -0.00*** &             &           &   -0.01*** &    -0.01*** &              \\
                 &           &    (0.00) &     (0.00) &             &           &     (0.00) &      (0.00) &              \\
UEprobInd        &           &  -0.01*** &   -0.01*** &             &           &   -0.01*** &    -0.01*** &              \\
                 &           &    (0.00) &     (0.00) &             &           &     (0.00) &      (0.00) &              \\
N                &     52102 &     47112 &      47112 &       45745 &     47728 &      43099 &       43099 &        41908 \\
R2               &      0.01 &      0.02 &       0.02 &        0.01 &      0.01 &       0.04 &        0.05 &         0.02 \\
\bottomrule
\end{tabular}
\begin{tablenotes}\item Standard errors are clustered by household. *** p$<$0.001, ** p$<$0.01 and * p$<$0.05. 
\item This table reports regression results of perceived labor income(incexp for nominal, rincexp for real) growth on household specific variables. HHinc: household income group ranges from lowests (=1, less than \$10k/year) to the heightst (=11, greater than \$200k/year). Education, educ ranges from the lowest (=1, less than high school) to the highest (=9).
\end{tablenotes}
\end{threeparttable}
\end{adjustbox}
\end{table}
\begin{table}[p]
\centering
\begin{adjustbox}{width=\textwidth}
\begin{threeparttable}
\caption{Perceived Income Risks and Individual Characteristics}
\label{micro_reg}\begin{tabular}{ccccccccc}
\toprule
{} & incvar I & incvar II & incvar III & incvar IIII & rincvar I & rincvar II & rincvar III & rincvar IIII \\
\midrule
HHinc\_gr=low inc &          &           &    1.52*** &             &           &            &     6.97*** &              \\
                 &          &           &     (0.10) &             &           &            &      (0.20) &              \\
educ\_gr=low educ &          &           &            &     0.44*** &           &            &             &      4.01*** \\
                 &          &           &            &      (0.12) &           &            &             &       (0.22) \\
gender=male      &          &           &            &    -0.82*** &           &            &             &      2.67*** \\
                 &          &           &            &      (0.10) &           &            &             &       (0.19) \\
parttime=yes     &    0.25* &   0.35*** &      -0.02 &             &   1.90*** &    2.23*** &      0.56** &              \\
                 &   (0.13) &    (0.13) &     (0.14) &             &    (0.25) &     (0.27) &      (0.27) &              \\
selfemp=yes      &  7.58*** &     -0.00 &    -0.00** &             &   6.72*** &   -0.00*** &    -0.00*** &              \\
                 &   (0.15) &    (0.00) &     (0.00) &             &    (0.29) &     (0.00) &      (0.00) &              \\
Intercept        &  4.62*** &   3.76*** &    3.30*** &     5.71*** &  12.42*** &   12.26*** &    10.22*** &     11.21*** \\
                 &   (0.05) &    (0.12) &     (0.13) &      (0.08) &    (0.10) &     (0.25) &      (0.25) &       (0.14) \\
                 &     0.05 &      0.00 &       0.01 &        0.00 &      0.01 &       0.01 &        0.04 &         0.01 \\
Stkprob          &          &   0.01*** &    0.01*** &             &           &   -0.06*** &    -0.05*** &              \\
                 &          &    (0.00) &     (0.00) &             &           &     (0.00) &      (0.00) &              \\
UEprobAgg        &          &    0.00** &       0.00 &             &           &    0.05*** &     0.04*** &              \\
                 &          &    (0.00) &     (0.00) &             &           &     (0.00) &      (0.00) &              \\
UEprobInd        &          &   0.03*** &    0.03*** &             &           &    0.05*** &     0.04*** &              \\
                 &          &    (0.00) &     (0.00) &             &           &     (0.00) &      (0.00) &              \\
N                &    51788 &     45903 &      45903 &       45430 &     48665 &      43077 &       43077 &        42654 \\
R2               &     0.05 &      0.00 &       0.01 &        0.00 &      0.01 &       0.01 &        0.04 &         0.01 \\
\bottomrule
\end{tabular}
\begin{tablenotes}\item Standard errors are clustered by household. *** p$<$0.001, ** p$<$0.01 and * p$<$0.05. 
\item This table reports regression results of perceived income risks (incvar for nominal, rincvar for real) on household specific variables. HHinc: household income group ranges from lowests (=1, less than \$10k/year) to the heightst (=11, greater than \$200k/year). Education, educ ranges from the lowest (=1, less than high school) to the highest (=9).
\end{tablenotes}
\end{threeparttable}
\end{adjustbox}
\end{table}
\begin{table}[ht]
\centering
\begin{adjustbox}{width={\textwidth}}
\begin{threeparttable}
\caption{Correlation between Perceived Income Risks and Stock Market Return}
\label{macro_corr}
\begin{tabular}{cccccclll}
\toprule
{} & median:var & median:iqr & median:rvar & median:skew &  mean:var &  mean:iqr & mean:rvar & mean:skew \\
\midrule
0  &      -0.06 &      -0.01 &        0.09 &         nan &      0.05 &      0.12 &      0.13 &      0.08 \\
1  &      -0.13 &      -0.08 &        0.15 &         nan &      0.11 &      0.16 &      0.2* &     -0.02 \\
2  &      -0.12 &      -0.08 &        0.14 &         nan &       0.1 &      0.16 &   0.31*** &     -0.17 \\
3  &      -0.12 &      -0.06 &        0.08 &         nan &      0.03 &      0.09 &    0.26** &     -0.16 \\
4  &      -0.12 &      -0.03 &       -0.01 &         nan &      -0.1 &     -0.06 &      0.07 &     -0.08 \\
5  &     -0.22* &      -0.15 &       -0.12 &         nan &     -0.06 &     -0.05 &      0.01 &     -0.05 \\
6  &      -0.18 &      -0.13 &       -0.17 &         nan &     -0.12 &     -0.12 &       0.0 &     -0.15 \\
7  &    -0.25** &     -0.21* &      -0.3** &         nan &      -0.2 &    -0.23* &     -0.13 &    -0.23* \\
8  &     -0.23* &       -0.2 &     -0.31** &         nan &    -0.24* &   -0.28** &     -0.13 &      -0.2 \\
9  &     -0.22* &      -0.19 &     -0.27** &         nan &  -0.33*** &   -0.4*** &      -0.2 &     -0.17 \\
10 &      -0.17 &      -0.12 &     -0.29** &         nan &  -0.39*** &  -0.44*** &     -0.18 &   -0.26** \\
11 &     -0.23* &       -0.2 &      -0.25* &         nan &   -0.28** &  -0.37*** &     -0.12 &   -0.26** \\
\bottomrule
\end{tabular}
\begin{tablenotes}
\item *** p$<$0.001, ** p$<$0.01 and * p$<$0.05.
\item This table reports correlation coefficients between different perceived income moments(inc for nominal
and rinc for real) at time
$t$ and the monthly s\&p500 return by the end of $t+k$ for $k=0,1,..,11$.
\end{tablenotes}
\end{threeparttable}
\end{adjustbox}
\end{table}
\begin{table}[p]
\centering
\begin{adjustbox}{width={0.9\textwidth}}
\begin{threeparttable}
\caption{Perceived Income Risks and Household Spending}
\label{spending_reg}\begin{tabular}{ccccccll}
\toprule
{} & spending I & spending II & spending III & spending IIII & spending IIIII & spending IIIIII & spending IIIIIII \\
\midrule
Intercept &    3.46*** &     4.23*** &      4.29*** &       3.73*** &        4.67*** &         3.21*** &          4.63*** \\
          &     (0.31) &      (0.07) &       (0.22) &        (0.26) &         (0.31) &          (0.72) &           (0.20) \\
          &       0.00 &        0.00 &         0.00 &          0.00 &           0.00 &            0.00 &             0.00 \\
UEprobAgg &            &             &              &               &                &         0.04*** &                  \\
          &            &             &              &               &                &          (0.02) &                  \\
UEprobInd &            &             &              &               &          -0.01 &                 &                  \\
          &            &             &              &               &         (0.01) &                 &                  \\
incexp    &    0.39*** &             &              &               &                &                 &                  \\
          &     (0.08) &             &              &               &                &                 &                  \\
incskew   &            &             &              &               &                &                 &             0.19 \\
          &            &             &              &               &                &                 &           (0.45) \\
incvar    &            &             &      0.07*** &               &                &                 &                  \\
          &            &             &       (0.02) &               &                &                 &                  \\
rincexp   &            &      -0.04* &              &               &                &                 &                  \\
          &            &      (0.02) &              &               &                &                 &                  \\
rincvar   &            &             &              &       0.07*** &                &                 &                  \\
          &            &             &              &        (0.01) &                &                 &                  \\
N         &      53455 &       48983 &        53171 &         49986 &          52632 &           76531 &            52751 \\
R2        &       0.00 &        0.00 &         0.00 &          0.00 &           0.00 &            0.00 &             0.00 \\
\bottomrule
\end{tabular}
\begin{tablenotes}\item Standard errors are clustered by household. *** p$<$0.001, ** p$<$0.01 and * p$<$0.05. 
\item This table reports regression results of expected spending growth on perceived income risks (incvar for nominal, rincvar for real).
\end{tablenotes}
\end{threeparttable}
\end{adjustbox}
\end{table}       % Add a bibliography block to the postdoc
    
    
\bibliographystyle{apalike}
\bibliography{PerceivedIncomeRisk}

    
\end{document}
