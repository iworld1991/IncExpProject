\documentclass{beamer}
\usepackage{default}
\usepackage{amsmath}
\usepackage{graphicx}
\usepackage{adjustbox}  % Allows for fitting tables into slide
\usepackage{hyperref}
\usepackage{threeparttable}
\usepackage{caption}
%\usepackage{subcaption}
\usepackage{natbib}
\usepackage{adjustbox}
\usepackage{subcaption}
%\usetheme{AnnArbor}
%\usetheme{Antibes}
%\usetheme{Bergen}
%\usetheme{Berkeley}
%\usetheme{Berlin}
%\usetheme{Boadilla}
%\usetheme{boxes}
\usetheme{CambridgeUS}
%\usetheme{Copenhagen}
%\usetheme{Darmstadt}
%\usetheme{default}
%\usetheme{Frankfurt}
%\usetheme{Goettingen}
%\usetheme{Hannover}
%\usetheme{Ilmenau}
%\usetheme{JuanLesPins}
%\usetheme{Luebeck}
%\usetheme{Madrid}
%\usetheme{Malmoe}
%\usetheme{Marburg}
%\usetheme{Montpellier}
%\usetheme{PaloAlto}
%\usetheme{Pittsburgh}
%\usetheme{Rochester}
%\usetheme{Singapore}
%\usetheme{Szeged}
%\usetheme{Warsaw}

% colortheme to choose one 

%\usecolortheme{beaver}
%\usecolortheme{crane}
%\usecolortheme{default}
\usecolortheme{dolphin}
%\usecolortheme{seagull}
%\usecolortheme{seahorse}
%\usecolortheme{whale}


% fond theme to choose one 
%\usefonttheme{structuresmallcapsserif}
%\usefonttheme{structureitalicserif}
%\usefonttheme{structurebold}
\usefonttheme{serif}
%\usefonttheme{professionalfonts}
%\usefonttheme{default}

\title{Perceived Income Risks}


% A subtitle is optional and this may be deleted

\author{Tao Wang \\ Johns Hopkins University}
% - Give the names in the same order as the appear in the paper.
% - Use the \inst{?} command only if the authors have different
%   affiliation.

\date{\today}
% - Either use conference name or its abbreviation.
% - Not really informative to the audience, more for people (including
%   yourself) who are reading the slides online

% This is only inserted into the PDF information catalog. Can be left
% out. 

% If you have a file called "university-logo-filename.xxx", where xxx
% is a graphic format that can be processed by latex or pdflatex,
% resp., then you can add a logo as follows:

% \pgfdeclareimage[height=0.5cm]{university-logo}{university-logo-filename}
% \logo{\pgfuseimage{university-logo}}

% Delete this, if you do not want the table of contents to pop up at
% the beginning of each subsection:
\AtBeginSubsection[]
{
	\begin{frame}<beamer>{Outline}
	\tableofcontents[currentsection]
\end{frame}
}

\begin{document}
	

\begin{frame}
	\titlepage
\end{frame}
\begin{frame}{Outline}
	\tableofcontents
	% You might wish to add the option [pausesections]
\end{frame}


\section{Motivation}

\begin{frame}{Motivation}
	\begin{itemize}
		\item ddddddddd
	\end{itemize}
\end{frame}


\begin{frame}{What this paper does}
	\begin{enumerate}
		\item dddd
	\end{enumerate}
\end{frame}


\begin{frame}{Literature}
\begin{itemize}
	\item ddddd
	\begin{itemize}
		\item dddd
	\end{itemize}
\end{itemize}
\end{frame}



\begin{frame}{Definition and notation}
	\begin{table}[ht]
		\centering
		\label{MomSum}
		\begin{tabular}{ll}
			
			\hline 
			Individual moments                                  & Population moments                             \\
			\hline 
			Mean forecast: $y_{i,t+h|t}$                   & Average forecast: $\bar y_{t+h|t}$                   \\
			Forecast error: $FE_{i,t+h|t}$ & Average forecast error: $\overline{FE}_{t+h|t}$ \\
			Uncertainty: $Var_{i,t+h|t}$         & \textbf{Average uncertainty}:  $\overline{Var}_{t+h|t}$ \\
			& Disagreement:  $\overline{Disg}_{t+h|t}$       \\
			\hline 
		\end{tabular}
	\end{table}
	
\end{frame}


\begin{frame}{Data}
	\begin{table}[]
		\resizebox{\textwidth}{!}{	\begin{tabular}{lll}
				
				\hline 
				& SCE & SPF        \\
				\hline 
				Time period                                    & 2013M6-2018M6                            & 2007Q1-2018Q4             \\
				Frequency                                      & Monthly                                 & Quarterly                \\
				Sample Size                                    & 1,300                                   & 30-50                    \\
				Aggregate Var in Density                       & \textcolor{blue}{1-yr-ahead inflation}          & \textcolor{blue}{1-yr and 3-yr core CPI and core PCE}         \\
				Pannel Structure                               & stay up to 12 months                    & average stay for 5 years \\
				Demographic Info                        & Education, Income, Age        & Industry    \\
				\hline 
		\end{tabular}}
	\end{table}
\begin{itemize}
	\item density estimation following (\citet{engelberg2009comparing})
	\item exclude top and bottom 5\% values for forecast errors and uncertainty
\end{itemize}
\end{frame}


\begin{frame}{Basic patterns: uncertainty and realized inflation}
	\begin{figure}
		\centering
		\label{InfVar}
		\includegraphics[width=0.3\textwidth]{figuresDraft/Inf1yf_CPIAU_varSPFCPIQ.png}
		\includegraphics[width=0.3\textwidth]{figuresDraft/Inf1yf_PCE_varSPFPCEQ.png}
		\includegraphics[width=0.3\textwidth]{figuresDraft/Inf1yf_CPIAU_varSCEM.png}
	\end{figure}

\end{frame}


\begin{frame}{Basic patterns: uncertainty and the size of forecast errors}
	\begin{figure}
		\centering
		\label{FEVar}
		\includegraphics[width=0.3\textwidth]{figuresDraft/SPFCPI_abFE_varSPFCPIQ.png}
		\includegraphics[width=0.3\textwidth]{figuresDraft/SPFPCE_abFE_varSPFPCEQ.png}
		\includegraphics[width=0.3\textwidth]{figuresDraft/SCE_abFE_varSCEM.png}
	\end{figure}
\begin{itemize}
\item no evidence for positive correlation betwen high ex ante uncertainty and ex post forecast errors.
\end{itemize}
\end{frame}

\begin{frame}{Basic patterns: uncertainty and disagreement}
	\begin{figure}
		\centering
		\label{DisgVar}
		\includegraphics[width=0.3\textwidth]{figuresDraft/CPI_disg_varSPFCPIQ.png}
		\includegraphics[width=0.3\textwidth]{figuresDraft/PCE_disg_varSPFPCEQ.png}
		\includegraphics[width=0.3\textwidth]{figuresDraft/Q9_disg_varSCEM.png}
	\end{figure}
\begin{itemize}
	\item uncertainty are not the same as disagreement for professionals 
\end{itemize}
\end{frame}

\begin{frame}{Basic patterns: summary}
	\begin{itemize}
		\item uncertainty varies across time 
		\item uncertainty contains different information from widely  proxies such as disagreement and forecast error
	\end{itemize}
\end{frame}

\section{Theory}

\begin{frame}{AR(1) model of inflation}
	

\begin{itemize}
	
\item 	\textbf{Inflation process}	
	$$y_{t} = \rho y_{t-1} + \omega_t  $$ 
	$$\omega_t \sim N(0,\sigma^2_{\omega})$$

\item \textbf{Uncertainty}
		\begin{itemize}
			\item FIRE: time-invariant 
			\begin{eqnarray*}\label{VarREPop}
				\overline{Var}^*_{t+h|t} = \sum^{h}_{s=1}\rho^{2s} \sigma^2_{\omega}
			\end{eqnarray*}
			
			\item SE: time-invariant 
			\begin{eqnarray*}\label{VarSEPopRE}
				\overline{Var}^{se}_{t+h|t} = \sum^{+\infty}_{\tau =0} \lambda (1-\lambda)^\tau \overline{Var}^*_{t+h|t-\tau}
			\end{eqnarray*}
			
			\item NI: time-variant but quantitatively tiny due to highly efficient Kalman gain
			
			\begin{eqnarray*}\label{VarNIPopRE}
				\overline{Var}^{ni}_{t+h|t} = \rho^{2h} \overline{Var}^{ni}_{t|t} + \overline{Var}^*_{t+h|t}
			\end{eqnarray*}
		\end{itemize}
		
	\end{itemize}
	
\end{frame}

\begin{frame}{Stocastic volatility (UCSV) inflation process (\citet{stock2007has})}

\begin{itemize}
\item \textbf{Inflation process}
\begin{eqnarray*}
\begin{split}
& y_t = \theta_t + \eta_t,\quad \textrm{where } \eta_t =\sigma_{\eta,t} \xi_{\eta,t} \\
& \theta_t = \theta_{t-1} + \epsilon_t, \quad \textrm{where }  \epsilon_t =\sigma_{\epsilon,t} \xi_{\epsilon,t} \\
& \log\sigma^2_{\eta,t} = \log\sigma^2_{\eta,t-1} + \mu_{\eta,t} \\
& \log\sigma^2_{\epsilon,t} = \log\sigma^2_{\epsilon,t-1} + \mu_{\epsilon,t} 
\end{split}
\end{eqnarray*}

\begin{eqnarray*}
\begin{split}
& \xi_t =[\xi_{\eta,t},\xi_{\epsilon,t}] \sim N(0,I_2) \\
& \mu_{t} = [\mu_{\eta,t},\mu_{\epsilon,t}]' \sim N(0,\gamma I_2) 
\end{split}
\end{eqnarray*}
\end{itemize}

\end{frame}


\begin{frame}{UCSV inflation process}
	
	\begin{itemize}
		\item \textbf{Uncertainty}
		\begin{itemize}
			\item FIRE: time-varying  \begin{eqnarray*}\label{VARRESVPop}
			\begin{split}
			\overline{Var}^*_{t+h|t} = \sum_{k=1}^h exp^{- 0.5k\gamma_{\eta}} \sigma^2_{\eta,t}  +  exp^{- 0.5 h \gamma_{\epsilon}} \sigma^2_{\epsilon,t} 
			\end{split} 
			\end{eqnarray*}
			
			\item SE: time-varying 
			\begin{eqnarray*}
		\overline {Var}^{se}_{t+h|t} = \sum^{\infty}_{\tau=0} (1-\lambda)^\tau\lambda\overline{Var}^*_{t+h|t-\tau} 
			\end{eqnarray*}
			
			\item NI (1-step-ahead): time-varying 
		\begin{eqnarray*}
		\overline{Var}^\theta_{t|t-1} = \overline{Var}^\theta_{t-1|t-1} + Var^*_{t|t-1}(y_t) 
		\end{eqnarray*}
			
		\end{itemize}
	\end{itemize}
	
\end{frame}


\section{Estimation}
\begin{frame}{Simulated method of moment estimation}
	
	\begin{eqnarray*}
	\widehat \Omega = \underset{\{\Omega \in \Gamma\} }{argmin} (M_{\textrm{data} } - F^{o}(\Omega, Y)) W  (M_{\textrm{data} } - F^{o}(\Omega, Y))'
	\end{eqnarray*}
	
	\begin{itemize}
		\item  $\Omega$: parameters of the particular $o \in \{{fire}, {se}, {ni} \} \times \{ar, sv\}$
		\item  $\Gamma$: constraints for the parameter. 
		\item $M_{data}$: data moments
		\item $F$: simulated model moments according to a particular theory $o$, a function of parameters $\Omega$ as well as the $Y$, the real-time data (including history) up till each point of the time $t$. 
		\begin{itemize}
			\item unconditional moments, not specific to time
			\item moments selected from average forecast, variance and autocovariance of forecasts, average diagreement, variance and autovariance of disagreement, average uncertainty, etc. 
		\end{itemize}
		\item  $W$: weight matrix, identity matrix for now 
	\end{itemize}
\end{frame}


\begin{frame}{Estimation procedure and algorithm}
	\begin{enumerate}
		\item for each theory of expectation formation and the inflation process, start with an initial value for the parameter(s) of interest
		\item simulate individual forecasts for a large enough ($N=200$) number of forecasters
		\item compute the average forecast errors, disagreement and average uncertainty across all agents
		\item compute the time-series moments of the average forecast, disagreement, and uncertainty
		\item compute the difference between the simulated moments and the data moments 
		\item keep searching the parameter value until reaching below a threshold of the loss
	\end{enumerate}
\end{frame}


\begin{frame}{Two-step and joint estimation}
	\begin{enumerate}
		\item two-step estimation:  separately estimate inflation process parameters and then parameters of the inflation process
		\begin{itemize}
			\item pros: computationally lighter 
			\item cons: potential misspecification. does not utilize the expectation data to understand inflation process per se.  
		\end{itemize}
		\item joint estimation: targeting both moments of realized inflation series and moments of forecasts to simultaneously estimate both the inflation process and the parameter of expectation formation
		\begin{itemize}
			\item pros: additional information gain from expectations data about inflation process itself
			\item cons: more computation burden  
		\end{itemize}
	\end{enumerate}
\end{frame}


\subsection{AR(1)}

%%%%%%%%%%%%%%%%%%%
\subsubsection{SE}

\begin{frame}{SE parameter estimate: professionals}
	\begin{table}
		\centering
		\caption{SMM Estimates of SE: professionals}
		\label{SMM_Est_SE_SPF_Table}
		\adjustbox{max height=0.5\textheight, max width=\textwidth}{ 
		\begin{tabular}{lllllllllllll}
			\hline 
			0     & 1       & 2       & 3       & 4  & SE: $\hat\lambda_{SPF}$(Q) & SE: $\hat\lambda_{SPF}$(Q) & SE: $\rho$ & SE: $\sigma$ & SE: $\hat\lambda_{SCE}$(M) & SE: $\hat\lambda_{SCE}$(M) & SE: $\rho$ & SE: $\sigma$ \\
			\hline 
			FEVar & FEATV   &         &         &    & 0.47                       & 0.36                       & 1.00       & 0.08         & 0.2                        & 0.59                       & 0.99       & 0.08         \\
			FEVar & DisgATV & DisgVar &         &    & 0.27                       & 0.38                       & 1.00       & 0.11         & 0.2                        & 0.56                       & 0.98       & 0.08         \\
			FEVar & FEATV   & DisgVar & DisgATV &    & 0.47                       & 0.36                       & 1.00       & 0.10         & 0.2                        & 0.59                       & 0.99       & 0.08         \\
			FEVar & FEATV   & DisgVar & DisgATV & FE & 0.47                       & 0.36                       & 1.00       & 0.08         & 0.2                        & 0.59                       & 0.99       & 0.08       \\
			\hline  
		\end{tabular}
	}
	\end{table}
	\begin{itemize}
		\item $\lambda$: update rate in SE
	\end{itemize}
\end{frame}



\begin{frame}{Professionals and SEAR}
	\begin{figure}[ht]
		\label{SE_diag_SPF}
		\begin{subfigure}[b]{0.2\textwidth}
			\centering
			\caption{FE}
			\includegraphics[width=\textwidth, height = 0.8\textheight]{figuresDraft/spf_se_est_diag0.png}
		\end{subfigure}
		\hfill
		\begin{subfigure}[b]{0.2\textwidth}
			\caption{Disg}
			\includegraphics[width=\textwidth, height = 0.8\textheight]{figuresDraft/spf_se_est_diag1.png}
		\end{subfigure}
		\hfill
		\begin{subfigure}[b]{0.2\textwidth}
			\caption{FE/Disg}
			\includegraphics[width=\textwidth, height = 0.8\textheight]{figuresDraft/spf_se_est_diag2.png}
		\end{subfigure}
	\end{figure}
\end{frame}


%%%%%%%%%%%%%%%%%%%%%%%%%%%%%%%%%



\section{Conclusion}

\begin{frame}{Conclusion}
	\begin{itemize}
		\item ddddd
	\end{itemize}	
\end{frame}


\bibliographystyle{apalike}
\bibliography{PerceivedIncomeRisk}


\end{document}
