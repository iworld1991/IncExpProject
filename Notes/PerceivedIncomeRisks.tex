\documentclass[]{article}
 \usepackage{hyperref}
 \hypersetup{
 	colorlinks=true,
 	linkcolor=blue,
 	filecolor=blue,      
 	urlcolor=blue,
 	citecolor=blue
 }
 \usepackage{natbib}
 
 
%opening
\title{Perceived Income Risks}
\author{Tao Wang}

\begin{document}

\maketitle

\begin{abstract}
This is a research proposal. The paper attempts to explore the heterogeneity in perceived income risks by directly using the density surveys of households. After establishing patterns of the hetergeneity in perceived income risks, it will also look into its impacts on economic decisions. In the last, the discovered patterns of perceived income risks can be incoporated into an otherwise standrad life-cycle models with heterogeneous agents to examine their macroeconomic implications. 
 
\end{abstract}

\section{Introduction}


Even two agents share the same mean realized income, the difference in magnitude of income risks, have a non-trivial impacts on the decisions of the agents. Precautionary saving arises when agents are faced with a mean-preserving spread of income compared to its certainty counterpart. 

Expectation data, especially the directly estimated perceived income risks from surveys provide additional moments for identification. It allows for differentiating insurence from information (\citet{kaufmann2009disentangling}, \citet{meghir2011earnings}). For instance, the well-known empirical finding of excessive sensitivity may be either due to the limited insurance, or due to the unexpected nature of the shocks. What economists typically do is to intereprete the empirical evidence via the first by making rationality assumptions about the second. 

This paper is related to three lines of literature. First, the literature that studies the expectations of economic agents. Most of this literature focues on the firs moment of the expectations, i.e. mean value. The availability of income risks allows for studying the second moments of income, i.e. variances, namely income risks. The most relevant work includes \citet{rozsypal2017overpersistence}, where the authors compare expected income growth with realized income growth in the survey data, finding evidence for what is called ``overpersistence bias'' by agents. 

Second, the literature on expectation formation in general. This paper focuses on micro variable instead of macro. Relative to macro variables, idiosyncratic variables are more generally more relevant to individual decisions. 

Third, the literature that develped under the standard onconsumption/saving models with uninsured income risks. 


\bibliographystyle{apalike}
\bibliography{PerceivedIncomeRisks}

\end{document}
